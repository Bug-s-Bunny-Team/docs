\documentclass{classes/base}

\title{Verbale interno}
\date{2022/03/09}
\author{Nomone Cognomone}

\renewcommand{\maketitle}{
    \begin{titlepage}
    \begin{center}
        \makeatletter
        \vspace*{\fill}
        
        %\includegraphics[width=2.5cm]{assets/unipd}
        %\subsection*{Università degli Studi di Padova}
        %\vspace{2cm}
        
        \begin{minipage}[]{0.3\textwidth}
            \centering
            \includegraphics[width=3cm]{assets/unipd}
            \bigskip
        \end{minipage}
        \begin{minipage}[]{0.7\textwidth}
            \centering
            \color[HTML]{B5121B}{
                \textbf{Università degli Studi di Padova} \\
                Ingegneria del Software \\
                Anno Accademico: 2021/2022 \\
                }
                \vspace*{2cm}
        \end{minipage}
        

        \includegraphics[width=5cm]{assets/logo}

        \Huge
        \textbf{\teamname}
        
        \vspace{3cm}
        
        \Huge
        \textbf{\@title}

        \Large
        \@date

        \vspace{3cm}
        
        \textbf{Redazione:} \@author\\
        \textbf{Verifica:} \@verificatore\\
        \textbf{Approvazione:} \@approvatore\\
        
        \vfill
        \makeatother
    \end{center}
\end{titlepage}

}

\begin{document}
    \maketitle

    \section{Generalità}
    \begin{itemize}
        \item \textbf{Ora inizio:} 15.00
        \item \textbf{Ora fine:} 17.00
    \end{itemize}

    \section{Resoconto}
    E' stata confermata la scelta del capitolato precedentemente favorito, vista anche l’assenza di conflitti con gli altri gruppi.
    E' stata fatta una discussione generale sui documenti da redigere in vista della candidatura, di come iniziare la stesura del progetto.

    Inoltre, è stato deciso di intraprendere un incontro con i proponenti dei rispettivi capitolati C4 e C3.
    
    Sono state stilate delle domande volte a soddisfare i dubbi emersi riguardanti la proposta del capitolato C4, le quali:
    \begin{itemize}
        \item  Il repository è fornito o dobbiamo crearlo?
        \item  Obbligatori lo sviluppo di app native mobile o si può usare un webapp?
        \item  I linguaggi da utilizzare vengono decisi a monte?
        \item  SI dovranno utilizzare solamente API fornite da TikTok/Instagram o si possono usarne di terze parti?
        \item  Nella scelta del WOW è consigliato utilizzare quello presente nel pdf?
        \item  Come vengono forniti gli applicati da usare (lambda chain, AWS, ...)?
        \item  Viene fornita formazione da parte dell'azienda? Se sì, come?
        \item  Verrà fornito supporto per la documentazione iniziale?
    \end{itemize}

    

\end{document}
