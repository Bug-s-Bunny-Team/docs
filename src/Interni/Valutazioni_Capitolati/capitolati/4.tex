\subsubsection{Descrizione:}
Il capitolato propone lo sviluppo di una piattaforma simile ad una guida Michelin, basandosi sulle esperienze che vengono condivise sui social network Instagram e TikTok.
La richiesta prevede che la piattaforma sia in grado di ispezionare ed estrarre determinate informazioni quali immagini, audio o commenti relativi al contenuto analizzato, dalle storie dei relativi social network.
L'obiettivo è quello di riuscire a formare una mappa di location e determinare se quest'ultime vengono recensita negativamente o positivamente, e a tal scopo stilare un ranking di esse incrociando ciò che viene analizzato dalla piattaforma con altre classifiche per rendere omogeneo il risultato.
\subsubsection{Tecnologie:}
Il committente raccomanda l'utilizzo della tecnologia Amazon Web Services (AWS), in particolare i servizi di:
\begin{itemize}
	\item \textbf{AWS fargate:} servizio serverless per gestione container;
	\item \textbf{AWS appsync:} servizio gestito per lo sviluppo rapido di API GraphQL;
	\item \textbf{Neptune :} database a grafo ideale per tracciare le relazioni tra i dati.
	
	Inoltre raccomanda linguaggi di programmazione come:
	\item \textbf{NodeJS:} ideale per lo sviluppo di API Restful JSON a supporto dell'applicativo.
	\item \textbf{Swift:} linguaggio di programmazione per lo sviluppo di app in ambito iOS/MacOS
	\item \textbf{Kotlin:} linguaggio di programmazione per lo sviluppo di app in ambito Android.
	\item \textbf{React / Angular:} interfaccia web.
\end{itemize}

\subsubsection{Vincoli Generali:}
Svolgere analisi sulle API social dei rispettivi social network Instagram e TikTok per identificare il miglior approccio per la raccolta ed analisi delle informazioni, al fine di:
\begin{itemize}
	\item Creare un sistema di crawling efficiente;
	\item Valutare eventuali strategie VoicetoText se le informazioni raccolte non sono sufficienti;
	\item Identificare le tecnologie cloud adeguate per questo tipo di attività;
	\item Sviluppo Mobile App (iOS e Android);
	\item Architettura basata su micro-servizi; 
\end{itemize}
Quest'ultima in particolare prevede di suddividere il progetto in tante funzioni di base (…) Pertanto i singoli servizi possono funzionare, o meno, senza compromettere gli altri.

\subsubsection{Fattori Critici:}
\begin{itemize}
	\item Utilizzo di nuove tecnologie;
	\item Analisi lunga e approfondita per capire se il progetto è praticabile.
\end{itemize}

\subsubsection{Aspetti Positivi:}
\begin{itemize}
	\item Formazione sulle tecnologie usate;
	\item Grande disponibilità da parte dell'azienda;
	\item Fornitura spazi di lavoro(in sede);
	\item Progetto moderno e innovativo; 
\end{itemize}

\subsubsection{Conclusioni:}
Dopo un incontro tra i membri del gruppo, è stato deciso questo capitolato valutandolo come il più interessante e promettente.