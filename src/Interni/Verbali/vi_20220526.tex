\documentclass{classes/base}

\title{Verbale interno}
\date{2022/05/26}
\author{\marcob}
\verificatore{\angela}
\approvatore{\marcov}

\renewcommand{\maketitle}{
    \begin{titlepage}
    \begin{center}
        \makeatletter
        \vspace*{\fill}
        
        %\includegraphics[width=2.5cm]{assets/unipd}
        %\subsection*{Università degli Studi di Padova}
        %\vspace{2cm}
        
        \begin{minipage}[]{0.3\textwidth}
            \centering
            \includegraphics[width=3cm]{assets/unipd}
            \bigskip
        \end{minipage}
        \begin{minipage}[]{0.7\textwidth}
            \centering
            \color[HTML]{B5121B}{
                \textbf{Università degli Studi di Padova} \\
                Ingegneria del Software \\
                Anno Accademico: 2021/2022 \\
                }
                \vspace*{2cm}
        \end{minipage}
        

        \includegraphics[width=5cm]{assets/logo}

        \Huge
        \textbf{\teamname}
        
        \vspace{3cm}
        
        \Huge
        \textbf{\@title}

        \Large
        \@date

        \vspace{3cm}
        
        \textbf{Redazione:} \@author\\
        \textbf{Verifica:} \@verificatore\\
        \textbf{Approvazione:} \@approvatore\\
        
        \vfill
        \makeatother
    \end{center}
\end{titlepage}

}

\begin{document}
    \maketitle

    \section*{Generalità}
    \begin{itemize}
        \item \textbf{Ora inizio:} 14.30
        \item \textbf{Ora fine:} 17.30
    \end{itemize}

    \section*{Resoconto}
    Deciso sistema per meglio tracciare verifica ed approvazione dei documenti: una volta 
    che un documento è completato viene effettuata una pull request dal branch nel quale è stato
    scritto (es: \textit{verbali} se si tratta di un verbale) verso il branch \textit{dev}. \\
    A questo punto il verificatore controlla il documento e, una volta marchiato come verificato,
    l'approvatore chiude la pull request completando il merge.\\
    \\
    Introduzione branch \textit{dev}: un \textit{master} di secondo livello, da intendersi esattamente come il
    \textit{master} branch però con una versione contenuta al suo interno che è ufficiale ma in sviluppo.
    Una volta arrivati alla revisione si andrà a fare il merge tra \textit{dev} e \textit{master}.\\
    Preferibile il sistema \textit{squash and merge} così da accorpare i commits.\\
    \\
    Si è poi passati a discussione e stesura delle sezioni \textit{Pianificazione} e \textit{Preventivi}
    del \PdP.\\
    Infine il gruppo ha deciso di fermare i lavori durante la settimana appena precedente allo
    scritto di SWE.
\end{document}