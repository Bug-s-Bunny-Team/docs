\documentclass{classes/base}

\title{Verbale interno}
\date{2022/08/23}
\author{\angela}
\verificatore{\giulio}
\approvatore{Nome o comando Latex}

\renewcommand{\maketitle}{
    \begin{titlepage}
    \begin{center}
        \makeatletter
        \vspace*{\fill}
        
        %\includegraphics[width=2.5cm]{assets/unipd}
        %\subsection*{Università degli Studi di Padova}
        %\vspace{2cm}
        
        \begin{minipage}[]{0.3\textwidth}
            \centering
            \includegraphics[width=3cm]{assets/unipd}
            \bigskip
        \end{minipage}
        \begin{minipage}[]{0.7\textwidth}
            \centering
            \color[HTML]{B5121B}{
                \textbf{Università degli Studi di Padova} \\
                Ingegneria del Software \\
                Anno Accademico: 2021/2022 \\
                }
                \vspace*{2cm}
        \end{minipage}
        

        \includegraphics[width=5cm]{assets/logo}

        \Huge
        \textbf{\teamname}
        
        \vspace{3cm}
        
        \Huge
        \textbf{\@title}

        \Large
        \@date

        \vspace{3cm}
        
        \textbf{Redazione:} \@author\\
        \textbf{Verifica:} \@verificatore\\
        \textbf{Approvazione:} \@approvatore\\
        
        \vfill
        \makeatother
    \end{center}
\end{titlepage}

}

\begin{document}
    \maketitle

    \section*{Generalità}
    \begin{itemize}
        \item \textbf{Ora inizio:} 14.30
        \item \textbf{Ora fine:} 15.00
    \end{itemize}

    \section*{Assenze}
    \begin{itemize}
        \item \marcob{} (malattia)
        \item \ruth{} (motivi personali)
    \end{itemize}

    \section*{Agenda}
    Chiarire il punto del gruppo backend e frontend, con i rispettivi documenti

    \section*{Resoconto}
    Il gruppo ha cercato di capire la natura di alcuni bug della piattaforma, pensando a delle possibili soluzioni. 
    Quindi ogni sottogruppo (backend e frontend) lavorerà per risolvere tali problemi. 
    Si è fatto un punto della situazione sui seguenti documenti:
    \begin{itemize}
        \item \PdP: quando le ore di ogni componente saranno terminate verrà steso il consuntivo di periodo;
        \item \NdP: stendere le norme di codifica; 
        \item \PdQ: aggiornare i test eseguiti;
        \item \textit{Manuale Utente}: inserire le foto per spiegare i casi in modo più dettagliato e ampio;
        \item \textit{Specifica Architetturale}: concluderlo, aggiungendo gli ultimi dati mancanti.
    \end{itemize}
 


    \section*{Issues aperte}
    {
        
        \newlength{\freewidth}
        \setlength{\freewidth}{\dimexpr\textwidth-10\tabcolsep}
        \renewcommand{\arraystretch}{1.5}
        \centering
        \setlength{\aboverulesep}{0pt}
        \setlength{\belowrulesep}{0pt}
        \rowcolors{2}{Arancione!10}{white}
        \begin{longtable}{C{.13\freewidth} C{\freewidth}}
            \toprule
        \rowcolor{Arancione}
        \textcolor{white}{\textbf{Codice}}&
        \textcolor{white}{\textbf{Descrizione}}\\	
        \toprule
        \endhead
        
        \#116 & Stesura verbale 31 agosto \\

        \bottomrule
        \end{longtable}
    }

\end{document}
