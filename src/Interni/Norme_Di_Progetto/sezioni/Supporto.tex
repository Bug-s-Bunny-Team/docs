\subsection{Documentazione}
	\subsubsection{Scopo}
	Il processo di documentazione comprende le attività di stesura e aggiornamento di tutti i documenti creati durante il ciclo di vita del \emph{software} in modo da renderli formalmente concordi. In particolare, in questa sezione ne vengono normate le attività, comprendenti stesura, \emph{verifica}\G{} e \emph{validazione}\G. 
	
	\subsubsection{Aspettative}
	Durante questo processo il \emph{team} ha le seguenti aspettative:
		\begin{itemize}
			\item Ideare una struttura ben organizzata comune a tutti i documenti;
			\item Stesura di norme per facilitare tale processo.
		\end{itemize}
	
	\subsubsection{Documenti prodotti}
		I documenti prodotti saranno di due tipi:
			\begin{itemize}
				\item \textbf{Formali}:
				\begin{itemize}
					\item \textbf{interni}: riguardanti le dinamiche del gruppo;
					\item \textbf{esterni}: di interesse ai committenti e al \emph{proponente}.
				\end{itemize} 
				In particolare, i documenti formali prodotti saranno:
				
				{
				\setlength{\freewidth}{\dimexpr\textwidth-1\tabcolsep}
				\renewcommand{\arraystretch}{1.5}
				\centering
				\setlength{\aboverulesep}{0pt}
				\setlength{\belowrulesep}{0pt}
				\rowcolors{2}{Arancione!10}{white}
				\begin{longtable}{C{.24\freewidth} C{.58\freewidth} C{.12\freewidth}}
					\toprule
					\rowcolor{Arancione}
					\textcolor{white}{\textbf{Nome}}&
					\textcolor{white}{\textbf{Descrizione}}&
					\textcolor{white}{\textbf{Uso}}\\	
					\toprule
					\endhead
					\NdP{} & Contiene tutte le regole stabilite dai membri alle quali attenersi durante l'intera durata del progetto & Interno \\	
					\VdC{} & Contiene l'analisi dei capitolati messi a disposizione, evidenziandone pregi e difetti. Contiene, inoltre, il capitolato scelto dal gruppo & Interno \\
					\AdR{} & Descrive i \emph{requisiti} che il prodotto dovrà possedere per essere in linea con le richieste dei committenti & Esterno \\
					\PdP{} & Contiene la pianificazione di tutte le attività previste, comprendente il preventivo delle spese e una previsione dell'impegno in ore per ogni membro del gruppo & Esterno \\
					\PdQ{} & Descrive i criteri di valutazione della qualità impiegate dal gruppo & Esterno \\
					\Glo{} & Elenco di tutti i termini presenti nella documentazione che, secondo i membri, necessitano di una definizione al fine di chiarirne il significato e rimuovere eventuali ambiguità & Esterno \\
					\bottomrule
					\rowcolor{white}
					\caption{Tabella dei documenti formali prodotti}
				\end{longtable}
			}
			\item \textbf{Informali}: 
			In particolare, i documenti informali prodotti saranno:
			{
				\setlength{\freewidth}{\dimexpr\textwidth-1\tabcolsep}
				\renewcommand{\arraystretch}{1.5}
				\centering
				\setlength{\aboverulesep}{0pt}
				\setlength{\belowrulesep}{0pt}
				\rowcolors{2}{Arancione!10}{white}
				\begin{longtable}{C{.24\freewidth} C{.58\freewidth} C{.12\freewidth}}
					\toprule
					\rowcolor{Arancione}
					\textcolor{white}{\textbf{Nome}}&
					\textcolor{white}{\textbf{Descrizione}}&
					\textcolor{white}{\textbf{Uso}}\\	
					\toprule
					\endhead
					\emph{Verbali interni} & Contengono le informazioni e le decisioni prese durante gli incontri tra i membri del gruppo & Interno \\	
					\emph{Verbali esterni} & Comprendono le informazioni ed i chiarimenti ricevuti durante gli incontri tra i membri ed il committente o tra i membri ed il \emph{proponente} & Esterno \\
					\bottomrule
					\rowcolor{white}
					\caption{Tabella dei documenti informali prodotti}
				\end{longtable}
			}
		\end{itemize}
	
	\subsubsection{Directory di un documento}
	Le directory prendono il nome dal documento contenuto, nella forma \textbf{Nome\_Del\_Documento}. Questa viene poi collocata, a seconda del tipo di contenuto, in una delle directory \textbf{Interni} o \textbf{Esterni}. 
	
	\subsubsection{Ciclo di vita di un documento}
		Ogni documento passa per le seguenti fasi:
			\begin{itemize}
				\item \textbf{Stesura}: il documento viene creato, aggiornato e modificato fino ad \emph{Approvazione};
				\item  \textbf{Verifica}: il documento viene verificato sia dal punto di vista grammaticale che contenutistico; la verifica viene effettuata da uno o più \verf{};
				\item \textbf{Approvazione}: il documento viene approvato da un \emph{Approvatore} che ne consente il rilascio.
			\end{itemize}
	\subsubsection{Struttura dei documenti}
		\paragraph*{Frontespizio}
		\aCapo{}    
			La prima pagina di ogni documento sarà così strutturata:
				\begin{itemize}
					\item \textbf{Logo Università e didascalia}: posizionato in alto a sinistra e al suo fianco, al centro, la scritta "Università degli studi di Padova", corso del progetto e anno accademico;
					\item \textbf{Logo gruppo e nome}: entrambi al centro, il nome del gruppo è sotto il logo;
					\item \textbf{Titolo del documento};
					\item \textbf{Recapito email del gruppo};
					\item \textbf{Informazioni del documento}: Redattore, Verificatore, Approvatore ed Uso(interno od esterno).
				\end{itemize}
			
		\paragraph*{Registro delle modifiche}
		\aCapo{}  
			Ogni documento formale presenta subito dopo il frontespizio il registro modifiche, dove ogni componente del gruppo è tenuto alla stesura, pressochè immediata, delle modifiche apportate.
			Esso è una tabella che traccia tutte le modifiche significative apportate al documento durante il suo ciclo di vita. Ogni voce della tabella riporta:
				\begin{itemize}
					\item \textbf{Versione}: numero versione documento dopo la modifica;
					\item \textbf{Data}: data in cui è stata effettuata la modifica;
					\item \textbf{Nominativo}: nome e cognome dell'autore della modifica;
					\item \textbf{Ruolo}: ruolo dell'autore che ha apportato la modifica;
					\item \textbf{Descrizione}: sintetica descrizione delle modifiche apportate;
				\end{itemize}
	
		\paragraph*{Corpo del documento}      
		\aCapo{}   
			Tutte le pagine del corpo del documento contengono un intestazione composta da:
				\begin{itemize}
					\item \textbf{Logo Gruppo}: a sinistra;
					\item \textbf{Titolo Documento}: a destra;
				\end{itemize}
			e un piè di pagina al cui centro c'è la pagina corrente del documento.
			
		\paragraph*{Verbali} %spiegare nomenclatura
		\aCapo{}  
			I verbali, sia interni che esterni, presentano una struttura fissa e a differenza degli altri documenti, essendo informali, non sono soggetti a versionamento e non riportano il registro delle modifiche.
			La struttura sarà quindi la seguente:
				\begin{itemize}
					\item \textbf{Logo Università e logo gruppo};
					\item \textbf{Uso}: indica se il verbale è esterno o interno;
					\item \textbf{Data}: data della riunione;
					\item \textbf{Informazioni del documento}: Redattore, Verificatore ed Approvatore;
					\item \textbf{Orari:} ora di inizio e ora di fine della riunione;
					\item \textbf{Resoconto della riunione:} Riporta l'esito della discussione dei singoli argomenti trattati.								
				\end{itemize}
	
	\subsubsection{Norme tipografiche}
		\paragraph*{Convenzioni di denominazione}   % serve ?
		\aCapo{}  
			I nomi dei file relativi alla documentazione presentano l'iniziale della prima e terza parola che lo compone in maiuscolo e presenta separazione tra le parole. 
		\paragraph*{Stili di testo} % Ha senso?
			\begin{itemize}
				\item \textbf{Grassetto}: viene usato nei titoli delle sezioni e dei paragrafi e per enfatizzare parole;
				\item \textbf{Corsivo}: viene usato per i nomi propri (membri del gruppo,  proponente e committenti) e per citare il nome di un documento.
			\end{itemize}
	
		%Aggiungere Paragrafo Tabelle?
		\paragraph*{Convenzioni scrittura liste puntante} \aCapo{}
		In una lista senza ulteriori liste innestate gli elementi si scrivono con la prima lettera
		maiuscola, sia che siano in \textbf{grassetto}, sia che siano in \textit{corsivo}, sia che non
		abbiano alcun tipo di formattazione. Gli elementi vanno separati con un \textit{punto e virgola}, e 
		l'ultimo segna la fine della lista con un \textit{punto}. La descrizione va anticipata dai \textit{due punti}.\\
		Per quanto riguarda le liste innestate le differenze risiedono in due fattori: la lista
		a maggiore profondità ha gli elementi con la prima lettera in minuscolo, e gli elementi padre
		di una lista innestata trattano la suddetta lista come una descrizione, quindi introducendola
		con i \textit{due punti}.\\
		Con "maggiore profondità" si intende localmente, cioè si intende il blocco di elementi di una
		lista che non ha, per nessuno dei suoi elementi, una lista figlia. Nell'esempio sottostante ciò
		è reso in maniera più chiara, per capirne il funzionamento si può immaginare che la divisione 
		dei blocchi della lista avvenga similmente all'indentazione del codice sorgente.\\
		Possono esistere dei casi nei quali è accettato deviare da queste convenzioni per questioni di 
		leggibilità o comprensione (ne sono un esempio le liste dei riferimenti normativi e informativi),
		si lascia quindi a discrezione di relatori e verificatori un certo grado di libertà.\\
		Di seguito viene mostrato un esempio per chiarezza. 
		\paragraph*{Lista senza nested lists, diverse formattazioni}
		\begin{itemize}
			\item \textbf{Elemento Livello 1 Grassetto}: descrizione;
			\item \textit{Elemento Livello 1 Corsivo}: descrizione;
			\item Elemento Livello 1: descrizione.
		\end{itemize}
		\paragraph*{Lista con nested lists}
		\begin{itemize}
			\item Primo elemento livello 1:
			\begin{itemize}
				\item Primo Elemento livello 2:
				\begin{itemize}
					\item primo elemento  livello 3;
					\item secondo elemento livello 3;
					\item terzo elemento livello 3.
				\end{itemize}
				\item Secondo elemento livello 2;
				\item Terzo elemento livello 2.
			\end{itemize}
			\item Secondo elemento livello 1;
			\item Terzo elemento livello 1:
			\begin{itemize}
				\item quarto elemento livello 2;
				\item quinto elemento livello 2.
			\end{itemize}
			\item Quarto elemento livello 1.
		\end{itemize}
		
		\subsubsection{Strumenti}
			\begin{itemize}
				\item \textbf{\LaTeX\G:} stesura in bella copia dei documenti caricati poi sul repository\G{} del gruppo;
				\item \textbf{Visual Studio Code\G:} usato per la scrittura di codice e documenti, e visione del versioning tramite collegamento a GitHub\G;
				\item \textbf{Google Docs\G:} cartella condivisa contenente documenti in brutta copia, permette scrittura collaborativa contemporanea da parte di tutti i membri con visualizzazione delle modifiche in live;
			\end{itemize}

			\subsubsection{Metriche}
			\setlength{\freewidth}{\dimexpr\textwidth-0\tabcolsep}
			\renewcommand{\arraystretch}{1.5}
			\setlength{\aboverulesep}{0pt}
			\setlength{\belowrulesep}{0pt}
			\rowcolors{2}{Arancione!10}{white}
			\begin{longtable}{L{.22\freewidth} L{.4\freewidth} L{.2\freewidth}}
				\toprule
				\rowcolor{Arancione}
				\textcolor{white}{\textbf{Metrica}}&
				\textcolor{white}{\textbf{Descrizione}}&	
				\textcolor{white}{\textbf{Riferimento}}\\
				\toprule
				\endhead
				
				\textbf{MPR1} & Indice di Gulpease & \S 6.2 MPR1 \\
				\textbf{MPR13} & Errori Ortografici & \S 6.2 MPR13 \\

				\bottomrule
				\rowcolor{white}
				\caption{\centering{Metriche utilizzate per la valutazione della documentazione}}
			\end{longtable}
			
	\subsection{Processo di Configurazione}
	\subsubsection{Scopo}
		Il processo di gestione della configurazione ha lo scopo di gestire in maniera ordinata e sistematica la produzione di documenti. %codice in futuro
		 In particolare, un elemento sottoposto a configurazione ha una collocazione, una denominazione e uno stato definiti, oltre a norme e versionamento. 
 	
	\subsubsection{Versionamento}
		\paragraph*{Tecnologie adottate}
		\aCapo{}  
			Per gestire il versionamento del codice sorgente, viene utilizzato il sistema di versionamento distribuito \emph{Git\G}, con un \emph{repository\G{}} remoto presente su \emph{GitHub\G}.
			
		\paragraph*{Repository}
		\aCapo{}
			Il gruppo \teamname{} ha scelto di creare i seguenti \emph{repository\G{}} per il progetto:
			\begin{itemize}
				\item \href{https://github.com/Bug-s-Bunny-Team/docs}{\textbf{docs}}: per il versionamento dei documenti.
				\item \href{https://github.com/Bug-s-Bunny-Team/poc}{\textbf{poc}}: per la gestione e il versionamento del \textit{POC\G}.
			\end{itemize}

		\paragraph*{Struttura dei repository} %inserire struttura cartelle e disposizione file
		\aCapo{}  
		Il gruppo mantiene diversi repository\G{} separati, ognuno con struttura e fini propri.
		Di base, ogni repository\G{} conterrà i seguenti file:
		\begin{itemize}
			\item \verb#README.md#: contiene istruzioni d'uso, installazione, e sviluppo proprie del repository\G. Ne viene inoltre descritta la struttura nel dettaglio. Redatto in formato \textit{Markdown}.
			\item \verb#.gitignore#: descrive quali file ignorare dal sistema di versionamento, deve essere utilizzato per evitare la proliferazione di file inutili e che non necessitano di versionamento. È consigliato utilizzare \href{https://www.toptal.com/developers/gitignore}{servizi di generazione} per generare le regole.
			\item \verb#.editorconfig#: descrive regole di formattazione, quali indentazione e stile dei fine riga. Viene utilizzato dai formattatori automatici degli \textit{IDE} ed editor di testo. Segue il formato \href{https://editorconfig.org/}{\textit{EditorConfig}}.
		\end{itemize}

		\paragraph*{Norme di branching}
		\aCapo{}  
			Il \emph{repository\G{}} inerente alla documentazione sarà composto da diversi \textit{branch\G}:
			\begin{itemize}
				\item \verb#master#: \emph{branch}\G{} principale che viene aggiornato quando un documento è approvato o per introdurre nuove funzionalità; 
				\item un \textit{branch}\G{} per ogni documento o tipologia di documento.
			\end{itemize}
		
		\paragraph*{Modifiche ai repository}
		\aCapo{}  
		Tutti i membri del gruppo possono apportare modifiche ai file elaborati, salvo quelli presenti nel ramo \textit{master} o nel ramo \textit{dev}. \\
		 Il ramo \textit{master} verrà aggiornato alle ultime modifiche prima di una delle revisioni tramite merge dal ramo \textit{dev}. \\
		Mentre il ramo \textit{dev} verrà aggiornato con le ultime modifiche tramite merge dagli altri rami di lavoro, dopo la chiusura di una \textit{Issue\G{}} o al raggiungimento di una versione avanzata di un documento. \\
		Per entrambi i rami i merge vanno effettuati tramite \textit{Pull Request}, aperta dall'autore delle modifiche, con conseguente verifica di almeno un altro elemento del gruppo e approvazione da parte di un ulteriore elemento del gruppo che chiuderà la \textit{Pull Request} e farà il merge, preferibilmente tramite sistema \textit{"Squash and Merge"}.
		
		\subsubsection{Strumenti}
			Per la gestione della configurazione o versionamento il gruppo utilizza \emph{Visual Studio Code\G{}} o i seguenti client \emph{Git\G}:
				
				\begin{itemize}
					\item \textbf{GitHub Desktop}: Client ufficiale di \emph{GitHub\G{}} utilizzato per la gestione delle \emph{repository} \emph{Git\G};
					\item \textbf{GitKraken}.
				\end{itemize}\
		
		\subsection{Gestione della qualità}
			\subsubsection{Scopo}
				Per la gestione della qualità è dedicato il documento \PdQ{}\G: il documento fissa i \emph{requisiti} qualitativi individuati dagli \emph{stakeholder\G{}} e le metriche\G{} per la \emph{verifica} e \emph{validazione} per garantire la qualità del prodotto finale.
		
			\subsubsection{Attività di processo}
				Si possono individuare delle attività principali nel processo di gestione della qualità:
				\begin{itemize}
					\item \textbf{Pianificazione}: attività volta a definire gli obiettivi e i metodi con i quali raggiungerli;
					\item \textbf{Esecuzione}: attività che si occupa dell'esecuzione di ciò che è stato pianificato nell'attività precedente;
					\item \textbf{Controllo}: attività per analizzare i risultati ottenuti con lo scopo di misurarli e capire se l'obiettivo è stato raggiunto.
				\end{itemize}
				Queste attività mirano soprattutto all'\emph{Auto-Miglioramento} del processo, facendo un'analisi dei risultati ottenuti con lo scopo di migliorare e/o correggere eventuali attività
				che non raggiungono obiettivi prefissati.
			
			\subsubsection{Aspettative}
				 Le aspettative di questo processo sono:
					\begin{itemize}
						\item Conseguimento della qualità nel prodotto, secondo le richieste del \emph{proponente};
						\item Prova oggettiva della qualità del prodotto;						
						\item Conseguimento della qualità nell'organizzazione delle attività del gruppo e dei processi;
						\item Raggiungimento della piena soddisfazione del \emph{proponente}.
					\end{itemize}

			\subsubsection{Strumenti}
			\begin{itemize}
				\item \textbf{LaTeX\G:} stesura in bella copia dei documenti caricati poi sul repository\G{} del gruppo;
				\item \textbf{Visual Studio Code\G:} usato per la scrittura di codice e documenti, e visione del versioning tramite collegamento a GitHub\G;
				\item \textbf{Google Docs:} cartella condivisa contenente documenti in brutta copia, permette scrittura collaborativa contemporanea da parte di tutti i membri con visualizzazione delle modifiche in live.
			\end{itemize}

			\subsubsection{Metriche}
			\setlength{\freewidth}{\dimexpr\textwidth-0\tabcolsep}
			\renewcommand{\arraystretch}{1.5}
			\setlength{\aboverulesep}{0pt}
			\setlength{\belowrulesep}{0pt}
			\rowcolors{2}{Arancione!10}{white}
			\begin{longtable}{L{.22\freewidth} L{.4\freewidth} L{.2\freewidth}}
				\toprule
				\rowcolor{Arancione}
				\textcolor{white}{\textbf{Metrica}}&
				\textcolor{white}{\textbf{Descrizione}}&	
				\textcolor{white}{\textbf{Riferimento}}\\
				\toprule
				\endhead
				
				\textbf{MPR18} & Metriche di qualità soddisfatte & \S 6.2 MPR18 \\
				\textbf{MPR11} & Densità Errori & \S 6.2 MPR11 \\
				\textbf{MPP7} & Non-calculetd Risk & \S 6.1 MPP7 \\

				\bottomrule
				\caption{Metriche utilizzate per la valutazione della qualità}
			\end{longtable}
			
		
			%Magari subsubsection per norme e regole per facilitare gestione della qualità da parte del gruppo: come continua comunicazione, lavoro costante, rispetto delle norme e cosi via
	
		\subsection{Verifica}
			\subsubsection{Scopo}
				Il processo di \emph{verifica}\G{} viene applicato per individuare eventuali errori introdotti nel prodotto durante la fase di sviluppo di un processo. La \emph{verifica}\G{} viene applicata sia alla documentazione che al codice. %in futuro
			\subsubsection{Aspettative}
				Le aspettative di questo processo sono:
					\begin{itemize}
						\item Seguire procedure definite con criteri chiari ed affidabili;
						\item Verificare ad ogni fase;
						\item La verifica\G{} deve garantire che il prodotto si trovi in uno stato stabile;
						\item La verifica\G{} deve risultare più automatica possibile.
					\end{itemize}
				
				\subsubsection{Analisi}
					Il processo di analisi si suddivide in Analisi statica e Analisi dinamica:
						\paragraph*{Analisi statica}
						\aCapo{}  
							L'analisi statica permette di effettuare controlli su documenti e codice, verificando così l'assenza di errori e difetti. Esistono due metodologie per applicarla:
								\begin{itemize}
									\item \textbf{Walkthrough:} consiste nella lettura da parte del \emph{team} dell'intero documento o codice in cerca di anomalie. Viene applicata quando non si conosce in modo chiaro la sorgente dei difetti. Questa tecnica risulta molto onerosa in termini di efficienza ed efficacia.
									\item \textbf{Inspection:} consiste in una lettura mirata del documento o del codice nei punti in cui si sa già che possano essere presenti degli errori. Risulta meno dispendiosa in termini di tempo ma richiede una buona conoscenza della situazione.
								\end{itemize}
						\paragraph*{Analisi dinamica}
						\aCapo{}  
							L'analisi dinamica prevede l'esecuzione del prodotto \emph{software} e la sua analisi tramite l'utilizzo di \emph{test} che verificano se il prodotto funziona o se vi sono presenti anomalie
						\paragraph*{Test}
						\aCapo{}  
							L'attività di testing è la base dell'analisi dinamica. I \emph{test} permettono di individuare tutti i possibili errori che possono essere stati commessi e tutti i casi limite che possono risultare problematici. \newline
							I test sono ben progettati e scritti se e solo se:
								\begin{itemize}
									\item Sono ripetibili;
									\item Sono automatici;
									\item Forniscono informazioni tramite artefatti di vario genere, quali file di log.
								\end{itemize}

								
				\subsubsection{Metriche}
					\setlength{\freewidth}{\dimexpr\textwidth-0\tabcolsep}
					\renewcommand{\arraystretch}{1.5}
					\setlength{\aboverulesep}{0pt}
					\setlength{\belowrulesep}{0pt}
					\rowcolors{2}{Arancione!10}{white}
					\begin{longtable}{L{.22\freewidth} L{.4\freewidth} L{.2\freewidth}}
						\toprule
						\rowcolor{Arancione}
						\textcolor{white}{\textbf{Metrica}}&
						\textcolor{white}{\textbf{Descrizione}}&	
						\textcolor{white}{\textbf{Riferimento}}\\
						\toprule
						\endhead
						
						\textbf{MPR3} & Code Coverage & \S 6.2 MPR3 \\
						\textbf{MPR4} & Branch Coverage & \S 6.2 MPR4 \\
						\textbf{MPR16} & Percentuale test passati & \S 6.2 MPR16 \\
						\textbf{MPR17} & Percentuale test falliti & \S 6.2 MPR17 \\
						\textbf{MPP1} & Schedule\G{} variance & \S 6.1 MPP1 \\
						\textbf{MPP2} & Budget variance & \S 6.1 MPP2 \\
						\textbf{MPP3} & Budgeted Cost of Work Performed & \S 6.1 MPP3 \\
						\textbf{MPP4} & Budgeted Cost of Work Scheduled & \S 6.1 MPP4 \\

						\bottomrule
						\rowcolor{white}
						\caption{Metriche utilizzate per la verifica}
					\end{longtable}
			
		\subsection{Processo di validazione}
			\subsubsection{Scopo}
				Il processo di \emph{validazione}\G{} prende in esame il prodotto ottenuto dalla fase di \emph{verifica}\G{} e stabilisce se esso rispetti i requisiti e le aspettative del \emph{committente}.
			\subsubsection{Aspettative}
				Le aspettative di questo processo sono:
					\begin{itemize}
						\item Identificazione oggetti da validare;
						\item Valutazione dei risultati rispetto alle attese;
						\item Rendere tale processo automatico e riutilizzabile.
					\end{itemize}
