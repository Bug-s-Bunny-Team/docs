\subsection{Fornitura}
\subsubsection{Scopo}
Il processo di fornitura viene svolto al fine di comprendere al meglio la richiesta di appalto del \emph{proponente}: il documento che verrà creato include le norme a cui i membri del gruppo \teamname{} devono fare riferimento al fine di rivestire in modo adeguato il ruolo di \emph{fornitore} dell'azienda \proponente{} e dei committenti \textit{Prof. Tullio Vardanega} e \textit{Prof. Riccardo Cardin}.
\subsubsection{Aspettative}
Durante lo svolgimento del progetto il gruppo intende mantenere un rapporto di costante collaborazione con il \emph{proponente} in modo da facilitare il lavoro svolto.
In particolare, si vuole:
\begin{itemize}
	\item Stimare costi e tempistiche del lavoro;
	\item Determinare aspetti chiave che il prodotto dovrà soddisfare;
	\item Determinare vincoli e requisiti sui processi;
	\item Accordarsi sulla qualifica del prodotto.
\end{itemize}

\subsubsection{Valutazione dei capitolati}
L'attività di Valutazione dei capitolati, che sfocia nell'omonimo documento, serve a spiegare le motivazioni per la scelta di un capitolato tra quelli proposti.
Il documento \VdC{} viene redatto dopo un'attenta analisi svolta dai membri.
Per ogni capitolato sono indicate:
\begin{itemize}
	\item \textbf{Informazioni generali}: nome del progetto, il \emph{proponente} e il committente;
	\item \textbf{Descrizione}: sintesi del prodotto, caratteristiche principali e obiettivi del progetto;
	\item \textbf{Tecnologie utilizzate}: elenco delle tecnologie richieste;
	\item \textbf{Vincoli Generali}: vincoli imposti dal \emph{proponente} e da rispettare durante il progetto;
	\item \textbf{Aspetti positivi, criticità e fattori di rischio}: considerazioni fatte dal gruppo riguardanti gli aspetti positivi e sulle criticità del capitolato;
	\item \textbf{Conclusioni}: ragioni per le quali il gruppo accetta o rifiuta il capitolato.
\end{itemize}

\subsubsection{Documentazione fornita}
Il gruppo fornisce all'azienda \proponente{} e ai committenti \textit{Prof. Tullio Vardanega} e \textit{Prof. Riccardo Cardin} i seguenti documenti, essenziali per tracciare le attività svolte e per iniziare la fase di implementazione:
\begin{itemize}
	\item \AdR{}: Documento che contiene l'analisi approfondita del capitolato scelto, comprendente tutti i \emph{requisiti} e i \emph{casi d'uso} individuati;
	\item \PdP{}: Documento nel quale viene pianificato nel dettaglio il modo di lavorare del gruppo, contenente un preventivo riguardante tempistiche, l'analisi dei rischi, la pianificazione delle attività e il consuntivo;
	\item \PdQ{}: Documento dove vengono stabilite e descritte le modalità di \emph{verifica} e \emph{validazione}, in modo da garantire la qualità del prodotto.  \\
\end{itemize}
I seguenti documenti, invece, sono forniti solo all'azienda \proponente{}:
\begin{itemize}
	\item Piano di test di unità;
	\item Documentazione dettagliata di tutte le API usate;
	\item Schema Design relativo alla base dati;
	\item Diagrammi UML relativi agli Use Cases di progetto.
\end{itemize}

\subsection{Sviluppo}
\subsubsection{Descrizione e aspettative}
Lo scopo del processo di sviluppo è descrivere i compiti e le attività da svolgere relative al prodotto \emph{software} richiesto dal \emph{proponente}.
Le attività coinvolte riguardano l'\AdR{}, la progettazione e la codifica. \\
Le aspettative sono le seguenti:
\begin{itemize}
	\item stabilire gli obiettivi di sviluppo;
	\item stabilire i vincoli tecnologici e di design;
	\item realizzare un prodotto finale che superi i \emph{test}, che soddisfi i \emph{requisiti} e le richieste del \emph{proponente}. 
\end{itemize}

\subsubsection{Analisi dei requisiti}
L'analisi dei requisiti è un'attività, che sfocia nell'omonimo documento, dove vengono individuati tutti i \emph{requisiti} che il \emph{proponente} richiede per la realizzazione del prodotto.\newline{}
Il documento \AdR{} va steso in maniera efficace ed è molto importante in quanto coinvolto in diverse fasi della realizzazione del prodotto: oltre che definire funzionalità e \emph{requisiti} individuati e concordati col cliente, fornisce ai \progrs{} riferimenti precisi e affidabili, ai \verf{} riferimenti per il processo di \emph{verifica} ed è la base dalla quale partire per eventuali raffinamenti successivi, garantendo un continuo miglioramento del prodotto. 
Ogni requisito può essere ricavato da diverse fonti:
\begin{itemize}
	\item \textbf{Capitolato d'Appalto}: attraverso la lettura del capitolato;
	\item \textbf{Casi d'uso}: estrapolati da uno o più casi d'uso; 
	\item \textbf{Verbali}: attraverso riunioni interne o incontri con l'azienda \emph{proponente}.
\end{itemize}

\subsection{Struttura}
La struttura del documento \AdR{} è la seguente:
\begin{itemize}
	\item \textbf{Introduzione}: contiene scopo del documento, introduzione al progetto e riferimenti;
	\item \textbf{Descrizione}: contiene informazioni riguardanti il prodotto, la piattaforma d'esecuzione e la descrizione degli utenti interessati;
	\item \textbf{Casi d'uso}: vengono identificati gli attori che interagiscono con le componenti del sistema e le interazioni tra sistema, attori ed elementi esterni;
	\item \textbf{Requisiti}: rappresentati in tabelle che riportano le seguenti informazioni:
		\begin{itemize}
			\item \textbf{Codifica}: codice di riferimento del requisito;
			\item \textbf{Classificazione}: obbligatorio o desiderabile;
			\item \textbf{Descrizione}: breve descrizione del requisito;
			\item \textbf{Fonti}: da dove è stato ricavato il requisito.
		\end{itemize}
\end{itemize}

\subsubsection{Classificazione dei requisiti}
La classificazione requisiti verrà effettuata mediante la seguente codifica:\newline \newline
\centerline{\textbf{R[Tipo][Numero]}}

{
	
	\setlength{\freewidth}{\dimexpr\textwidth-0\tabcolsep}
	\renewcommand{\arraystretch}{1.5}
	\setlength{\aboverulesep}{0pt}
	\setlength{\belowrulesep}{0pt}
	\rowcolors{2}{Arancione!10}{white}
	\begin{longtable}{L{.15\freewidth} L{.6\freewidth} L{.080\freewidth}}
		\toprule
		\rowcolor{Arancione}
		\textcolor{white}{\textbf{Nome}}&
		\textcolor{white}{\textbf{Descrizione}}\\	
		\toprule
		\endhead
		
		\textbf{R} & Abbrevia "Requisito" \\
		\multirow{4}*\textbf{Tipo}
		&  \textbf{F}: requisito funzionale, ossia la definizione di una particolare caratteristica che deve essere inclusa nel software \\
		\cline{2-2}
		&\textbf{V}: requisito di vincolo che rappresenta dei vincoli avanzati dal \emph{proponente} \\
		\cline{2-2} 
		&\textbf{Q}: requisito di qualità, relativo alle regole di qualità del software (efficienza ed efficacia) \\
		\cline{2-2} 
		&\textbf{P}: requisito di prestazione, relativo alle prestazioni del software \\
		\textbf{Numero} & Codice identificativo \\
		\bottomrule
		\caption{Tabella degli elementi che classificano i requisiti}
	\end{longtable}
}

\subsubsection{Classificazione dei casi d'uso}
La codifica scelta per la rappresentazione dei casi d'uso è la seguente: \newline \newline
\centerline{\textbf{UC[CodiceCaso][CodiceSottoCaso]}}

{
	
	\setlength{\freewidth}{\dimexpr\textwidth-0\tabcolsep}
	\renewcommand{\arraystretch}{1.5}
	\setlength{\aboverulesep}{0pt}
	\setlength{\belowrulesep}{0pt}
	\rowcolors{2}{Arancione!10}{white}
	\begin{longtable}{L{.22\freewidth} L{.6\freewidth} L{.080\freewidth}}
		\toprule
		\rowcolor{Arancione}
		\textcolor{white}{\textbf{Nome}}&
		\textcolor{white}{\textbf{Descrizione}}\\	
		\toprule
		\endhead
		
		\textbf{UC} & Acronimo di "Use Case" \\
		\textbf{CodiceCaso} & Identificativo del caso d'uso generico \\
		\textbf{CodiceSottoCaso} & Identificativo opzionale per gli eventuali sotto casi del caso d'uso \\
		
		\bottomrule
		\caption{Tabella degli elementi che classificano i casi d'uso}
	\end{longtable}
}

\subsubsection{Struttura dei casi d'uso}
{
	\setlength{\freewidth}{\dimexpr\textwidth-0\tabcolsep}
	\renewcommand{\arraystretch}{1.5}
	\setlength{\aboverulesep}{0pt}
	\setlength{\belowrulesep}{0pt}
	\rowcolors{2}{Arancione!10}{white}
	\begin{longtable}{L{.22\freewidth} L{.6\freewidth} L{.080\freewidth}}
		\toprule
		\rowcolor{Arancione}
		\textcolor{white}{\textbf{Nome}}&
		\textcolor{white}{\textbf{Descrizione}}\\	
		\toprule
		\endhead
		
		\textbf{Codifica} & Identificativo del caso d'uso generico \\
		\textbf{Nome} & Nome descrittivo del caso d'uso \\
		\textbf{Descrizione Grafica} & Realizzata con UML \\
		\textbf{Attore} & Interagisce col sistema per il raggiungimento di un obiettivo \\
		\textbf{Descrizione} & Breve descrizione dell'obiettivo \\
		\textbf{Scenario} & Rappresentato attraverso elenco numerato degli eventi \\
		\textbf{Estensioni} & Opzionali, per modellare scenari alternativi. Se si verifica, il caso d'uso ad essa collegata si interrompe \\
		\textbf{Precondizioni} & Condizioni del sistema prima del verificarsi del caso d'uso \\
		\textbf{Postcondizioni} & Condizioni del sistema dopo il verificarsi del caso d'uso \\
		
		\bottomrule
		\caption{Tabella della struttura dei casi d'uso}
	\end{longtable}
}

\paragraph*{Metriche}
\setlength{\freewidth}{\dimexpr\textwidth-0\tabcolsep}
\renewcommand{\arraystretch}{1.5}
\setlength{\aboverulesep}{0pt}
\setlength{\belowrulesep}{0pt}
\rowcolors{2}{Arancione!10}{white}
\begin{longtable}{L{.22\freewidth} L{.6\freewidth} L{.080\freewidth}}
	\toprule
	\rowcolor{Arancione}
	\textcolor{white}{\textbf{Metrica}}&
	\textcolor{white}{\textbf{Descrizione}}\\	
	\toprule
	\endhead
	
	\textbf{MPR2} & Percentuale Requisiti Obbligatori Soddisfatti \\
	
	\bottomrule
	\caption*{Metriche utilizzate per la valutazione di analisi dei requisiti.}
\end{longtable}

\subsubsection{Progettazione} 
L'attività di progettazione definisce le caratteristiche che il prodotto richiesto deve avere in modo da fornire una soluzione che soddisfa i requisiti specificati nell'\AdR{}.
Il procedimento è infatti l'opposto rispetto a quello utilizzato nell'Analisi dei Requisiti, in quest'ultimo avviene una suddivisione del problema in parti
per poter comprendere al meglio il dominio applicativo, mentre nella progettazione si ricostruisce il problema specificando ogni funzionalità di ogni parte.

\paragraph*{Technology Baseline}
Misura le specifiche della progettazione nelle tecnologie indivduate per realizzare l'architettura del prodotto. Vengono rese note:
\begin{itemize}
	\item Tecnologie adottate;
	\item Relazioni e interazioni tra i vari componenti;
	\item {\textit{Proof of Concept}}, un prototipo per dimostrare in modo pratico la corretta compatibilità  dellle tecnologie utilizzate.
\end{itemize}

\paragraph*{Product Baseline}
Rappresenta la base architetturale definita coerentemente nella Technology Baseline. Deve comprendere:
\begin{itemize}
	\item {\textit{Design Pattern}} che sono stati utilizzati e una loro descrizione;
	\item Diagrammi UML, in particolare diagrammi delle classi, sequenza, package;
	\item Tracciamento dei requisiti che devono essere soddisfatti da una classe.
\end{itemize}

\paragraph*{Metriche}
\setlength{\freewidth}{\dimexpr\textwidth-0\tabcolsep}
\renewcommand{\arraystretch}{1.5}
\setlength{\aboverulesep}{0pt}
\setlength{\belowrulesep}{0pt}
\rowcolors{2}{Arancione!10}{white}
\begin{longtable}{L{.22\freewidth} L{.6\freewidth} L{.080\freewidth}}
	\toprule
	\rowcolor{Arancione}
	\textcolor{white}{\textbf{Metrica}}&
	\textcolor{white}{\textbf{Descrizione}}\\	
	\toprule
	\endhead
	
	\textbf{MPR5} & Accoppiamento tra classi \\
	\textbf{MPR6} & Profondità delle gerarchie \\
	\textbf{MPR12} & Facilità di utilizzo \\
	
	\bottomrule
	\caption*{Metriche utilizzate per la valutazione della progettazione.}
\end{longtable}
%Diagrammi e test

\subsection{Codifica} %Da implementare convenzioni future
La codifica ha lo scopo di normare l'effettiva realizzazione del prodotto richiesto. I \progrs{} dovranno attenersi a queste norme durante la fase di programmazione e implementazione. L'uso di norme e convenzioni è fondamentale per permettere la generazione di codice leggibile e uniforme, agevolare la manutenzione e i processi di \emph{verifica} e \emph{validazione}.

\subsection{Strumenti e linguaggi di programmazione}
Di seguito vengono elencati e descritti i vari strumenti e linguaggi di programmazione utilizzati durante il progetto, siano essi richiesti dal \emph{proponente} o scelti autonomamente:
\paragraph*{AWS}
\begin{itemize}
	\item \href{https://aws.amazon.com/fargate/}{\emph{Fargate}}: servizio serverless per la gestione di container
	\item \href{https://aws.amazon.com/rds/}{\emph{RDS}}: database relazionale in modalità serverless
	\item \href{https://aws.amazon.com/lambda/}{\emph{Lambda}}: servizio di calcolo serverless, basato sugli eventi
	\item \href{https://aws.amazon.com/fargate/}{\emph{API Gateway}}: servizio gestito che semplifica creazione, pubblicazione, manutenzione, monitoraggio e protezione delle API dei vari microservizi sviluppati
	\item \href{https://aws.amazon.com/rekognition/}{\emph{Rekognition}}: servizio per l'estrazione di informazioni da immagini e video tramite tecniche di Machine Learning
	\item \href{https://aws.amazon.com/comprehend/}{\emph{Comprehend}}: servizio per l'estrazione di informazione da testi, tramite tecniche di Natural Language Processing
	\item \href{https://aws.amazon.com/cognito/}{\emph{Cognito}}: servizio che permette di aggiungere strumenti di registrazione degli utenti, accesso e controllo degli accessi 
\end{itemize}
\paragraph*{Linguaggi di programmazione}
% al momento messi come placeholder, ma probabilmente useremo soltanto questi
\begin{itemize}
	\item \href{https://www.python.org/}{\emph{Python}}
	\item \href{https://www.typescriptlang.org/}{\emph{Typescript}}
\end{itemize}
\paragraph*{Librerie e framework}
% al momento non si sa cosa andremo ad usare effettivamente
\begin{itemize}
	\item \href{https://reactjs.org/}{\emph{React}}: libreria per lo sviluppo web frontend
\end{itemize}
