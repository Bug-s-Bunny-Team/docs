\subsection{Metriche per qualità di processo}
Alcuni parametri per comprendere le tabelle:
\begin{itemize}
    \item ACWP : Actual Cost of Work Performed;
    \item NoC / NoD / NoA : Number of Changed / Added / Deleted \textit{: Numero di requisiti cambiati / aggiunti / eliminati};
    \item TNIR : Total Number of Initial Requirements.
\end{itemize}
{
    
    %\newlength{\freewidth}
    \setlength{\freewidth}{\dimexpr\textwidth-10\tabcolsep}
    \renewcommand{\arraystretch}{1.5}
    \centering
    \setlength{\aboverulesep}{0pt}
    \setlength{\belowrulesep}{0pt}
    \rowcolors{2}{Arancione!10}{white}
    \begin{longtable}{C{.115\freewidth} C{.257\freewidth} C{.26\freewidth} C{.4\freewidth}}
       \toprule
    \rowcolor{Arancione}
    \textcolor{white}{\textbf{Codice}}&
    \textcolor{white}{\textbf{Nome}}&
    \textcolor{white}{\textbf{Descrizione}}&
    \textcolor{white}{\textbf{Ottenimento}}\\	
    \toprule
    \endhead
    
    
    MPP1 & Schedule\G{} variance & Variazione rispetto ai tempi pianificati & $\frac{100\cdot(BCWP-BCWS)}{BCWS}$ \\
    MPP2 & Budget variance & Variazione rispetto ai costi preventivati & $\frac{100\cdot(BCWS-ACWP)}{BCWS}$ \\
    MPP3 & Budgeted Cost of Work Performed & Valore effettivo del prodotto al calcolo dell'indice & $sum(\forall ruolo (oreRuolo \cdot costoOrario))$ \\
    MPP4 & Budgeted Cost of Work Scheduled & Previsione costi  & \PdP \\
    MPP5 & SPICE capability & Misura qualità processi & \S 5 di questo documento \\
    MPP6 & Requirements Stabilty Index & Indica variabilità dei requisiti nel tempo & $(1-\frac{Noc+NoD+NoA}{TNIR})\cdot100$ \\
    MPP7 & Non-calculated Risk & Numero di rischi non preventivati & Numero intero \\	   
    \rowcolor{white}
    \bottomrule
    \rowcolor{white}
    \caption{Tabella metriche per qualità di processo}
    \end{longtable}
}

\subsection{Metriche per qualità di prodotto}
Alcuni parametri per comprendere le tabelle:
\begin{itemize}
    \item NdF / NdL / NdP : Numero di Frasi / Lettere / Parole;
    \item \# : numero, inteso come "quantità", di una certa collezione di elementi; 
    \item T\textsubscript{pos} : numero tests eseguiti sul programma che rilevano errori;
    \item  T\textsubscript{neg} : numero tests eseguiti sul programma che non rilevano errori;
    \item T\textsubscript{tot} : numero di tests eseguiti sul programma.
\end{itemize}
{
    
    %\newlength{\freewidth}
    \setlength{\freewidth}{\dimexpr\textwidth-10\tabcolsep}
    \renewcommand{\arraystretch}{1.5}
    \centering
    \setlength{\aboverulesep}{0pt}
    \setlength{\belowrulesep}{0pt}
    \rowcolors{2}{Arancione!10}{white}
    \begin{longtable}{C{.115\freewidth} C{.257\freewidth} C{.26\freewidth} C{.4\freewidth}}
       \toprule
    \rowcolor{Arancione}
    \textcolor{white}{\textbf{Codice}}&
    \textcolor{white}{\textbf{Nome}}&
    \textcolor{white}{\textbf{Descrizione}}&
    \textcolor{white}{\textbf{Ottenimento}}\\	
    \toprule
    \endhead

    
    MPR1 & Indice di Gulpease & Indice di leggibilità del testo & $89+\frac{300\cdot NdF-10\cdot NdL}{NdP}$  \\
    MPR2 & Percentuale requisiti obbligatori soddisfatti & Autoesplicativo & $100\cdot \frac{\# \: requisiti \: soddisfatti}{\# \: requisiti \: totali}$ \\
    MPR3 & Code coverage & \% linee codice percorse dai tests & $100\cdot \frac{linee \: codice \: percorse}{linee \: codice \: totali}$ \\
    MPR4 & Branch coverage & \% rami condizionali coperti dai tests  & $100\cdot \frac{rami \: condizionali \: percorsi}{rami \: condizionali \: totali}$ \\
    MPR5 & Accoppiamento tra classi & Numero di dipendenze per classe & Numero Intero \\
    MPR6 & Profondità gerarchie & Rappresenta la quantità di super classi & Numero Intero \\
    MPR7 & Numero attributi per classe & Autoesplicativo & Numero Intero \\
    MPR8 & Numero parametri per metodo & Autoesplicativo & Numero Intero \\
    MPR9 & Linee codice per metodo & Autoesplicativo & Numero Intero \\
    MPR10 & Linee commento per codice & Autoesplicativo & $\frac{\# \: linee \: codice}{\# \: linee \: commento}$ \\
    MPR11 & Densità errori & Percentuale che rappresenta la solidità del prodotto & $100\cdot \frac{T\textsubscript{pos}}{T\textsubscript{tot}}$ \\
    MPR12 & Facilità utilizzo  & Numero di input necessari all'utente per ottenere il risultato voluto & Numero Intero \\
    MPR13 & Errori ortografici & Autoesplicativo & PdF checker e simili \\
    MPR14 & Complessità ciclomatica media & Numero cammini linearmente indipendenti nel programma & Grafo controllo di flusso \\
    MPR15 & Tempo medio risposta WebApp & Autoesplicativo & Misurato in secondi \\
    MPR16 & Percentuale test passati & Autoesplicativo & $100\cdot \frac{T\textsubscript{neg}}{ T\textsubscript{tot}}$ \\
    MPR17 & Percentuale test falliti & Autoesplicativo & MPR11 \\
    MPR18 & Metriche di qualità soddisfatte & Percentuale di metriche che rientrano nei valori accettabili & $100\cdot \frac{\# \: metriche \: soddisfatte}{\# \: metriche \: totali}$ \\	   
    \rowcolor{white}
    \bottomrule
    \rowcolor{white}
    \caption{Tabella metriche per qualità di processo}
    \end{longtable}
}
