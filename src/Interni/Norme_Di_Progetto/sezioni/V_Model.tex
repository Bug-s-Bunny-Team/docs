\subsection{Sintesi}
Il modello a V\G{} è un sistema astratto atto a favorire lo sviluppo di software qualitativamente valido.\\
Struttura i vari step necessari secondo due rami, uno decrescente e uno crescente:\\
Nel primo si "scende" verso la fase di codifica decomponendo le necessità del cliente e degli stakeholders\G{}
in requisiti via via più specifici, per poi andare a progettarli e realizzarli.\\
Nel secondo ramo si "risale" dalla fase di codifica e si va a verificare e validare il lavoro fatto,
concettualmente andando a ricomporre le parti precedentemente divise per ottenere il prodotto finale.\\
Il tutto si conclude con il collaudo, dove il cliente e gli stakeholders\G{} controllano il software completo
e si accertano che esso svolga i compiti previsti e voluti. 

\subsection{V\&V}
Verifica\G{} e validazione\G{} possono essere espressi con le seguenti due frasi: "did I build the system right?"
e "did I build the right system?", ovvero la verifica\G{} si occupa di controllare che ciò che è stato
fatto segua le metriche\G{} di qualità e rispetti le best practices\G{} (di codifica, di documentazione, etc.), 
mentre la validazione\G{} accerta che ciò che è stato prodotto sia conforme alle
richieste e alle necessità degli stakeholders\G.

\subsection{Due rami}
Come precedentemente spiegato nella sintesi, il modello a V\G{} comprende due rami (in inlese chiamati anche
"streams") che possiamo schematizzare come: 
\subsubsection*{Specification Stream:}
\begin{itemize}
    \item Richieste utente;
    \item Analisi dei requisiti;
    \item Progettazione logica;
    \item Progettazione di dettaglio.
\end{itemize}

\subsubsection*{Codifica;}

\subsubsection*{Testing Stream:}
\begin{itemize}
    \item Verifica progettazione di dettaglio;
    \item Verifica progettazione logica;
    \item Validazione analisi dei requisiti;
    \item Collaudo (validazione richieste utente).
\end{itemize}

\subsection{Obiettivi}
Il modello a V\G{} si propone come linea guida per le fasi di pianificazione e realizzazione di un progetto,
nello specifico tenta di:
\begin{itemize}
    \item \textbf{Minimizzare il rischio:} specificando un approccio standardizzato al lavoro migliora 
    la trasparenza e rende più facile il controllo delle fasi di un progetto. Permette inoltre di 
    riconoscere con anticipo le deviazioni dal piano e l'insorgere di rischi;
    \item \textbf{Miglioramento della qualità:} seguire con cura il modello a V\G{} permette di mantenere una
    qualità generale del prodotto più alta;
    \item \textbf{Riduzione dei costi:} l'impegno necessario nelle fasi di sviluppo, produzione, rilascio
    e manutenzione di un sistema può essere stimato con più precisione applicando un modello standardizzato;
    \item \textbf{Miglioramento della comunicazione:} una descrizione uniforme degli elementi rilevanti 
    e dei termini contrattuali del progetto favoriscono una comunicazione migliore tra gli stakeholders\G.
\end{itemize}
