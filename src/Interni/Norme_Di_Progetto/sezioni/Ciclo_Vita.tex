\subsection{Introduzione}
È uno standard\G{} per la gestione del ciclo di vita del software. Stabilisce un processo\G{} di ciclo di vita del software, compreso processi ed attività relative alle specifiche ed alla configurazione di un sistema e ad ogni processo corrisponde un insieme di risultati.
La struttura dello standard\G{} è stata concepita per essere flessibile e modulare in modo che sia adattabile alle necessità di chiunque lo utilizzi.

\subsubsection{Principi fondamentali}
Lo standard\G{} è basato su due principi fondamentali:
\begin{itemize}
	\item \textbf{Modularità}: definire processi\G{} con il minimo accoppiamento e la massima coesione;
	\item \textbf{Responsabilità}: stabilire un responsabile per ogni processo\G.
\end{itemize}

\subsubsection{Tipi di processi}
Esistono tre tipi di processi:
\begin{itemize}
	\item Processi\G{} primari;
	\item Processi\G{} di supporto;
	\item Processi\G{} di organizzazione.
\end{itemize}
