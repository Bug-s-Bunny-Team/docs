\subsection{Introduzione}
È uno standard per la gestione del ciclo di vita del software. Stabilisce un processo di ciclo di vita del software, compreso processi ed attività relative alle specifiche ed alla configurazione di un sistema e ad ogni processo corrisponde un insieme di risultati.
La struttura dello standard è stata concepita per essere flessibile e modulare in modo che sia adattabile alle necessità di chiunque lo utilizzi.

\subsubsection{Principi Fondamentali}
Lo standard è basato su due principi fondamentali:
\begin{itemize}
	\item \textbf{Modularità}: definire processi con il minimo accoppiamento e la massima coesione;
	\item \textbf{Responsabilità}: stabilire un responsabile per ogni processo.
\end{itemize}

\subsubsection{Tipi di Processi}
Esistono tre tipi di processi:
\begin{itemize}
	\item Processi primari;
	\item Processi di supporto;
	\item Processi di organizzazione.
\end{itemize}