\subsection{Introduzione}
Lo standard \textit{SPICE} (Software Process Improvement and Capability Determination) è un insieme di
standard tecnici per i processi di sviluppo software.

\subsubsection{Classificazione processi}
Nello standard \textit{SPICE} i processi sono suddivisi in 5 categorie:
\begin{itemize}
	\item Cliente-fornitore;
	\item Ingegneristico;
	\item Supporto;
	\item Gestionale;
	\item Organizzativo.
\end{itemize}

\subsection{Livelli di capability e attributi di processo}
La capability viene definita come la capacità di un processo di raggiungere il suo scopo. Per ogni processo, \textit{SPICE} definisce un livello di capability, che sono i seguenti:
\begin{itemize}
	\item \textbf{Livello 0 - Incomplete process}: processo\G{} non implementato, incapace di raggiungere i suoi
	obiettivi;
	\item \textbf{Livello 1 - Performed process}: processo\G{} implementato e in grado di raggiungere i suoi obiettivi,
	ma non è sottoposto a nessun tipo di controllo;
	\item \textbf{Livello 2 - Managed process}: processo\G{} pianificato e sottoposto a controllo e correzione, gli
	obiettivi vengono raggiunti e sono tracciabili e verificati;
	\item \textbf{Livello 3 - Established process}: processo\G{} definito da standard e quindi regolamentato;
	\item \textbf{Livello 4 - Predictable process}: processo\G{} istanziato\G{} entro limiti ben definiti, viene monitorato
	in modo dettagliato con lo scopo di renderlo prevedibile e ripetibile;
	\item \textbf{Livello 5 - Optimizing process}: processo\G{} completamente definito e tracciato, soggetto ad analisi
	e miglioramento continui.
\end{itemize}
 
La capability di un processo è misurata tramite gli attributi di processo; lo standard\G{} definisce i seguenti 9 attributi
(riportati con la seguente codifica: [Livello di capability][Numero attributo]-[Nome attributo]):
\begin{itemize}
	\item \textbf{1.1 - Process performance}: numero di obiettivi raggiunti;
	\item \textbf{2.1 - Performance management}: livello di organizzazione degli obiettivi fissati;
	\item \textbf{2.2 - Work product management}: livello di organizzazione dei prodotti rilasciati;
	\item \textbf{3.1 - Process definition}: livello di adesione agli standard prefissati;
	\item \textbf{3.2 - Process deployment}: livello di ripetibilità del processo;
	\item \textbf{4.1 - Process measurement}: livello di efficacia di applicazione delle metriche\G{} al processo;
	\item \textbf{4.2 - Process control}: livello di predicibilità delle valutazioni;
	\item \textbf{5.1 - Process innovation}: misura gli aspetti positivi generati dei cambiamenti attuati dopo una
	fase di analisi;
	\item \textbf{5.2-Process optimization}: misura l'efficienza del processo, il rapporto tra i risultati ottenuti e
	le risorse impegnate.
\end{itemize}

Ogni processo è valutato tramite la seguente scala di valori che esprimono numericamente il grado di
soddisfacimento dell'attributo:
\begin{itemize}
	\item \textbf{N}: Not achieved (0 - 15\%);
	\item \textbf{P}: Partially achieved ($>$ 15 - 50\%);
	\item \textbf{L}: Largely achieved ($>$ 50 - 85\%);
	\item \textbf{F}: Fully achieved ($>$ 85 - 100\%).
\end{itemize}
