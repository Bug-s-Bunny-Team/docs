\subsection{Introduzione}
È uno standard\G{} che punta a creare un framework\G{} per l'evoluzione della qualità del prodotto software ed è nato dall'evoluzione di due precedenti standard:
\begin{itemize}
	\item \textbf{ISO/IEC 9126}: definisce un modello di qualità per la valutazione del prodotto software;
	\item \textbf{ISO/IEC 14598}: definisce il processo\G{} per la valutazione del prodotto software.
\end{itemize}

\subsection{Divisioni dello standard}
Lo standard ISO/IEC 25000 è suddiviso in cinque divisioni:
\begin{itemize}
	\item \textbf{ISO/IEC 2500n - Quality Management Division}: definisce modelli, termini e definizioni comuni per tutti gli standard\G{} delle serie SQuaRE;
	\item \textbf{ISO/IEC 2501n - Quality Model Division}: presenta modelli di qualità dettagliati per sistemi informatici e prodotti software, qualità d'uso e dati;
	\item \textbf{ISO/IEC 2502n - Quality Measurement Division}: presenta un modello di riferimento per la misurazione della qualità del prodotto software, definizioni delle misure di qualità e linee guide pratiche per la loro applicazione;
	\item \textbf{ISO/IEC 2503n - Quality Requirements Division}: aiuta a specificare i requisiti di qualità da seguire e soddisfare;
	\item \textbf{ISO/IEC 2504n - Quality Evaluation Division}: presenta requisti, raccomandazioni e linee guida per la valutazione dei prodotti software.
\end{itemize}

\subsection{Qualità perseguite dallo standard}
Di seguito vengono riportate le qualità perseguite dallo standard:
\begin{itemize}
	\item \textbf{Qualità di prodotto}:
		\begin{itemize}
			\item Adeguatezza Funzionale;
			\item Efficienza Prestazionale;
			\item Compatibilità;
			\item Usabilità;
			\item Affidabilità;
			\item Sicurezza;
			\item Manutenibilità;
			\item Portabilità.
		\end{itemize}
	\item \textbf{Qualità in uso}:
		\begin{itemize}
			\item Efficacia;
			\item Efficienza;
			\item Soddisfazione;
			\item Mitigazione dei rischi;
			\item Copertura.
		\end{itemize}
\end{itemize}

\subsection{Processo di valutazione}
Lo standard prevede un processo di valutazione da seguire per valutare la qualità del software; tale processo si compone dei seguenti cinque passi:
\begin{itemize}
	\item Stabilire i requisiti di valutazione;
	\item Specificare la valutazione;
	\item Progettare la valutazione;
	\item Eseguire la valutazione;
	\item Concludere la valutazione.
\end{itemize}
