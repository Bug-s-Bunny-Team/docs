\subsection{Gestione dei Processi} %Da lavorarci su
\subsubsection{Scopo}
La presente sezione espone gli strumenti impiegati dal gruppo \teamname{} per quanto concerne le attività di comunicazione interna ed esterna, l'organizzazione e la gestione dei ruoli di ogni componente.
A tale processo è dedicato il documento \PdP{}.

\subsubsection{Aspettative}
Le aspettative di questo processo sono:
\begin{itemize}
	\item Definizione strumenti e modalità di comunicazione
	\item Definizione strumenti, modalità e norme di organizzazione
	\item Definizione norme per la gestione dei ruoli dei membri del gruppo
\end{itemize}

\subsection{Strumenti di comunicazione}
Il gruppo \teamname{} utilizza vari strumenti per la comunicazione sia interna che esterna:
	\subsubsection{Comunicazione Interna}
		Le comunicazioni interne riguardano solamente i membri del gruppo \teamname{} attraverso l'utilizzo dei seguenti strumenti:
			\begin{itemize}
				\item \textbf{Telegram}: Usato per pianificare incontri su \emph{Discord} e per comunicazioni rapide e non troppo importanti
				\item \textbf{Discord}: Usato per incontri di discussione e pianificazione sul lungo periodo; inoltre è stato integrato con \emph{GitHub} per la ricezione di messaggi riguardanti lo stato dei repository
			\end{itemize}
	\subsubsection{Comunicazioni Esterne}
	\label{sec:Comunicazioni Esterne}
		Le comunicazioni esterne riguardano le comunicazione tra i membri del gruppo \teamname e persone esterne al \emph{gruppo}:
			\begin{itemize}
				\item \textbf{ProtonMail}: Usato per la comunicazione scritta tramite email con l'azienda \proponente e i \emph{committenti}
				\item \textbf{Slack}: Usato per la comunicazione con il rappresentante dell'azienda \proponente
			\end{itemize}
		
		\subsection{Gestione Degli Incontri}
		\subsubsection{Incontri Interni}
		Gli incontri interni si svolgono almeno una volta a settimana per gestire il lavoro settimanale del progetto e possono essere richiesti da un qualsiasi componente del gruppo. Sono un momento di confronto e risoluzione di eventuali problemi. Ad ogni riunione è auspicabile che partecipino tutti i componenti del gruppo.
		\subsubsection{Incontri Esterni}
		Per ogni incontro esterno il \RdP{} si occupa di comunicare con l'azienda, il professore o altri enti esterni al gruppo di lavoro. 
		Le vie utilizzate sono le due descritte nel \hyperref[sec:Comunicazioni Esterne]{paragrafo 4.2.2}.
		Questo tipo di incontri avvengono saltuariamente a seconda delle necessità burocratiche o operative del progetto (quindi revisioni o necessità di supporto sulle tecnologie utilizzate).
		Ad ogni riunione partecipano tutti i membri del gruppo, salvo imprevisti inderogabili.
		\subsubsection{Verbali}
		Per ogni incontro (interno o esterno) viene redatto un verbale da una persona scelta dal responsabile e successivamente verrà controllato dal verificatore e confermato dall’approvatore. Sia quest'ultimo che il verificatore, suggeriscono eventuali integrazioni o modifiche al redattore.

\subsection{Ruoli di progetto} % intabellare ?
Il \RdP{} deve assicurarsi che ogni membro del \emph{gruppo} ricopra almeno una volta ogni ruolo, che sono i seguenti:
	\begin{itemize}
		\item \textbf{\RdP}: è la figura centrale del progetto, esso ha il compito di coordinare il \emph{team} e il ruolo di rappresentante del progetto durante le comunicazioni con l'esterno;
		\item \textbf{\Amm}: ha il compito di gestire e configurare adeguatamente le piattaforme utilizzate dal gruppo per lo svolgimento del progetto. Da lui dipendono l'affidabilità e l'efficacia dei mezzi scelti per lo svolgimento del progetto;
		\item \textbf{\Ana}: segue il progetto principalmente nelle fasi iniziali ed è fortemente coinvolto nella stesura dell' \AdR{}. Il suo ruolo è quello di analizzare i problemi posti dal progetto e chiarire le dipendenze e le ramificazioni di ogni attività necessaria alla consegna del prodotto;
		\item \textbf{\Prog}: segue lo sviluppo del progetto e, a partire dai \emph{requisiti}, definisce le scelte tecniche necessarie per lo sviluppo del prodotto;
		\item \textbf{\Progm}: ha il compito di codificare i modelli realizzati dal \Prog{}. Il codice prodotto dal \Progm{} deve attenersi il più possibile alle specifiche elaborate dal \Prog{} e documentare opportunamente il codice creato per aumentarne la manutenibilità;
		\item \textbf{\Ver}: segue l'intero ciclo di vita del progetto. Egli si assicura che la qualità della documentazione prodotta aderisca alla norme stabilite.
	\end{itemize}

	\subsubsection{Assegnazione dei compiti}
	Per garantire l'efficienza e la flessibilità del processo di sviluppo, le attività necessarie al completamento di un progetto devono essere suddivise in compiti, che possono essere svolti sequenzialmente o in parallelo a seconda delle dipendenze tra di loro. Per ottimizzare tale processo il gruppo utilizza il sistema delle \emph{Issues} offerto da \emph{Github}.	
	Una volta individuate dall'\Ana{}, le attività vengono all'occorrenza suddivise in sotto attività e compiti. Il \RdP{} si occupa di assegnare ad ogni membro del team compiti diversi per ottimizzare il progresso del progetto e nel farlo deve valutarne l'idoneità, tenendo conto di disponibilità, competenze tecniche e carico di lavoro attuale del membro in esame. A questo punto, il membro assegnato all'incarico dovrà svolgere il compito nei tempi definiti dalla \emph{Milestone} collegata alla \emph{Issue} e, nel caso di contrattempi, notificarlo quanto prima possibile al \RdP{}. Il membro assegnato alla \emph{Issue} è inoltre responsabile della chiusura di quest'ultima. Inoltre ogni membro associato a una \emph{Issue} verrà notificato tramite email per ogni modifica relativa alla documentazione %o codice in futuro
	di cui si occupa la \emph{Issue}.