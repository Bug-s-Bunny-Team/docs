\subsection{Gestione dei processi}
\subsubsection{Scopo}
La presente sezione espone gli strumenti impiegati dal gruppo \teamname{} per quanto concerne le attività di comunicazione interna ed esterna, l'organizzazione e la gestione dei ruoli di ogni componente.
A tale processo è dedicato il documento \PdP{}\textit{\G}.

\subsubsection{Aspettative}
Le aspettative di questo processo sono:
\begin{itemize}
	\item Definizione strumenti e modalità di comunicazione;
	\item Definizione strumenti, modalità e norme di organizzazione;
	\item Definizione norme per la gestione dei ruoli dei membri del gruppo.
\end{itemize}


		
\subsubsection{Gestione degli incontri}
	\paragraph{Incontri interni}
	Gli incontri interni si svolgono almeno una volta a settimana per gestire il lavoro settimanale del progetto e possono essere richiesti da un qualsiasi componente del gruppo. Sono un momento di confronto e risoluzione di eventuali problemi. Ad ogni riunione è auspicabile che partecipino tutti i componenti del gruppo.
	\paragraph{Incontri esterni}
	Per ogni incontro esterno il \RdP{} si occupa di comunicare con l'azienda, i committenti o altri enti esterni al gruppo di lavoro. 
	Le vie utilizzate sono le due descritte nel \hyperref[sec:Comunicazioni esterne]{paragrafo \S4.2.2}.
	Questo tipo di incontri avvengono saltuariamente a seconda delle necessità burocratiche o operative del progetto (quindi revisioni o necessità di supporto sulle tecnologie utilizzate).
	Ad ogni riunione partecipano tutti i membri del gruppo, salvo imprevisti inderogabili.
	\paragraph{Verbali}
	Per ogni incontro (interno o esterno) viene redatto un verbale da una persona scelta dal responsabile e successivamente verrà controllato dal verificatore e confermato dall’approvatore. Sia quest'ultimo che il verificatore, suggeriscono eventuali integrazioni o modifiche al redattore.

\subsubsection{Ruoli di progetto} 
Il \RdP{} deve assicurarsi che ogni membro del \emph{gruppo} ricopra almeno una volta ogni ruolo, che sono i seguenti:
	\begin{itemize}
		\item \textbf{\RdP}: è la figura centrale del progetto, esso ha il compito di coordinare il \emph{team} e il ruolo di rappresentante del progetto durante le comunicazioni con l'esterno;
		\item \textbf{\Amm}: ha il compito di gestire e configurare adeguatamente le piattaforme utilizzate dal gruppo per lo svolgimento del progetto. Da lui dipendono l'affidabilità e l'efficacia dei mezzi scelti per lo svolgimento del progetto;
		\item \textbf{\Ana}: segue il progetto principalmente nelle fasi iniziali ed è fortemente coinvolto nella stesura dell' \AdR{}\textit{\G}. Il suo ruolo è quello di analizzare i problemi posti dal progetto e chiarire le dipendenze e le ramificazioni di ogni attività necessaria alla consegna del prodotto;
		\item \textbf{\Prog}: segue lo sviluppo del progetto e, a partire dai \emph{requisiti}, definisce le scelte tecniche necessarie per lo sviluppo del prodotto;
		\item \textbf{\Progm}: ha il compito di codificare i modelli realizzati dal \Prog{}. Il codice prodotto dal \Progm{} deve attenersi il più possibile alle specifiche elaborate dal \Prog{} e documentare opportunamente il codice creato per aumentarne la manutenibilità;
		\item \textbf{\Ver}: segue l'intero ciclo di vita del progetto. Egli si assicura che la qualità della documentazione prodotta aderisca alla norme stabilite.
	\end{itemize}

\subsubsection{Gestione dei rischi} {
	Nel \PdP{} i rischi sono analizzati e suddivisi in 4 tipi: 
	\begin{itemize}
		\item (RT): rischi tecnologici legati alla gestione e suddivisone del lavoro;
		\item (ROR): rischi organizzativi legati a comprensione ed uso delle tecnologie necessarie;
		\item (RI): rischi interpersonali legati specificatamente all'interazioe tra i membri del gruppo;
		\item (RO): rischi operativi legati alle possibilità di mettere in atto le pratiche di lavoro concordate.
	\end{itemize} 
	Ogni rischio è seguito da un numero (esempio: RT1), il quale parte da 1 ed incrementa. 
	Ogni volta che la tipologia cambia, il numero riparte da 1 e il ragionamento è analogo. 
	Ogni rischio è strutturato nel seguente modo:
	\begin{itemize}
		\item Nome: nome del rischio per identificarlo;
		\item Descrizione: breve descrizione del rischio a cui si va incontro;
		\item Notifica: ogni membro è tenuto a notificare eventuali problemi;
		\item Occorrenza: viene distinta in Bassa, Media ed Elevata; quindi si classifica un rischio in base alla sua occorrenza;
		\item Pericolosità: viene sempre distinta in Bassa, Media ed Elevata; quindi si classifica un rischio in base alla sua pericolosità;
		\item Gestione: viene proposta una soluzione per risolvere il rischio.
	\end{itemize}
}

	\paragraph{Assegnazione dei compiti}
	Per garantire l'efficienza e la flessibilità del processo di sviluppo, le attività necessarie al completamento di un progetto devono essere suddivise in compiti, che possono essere svolti sequenzialmente o in parallelo a seconda delle dipendenze tra di loro. Per ottimizzare tale processo il gruppo utilizza il sistema delle \textit{Issues\G{}} e delle \textit{Pull Request\G{}} offerto da \textit{Github\G}.	
	Una volta individuate dall'\Ana{}, le attività vengono all'occorrenza suddivise in sotto attività e compiti. Il \RdP{} si occupa di assegnare ad ogni membro del team \textit{Issue}\G{} diverse per ottimizzare il progresso del progetto e nel farlo deve valutarne l'idoneità, tenendo conto di disponibilità, competenze tecniche e carico di lavoro attuale del membro in esame. A questo punto, il membro assegnato all'incarico dovrà svolgere il compito nei tempi definiti dalla \textit{Milestone\G{}} collegata alla \textit{Issue}\G{} e, nel caso di contrattempi, notificarlo quanto prima possibile al \RdP{}. Una volta che avrà concluso il proprio lavoro e fatto un commit (che riporterà il numero della \textit{Issue}\G{}) con le relative modifiche, dovrà aprire una \textit{Pull Request\G{}} di verifica per il merge sul ramo \textit{dev}. Un altro membro del gruppo effettuerà l'azione di verifica e controllo delle modifiche verificando la \textit{Pull Request\G{}}. Se il verificatore accetta le modifiche effettuate aggiornerà il registro delle modifiche e la versione del relativo documento modificato e chiuderà la \textit{Pull Request}\G{}. Verrà aperta in seguito, prima della versione di rilascio, una \textit{Pull Request\G{}} di approvazione per il merge sul ramo \textit{master} che dovrà essere approvato da un altro membro del gruppo. Inoltre ogni membro associato a una \textit{Issue}\G{} o a una \textit{Pull Request\G{}} verrà notificato tramite email per ogni modifica relativa alla documentazione %o codice in futuro
	di cui si occupa la \textit{Issue}\G{}.
	\subsubsection{Metriche}
	Il processo di gestione organizzativa non fa uso di metriche qualitative particolari.
	\subsubsection{Strumenti}
Il gruppo \teamname{} utilizza vari strumenti per la comunicazione sia interna che esterna:
	\paragraph{Comunicazione interna}
		Le comunicazioni interne riguardano solamente i membri del gruppo \teamname{} attraverso l'utilizzo dei seguenti strumenti:
			\begin{itemize}
				\item \textbf{Telegram\G}: usato per pianificare incontri su \emph{Discord}\G{} e per comunicazioni rapide e non troppo importanti;
				\item \textbf{Discord\G}: usato per incontri di discussione e pianificazione sul lungo periodo; inoltre è stato integrato con \emph{GitHub\G{}} per la ricezione di messaggi riguardanti lo stato dei repository\G.
			\end{itemize}
	\paragraph{Comunicazioni esterne}
	\label{sec:Comunicazioni esterne}
		Le comunicazioni esterne riguardano le comunicazione tra i membri del gruppo \teamname e persone esterne al \emph{gruppo}:
			\begin{itemize}
				\item \textbf{ProtonMail}: usato per la comunicazione scritta tramite email con l'azienda \proponente e i \emph{committenti};
				\item \textbf{Slack\G}: usato per la comunicazione con il rappresentante dell'azienda \proponente.
			\end{itemize}
