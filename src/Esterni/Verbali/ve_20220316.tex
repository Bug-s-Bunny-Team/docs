\documentclass{classes/base}

\title{Verbale esterno}
\date{2022/03/16}
\author{\marcob}
\verificatore{\tommaso}
\approvatore{\matteo}

\renewcommand{\maketitle}{
    \begin{titlepage}
    \begin{center}
        \makeatletter
        \vspace*{\fill}
        
        %\includegraphics[width=2.5cm]{assets/unipd}
        %\subsection*{Università degli Studi di Padova}
        %\vspace{2cm}
        
        \begin{minipage}[]{0.3\textwidth}
            \centering
            \includegraphics[width=3cm]{assets/unipd}
            \bigskip
        \end{minipage}
        \begin{minipage}[]{0.7\textwidth}
            \centering
            \color[HTML]{B5121B}{
                \textbf{Università degli Studi di Padova} \\
                Ingegneria del Software \\
                Anno Accademico: 2021/2022 \\
                }
                \vspace*{2cm}
        \end{minipage}
        

        \includegraphics[width=5cm]{assets/logo}

        \Huge
        \textbf{\teamname}
        
        \vspace{3cm}
        
        \Huge
        \textbf{\@title}

        \Large
        \@date

        \vspace{3cm}
        
        \textbf{Redazione:} \@author\\
        \textbf{Verifica:} \@verificatore\\
        \textbf{Approvazione:} \@approvatore\\
        
        \vfill
        \makeatother
    \end{center}
\end{titlepage}

}

\begin{document}
    \maketitle

    \section*{Generalità}
    \begin{itemize}
        \item \textbf{Ora inizio:} 14.15
        \item \textbf{Ora fine:} 14.45
    \end{itemize}

    \section*{Presenze}
    \begin{itemize}
     	\item \angela
      	\item \marcob
       	\item \tommaso
        \item \ruth
        \item \matteo
        \item \marcov
        \item \giulio
    \end{itemize}

    \section*{Resoconto}
    Il progetto si divide in due parti, una di backend e una di frontend.\\
    La prima consiste primariamente di un crawler che ha il compito di prelevare i dati dai social (TikTok e Instagram), che andranno poi analizzati ed inseriti in un database.
    Il frontend si occupa semplicemente di mostrare i dati sotto forma di guida tramite una web app.\\
    Il backend è stato presentato come la parte del progetto che richiede più lavoro, in quanto si tratta di una procedura non standard e non supportata nativamente dalle API delle applicazioni in studio.\\
    Per quanto riguarda l'analisi dei video frame by frame (per l'estrapolazione dei dati da TikTok) andranno usati dei tools AWS (ovvero Recognition e Comprehend) che sfruttano reti neurali alle quali è già stato fatto il training per estrarre informazioni dalle immagini.

    \section*{Domande e risposte}
    \begin{itemize}
        \item  \textbf{Il repository è fornito o dobbiamo crearlo?}
        \\\textit{Amazon Code Commit} sarà il nostro repo privato, il quale ha un'integrazione con i servizi di deployment automatici.
        Altri repository (come GitHub) adnranno collegati al repository privato.        
        \item  \textbf{Obbligatori lo sviluppo di app native mobile o si può usare un webapp?}
        \\Una Web App responsive e usabile da mobile è sufficiente.
        \item  \textbf{I linguaggi da utilizzare vengono decisi a monte?} 
        \\Per la parte di backend: Node.js - \textit{javascript/typescript} oppure \textit{Python}.\\
            Per la parte di frontend: \textit{React} oppure \textit{Angular}.\\
            Da usare le lambda functions che implementano un'architettura serverless, meglio se si usano dei linguaggi di scripting come \textit{javascript/javascript} (\textit{NodeJs}) o \textit{Python}.\\
            \textit{AWS Recognition} - riconoscere elementi e luoghi all'interno delle immagini.\\
            \textit{AWS Comprehend} - comprendere la parte di testo, ad esempio se si parla bene o male di un certo luogo o ristorante.
        \item  \textbf{Si dovranno utilizzare solamente API fornite da TikTok/Instagram o si possono usarne di terze parti?}
        \\Non esistono API ufficiali di Insta/TikTok, quindi bisognerà sfogliare da terze parti.\\
            Per il crawling vedere altri crawler online per capire il funzionamento per poi applicarlo al nostro caso.
        \item  \textbf{Nella scelta del WOW è consigliato utilizzare quello presente nel pdf?}
        \\Fondamentale iniziare dalla parte di discover per capire cosa realmente si può fare e analizzare, proprio in virtù del fatto che essendo un progetto di ricerca non è detto che quello che si vuole fare sia possibile. 
            Superata la fase di discover, il resto è a scelta libera        
        \item  \textbf{Come vengono forniti gli applicati da usare (lambda chain, AWS, ...)?}
        \\Dato che sono servizi a pagamento ci verranno forniti dei crediti (1000/2000 dollari) da usare con il nostro account \textit{AWS} (da fare).
            Viene fornito un esempio di file \textit{.json} per la configurazione dell'architettura serverless di Amazon.
        \item  \textbf{Viene fornita formazione da parte dell'azienda? Se sì, come?}
        \\Verranno forniti 2 corsi di formazione: uno per la parte dell'infrastruttura (backend) e una per la parte di machine learning.
            Verrà inoltre instaurato un canale di comunicazione tramite Slack.        
        \item  \textbf{Verrà fornito supporto per la documentazione iniziale?}
        \\Dall'analisi dei requisiti in poi l'azienda è disponibile per dare un supporto diretto (con confronti via call), in modo da costruire sulle basi poste dai corsi di formazione.
        È anche anche stata data disponibilità di aiuto per i documenti di Implementation.
    \end{itemize}

    \section*{Conclusioni}
    L'incontro con il referente Massaro è stato valutato positivamente dai partecipanti in quanto gli strumenti che ci sono stati proposti da utilizzare sono innovativi e, potenzialmente, futuribili (si tratta infatti di applicativi Amazon di machine learning ed neural networks).\\
    Inoltre il fatto che si tratti di un capitolato essenzialmente di ricerca, e quindi potenzialmente molto creativo, è stato trovato stimolante da tutti i membri del gruppo.\\
    Per concludere anche il "lato umano" è stato apprezzato: si è infatti subito instaurato un buon rapporto col referente, anche grazie alla vicinanza d'età, inoltre gli strumenti di comunicazione proposti (chat slack) sono preferibili dalla maggioranza dei membri del gruppo.

\end{document}
