\subsection{Scopo}
Il documento seguente ha lo scopo di descrivere in modo esaustivo e coerente le specifiche e caratteristiche architetturali del prodotto software sviluppato da \teamname{}.

\subsection{Scopo del prodotto}
Lo scopo del \teamname{} e dell'azienda Zero12 è la creazione di un applicativo software (WebApp){\G} in grado di analizzare e 
classificare molteplici contenuti digitali, creati e condivisi dagli utenti sulla piattaforma social Instagram, in base alle reazioni e alle impressioni ricavabili dal contenuto di essi. 
La WebApp{\G} deve poter fornire una guida per ogni luogo di ristorazione, basandosi sulla classificazione precedente. 
E' possibile, quindi, fare una classifica di questi luoghi grazie al contributo degli utenti.

\subsection{Glossario}
Per maggiore chiarezza del lessico usato, è stato creato un glossario, il quale 
contiene spiegazioni dei termini più importanti che sono stati usati.

\subsection{Riferimenti}
\subsubsection{Riferimenti normativi}
\begin{itemize}
	\item
	{\textbf{Capitolato d'appalto C4:}}\\\url{https://www.math.unipd.it/~tullio/IS-1/2021/Progetto/C4.pdf}
    
\end{itemize}
\subsubsection{Riferimenti informativi}
\begin{itemize}
	\item 
    {\textbf{Slide dell'insegnamento di Ingegneria del Software: (Slide 12, 19)}}\\\url{https://www.math.unipd.it/~tullio/IS-1/2021/Dispense/PD2.pdf}
	\item
	{\textbf{Slide Progettazione e programmazione: diagrammi delle classi (UML):}}\\\url{https://www.math.unipd.it/~rcardin/swea/2021/Diagrammi%20delle%20Classi_4x4.pdf}
	\item
	{\textbf{Slide Progettazione: i pattern architetturali:}}\\\url{https://www.math.unipd.it/~rcardin/swea/2022/Software%20Architecture%20Patterns.pdf}
	\item 
	{\textbf{Slide Progettazione: design pattern creazionali:}}\\\url{https://www.math.unipd.it/~rcardin/swea/2022/Design%20Pattern%20Creazionali.pdf}
	\item 
	{\textbf{Slide Progettazione: design pattern strutturali:}}\\\url{https://www.math.unipd.it/~rcardin/swea/2022/Design%20Pattern%20Strutturali.pdf}
	\item 
	{\textbf{Funzioni Lambda in Python}}\\\url{https://docs.aws.amazon.com/lambda/latest/dg/python-handler.html}
	\item 
	{\textbf{Overview AWS Step Functions}}\\\url{https://docs.aws.amazon.com/step-functions/latest/dg/welcome.html#application}
	\item 
	{\textbf{Introduzione FastAPI}}\\\url{https://fastapi.tiangolo.com/tutorial/first-steps/}
	\item
	{\textbf{Overview AWS Fargate}}\\\url{https://docs.aws.amazon.com/AmazonECS/latest/developerguide/AWS_Fargate.html}
	\item 
	{\textbf{Evento \textit{pre signup} Cognito:}}\\\url{https://docs.aws.amazon.com/cognito/latest/developerguide/user-pool-lambda-pre-sign-up.html}
	\item 
	{\textbf{Documentazione SQLAlchemy}}\\\url{https://docs.sqlalchemy.org/en/14/intro.html}
	\item 
	{\textbf{Migrazioni autogenerate Alembic}}\\\url{https://alembic.sqlalchemy.org/en/latest/autogenerate.html}
	\item 
	{\textbf{Analisi facciale Amazon Rekognition}}\\\url{https://docs.aws.amazon.com/rekognition/latest/dg/faces.html}
	\item 
	{\textbf{Sentiment analysis Amazon Comprehend}}\\\url{https://docs.aws.amazon.com/comprehend/latest/dg/how-sentiment.html}
	\item 
	{\textbf{Documentazione Svelte}}\\\url{https://svelte.dev/docs}
\end{itemize}
