Nel Frontend è stato utilizzato il pattern architetturale Model-View-Presenter (MVP), design diffuso nelle WebApp, con lo scopo di separare le componenti di
visualizzazione dalla loro implementazione.
Il pattern si suddivide in tre elementi:
\begin{itemize}
    \item Model: elemento dove sono definiti i dati
    \item View: elemento passivo per la visualizzazione dei dati
    \item Presenter: elemento che fa da mediatore tra la View e il Model tramite data-binding ed eventi, prelevando o modificando dati del Model 
\end{itemize}
Quando l'utente interagisce con l'applicazione, la parte di View è incaricata di visualizzare i dati 
e di notificare le azioni dell'utente, la parte del Presenter fa da tramite per le interazioni tra View e Model, prelevando i dati da quest'ultimo e visualizzandoli, 
oltre che a rispondere in modo corretto agli eventi sollevati dalla View, modificando i dati del Model di conseguenza.
Il Model si occupa della persistenza dei dati oltre che a essere in grado di ottenerli tramite delle chiamate API (tramite metodo POST o GET) che si occuperà di interfacciarsi con
il Backend, il quale restituirà i dati o gli errori risultati dalla chiamata. 

A causa della scelta del framework Svelte\{G} la View non è una classe perchè è un Componente Svelte, 
che si compone da un blocco di script, un blocco di Html ed un blocco di stile.
Lo script viene utilizzato soltanto per inizializzare il Presenter, per il resto la View è passiva come previsto dal pattern MVP. 

La reattività ai cambiamenti dei dati del Model (e anche nel Presenter) non è implementata con una pattern Observer standard,
bensì a livello del singolo campo all'interno dei Model con un tipo speciale `Writable' che può essere osservato tramite il metodo subscribe.

\begin{figure}[!h]
    \includegraphics[width=10cm]{sezioni/images/mvp.png}
    \centering
    \caption{Schema pattern MVP}
\end{figure}

I punti di forza individuati in questo pattern sono che ogni classe Presenter possa gestire 
una classe View alla volta, quindi l'esistenza di una relazione uno a uno e questo permette di avere maggiore 
controllo sulle varie componenti, e la netta separazione presente tra Model e View. 
Quest'ultima risulta un punto chiave perchè permette di facilitare il testing relativo ai Presenter.

\subsection{Diagramma delle classi}
\begin{figure}[!h]
    \includegraphics[width=16cm]{sezioni/images/Main.jpg}
    \centering
    \caption{Frontend - Diagramma delle classi}
\end{figure}

