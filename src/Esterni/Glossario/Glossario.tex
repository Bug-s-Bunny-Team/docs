\documentclass{classes/base}
\usepackage{hyperref}

\title{Glossario}
\author{\giulio, \\ \angela, \\ \marcob}
\verificatore{\marcov, \\ \marcob}
\approvatore{\matteo}
\uso{Esterno}

\setcounter{secnumdepth}{-1}

\begin{document}
	\maketitle
	\newpage
	\tableofcontents

    \section{A}
        \subsection*{API}
        Application Program Interface, è un'interfaccia che offre servizi a computer o software (non ha quindi interazione con utenti).
       
        \subsection*{AWS}
        \label{sec:AWS}
        Amazon Web Services, è una suite di applicativi amazon dedicati alla gestione e sviluppo di applicativi web.

        \subsection*{Analisi dei requisiti (Adr)}
        Il documento descrive in maniera dettagliata i casi d’uso e i requisiti del progetto.

        \subsection*{Amazon API Gateway}
        Amazon API Gateway è un servizio AWS per la creazione, la pubblicazione, la gestione, il monitoraggio e la protezione di API.

        \subsection*{Amazon Appsync}
        AWS AppSync è un servizio completamente gestito che facilita lo sviluppo di API gestendo le attività impegnative derivanti dalla connessione sicura a origini dati come AWS DynamoDB, AWS Lambda.

        \subsection*{Amazon Cognito}
        Amazon Cognito fornisce autenticazione, autorizzazione e gestione degli utenti per le app Web e per dispositivi mobili. Gli utenti possono accedere direttamente con un nome utente e una password, oppure tramite terze parti, ad esempio Facebook, Amazon, Google o Apple.
        
        \subsection*{Amazon Comprehend}
        Amazon Comprehend è un servizio di elaborazione del linguaggio naturale (NLP) che adotta il machine learning per estrarre informazioni dettagliate e collegamenti in un testo.
        
        \subsection*{Amazon CloudFront}
        Amazon CloudFront è un servizio Web che accelera la distribuzione di contenuto Web statico e dinamico, come file immagine, .html, .css e .js, agli utenti.

        \subsection*{Amazon DynamoDB}
        Database fornito da AWS, NoSQL. Esso permette agli sviluppatori di caricare qualsiasi volume di dati garantendo prestazioni elevate per l'esecuzione degli applicativi web.

        \subsection*{Amazon EC2}
        Amazon Elastic Compute Cloud (Amazon EC2) fornisce capacità di calcolo scalabile in AWS Cloud. L'utilizzo Amazon EC2 elimina la necessità di investimenti anticipati in hardware e permette di sviluppare e distribuire più rapidamente le applicazioni.

        \subsection*{Amazon Fargate}
        Amazon Fargate è una tecnologia che permette l'esecuzione di container senza doversi occupare della gestione e della configurazione delle relative macchine virtuali presenti nei server/cluster. 

        \subsection*{Amazon Lambda}
        Servizio di AWS che permette di eseguire codici per qualsiasi tipo di applicazione o servizio senza doversi occupare della gestione del server.

        \subsection*{Amazon RDS}
        Amazon Relational Database Service (Amazon RDS) è un servizio Web che semplifica la configurazione, l'uso e il dimensionamento di un database relazionale in AWS Cloud.

        \subsection*{Amazon Rekognition}
        Servizio di AWS che semplifica l'aggiunta di analisi delle immagini nelle applicazioni. Può rilevare molteplici cose tra cui oggetti, scene e volti.

        \subsection*{Amazon S3}
        \label{sec:S3}
        Servizio di AWS per l'archiviazione di oggetti che offre notevole scalabilità, disponibilità dei dati sicurezza e prestazioni. Offre la possibilità di archiviare facilmente qualsiasi quantità di dati per qualsiasi caso d'uso.
        
        \subsection*{Amazon Secret Manager}
        Servizio di AWS per i sostituire le credenziali hardcoded nel codice, incluse le password, con una chiamata API al servizio stesso per recuperare le credenziali a livello di programmazione. Inoltre permette un rinnovo automatico e pianificato delle credenziali.

        \subsection*{Amazon VPC}
        Amazon Virtual Private Cloud (Amazon VPC) offre il pieno controllo sull' ambiente di reti virtuali, inclusi posizionamento della risorsa, connettività e sicurezza.

        \subsection*{Angular} 
        Framework open source per lo sviluppo di applicazioni web.

        \subsection*{AZ}
        Availability Zone
    
        \newpage  
    \section{B}
    
    	\subsection*{Back-end}
    	Parte non visibile all'utente del programma, che riceve i dati dal Front-end ed elabora/esegue le richieste.

        \subsection*{Ban}
        Letteralmente divieto, espulsione.

    	\subsection*{Baseline}
    	Punto di riferimento, per l'attivita di monitoraggio, rispetto al quale calcolare gli scostamenti delle principali variabili implicate nella gestione di un progetto.
        
        \subsection*{Best practices} 
        Si tratta delle pratiche, abitudini e comportamenti che è buona norma seguire per svolgere un lavoro.

        \subsection*{Bounding box}
        Regione rettangolare di un'immagine, solitamente identificata con le coordinate del sistema di riferimento.

        \subsection*{Branch} 
        Letteralmente “ramo”. Viene usato per isolare il lavoro di sviluppo senza influire su altri nel repository. 
        
        \subsection*{Bucket}
        Termine relativo all'ambito di utilizzo del servizio \hyperref[sec:S3]{\emph{Amazon S3}}.\\
        È un container per caricare i dati (foto, video, documenti ecc.) su \hyperref[sec:S3]{\emph{Amazon S3}} per poi poter essere usato anche da altri servizi \hyperref[sec:AWS]{\emph{AWS}}. 
        
        \newpage  
    \section{C}
        \subsection*{Casi d'uso} 
        E' un tecnica utilizzata per raccogliere requisiti, in maniera esaustiva e dettagliata, al fine di produrre software qualitativo.

        \subsection*{CDN}
        Content Delivery Network, è un gruppo di server distribuiti in tutto il mondo, in modo da coprire più aree geografiche avvicinandosi alle posizioni degli utenti in modo da velocizzare il delivery dei contenuti web.
    
        \subsection*{CIDR}
        Classless Inter Domain Routing, metodo che sfrutta la subnet mask per creare una sottorete partendo da un indirizzo IP.  Questo metodo sostituisce lo schema classful dove gli indirizzi dovevano appartenere ad una specifica classe (A, B e C).
    
        \subsection*{CI/CD}
        Continuous Integration Continuous Deployment, il metodo CI/CD introduce l'automazione costante e il monitoraggio continuo in tutto il ciclo di vita delle applicazioni, dalle fasi di integrazione e test a quelle di distribuzione e deployment.
    
        \subsection*{Cloud}
        Insieme di server distribuiti connessi tra loro tramite un'architettura distribuita. Il loro scopo è quello di erogare servizi come l'archiviazione, l'elaborazione, la trasmissione dati, l'esecuzione di applicazione tramite Internet.
        
        \subsection*{Cluster}
        Un cluster è un insieme di computer connessi tra di loro tramite una rete, con lo scopo di distribuire un'elaborazione molto complessa tra i vari computer, aumentando la potenza di calcolo del sistema e/o garantendo una maggiore disponibilità di servizio.
    
        \subsection*{Commit}
        Una serie di modifiche ad un documento.
        
        \subsection*{Crawler}
        Software o bot per analizzare i contenuti di una rete in modo automatico.
        
        \newpage  
    \section{D}
        \subsection*{Database}
        Un database è un insieme di informazioni (o dati) strutturate in genere archiviate elettronicamente in un sistema informatico.

        \subsection*{Dataset}
        Letteralmente insieme di dati, cioè una collezione di dati più o meno ordinati e di varia natura che servono ad un preciso scopo, per esempio l'addestramento di un modello di apprendimento automatico.
       
        \subsection*{Discord}
        Piattaforma di VoIp, progettata per la comunicazione. Gli utenti possono comunicare tramite chiamate vocali, videochiamate, messaggi di testo, media e file in chat privata come membri di un server Discord.

        \subsection*{DDOS}
        Un attacco di tipo Distributed Denial of Service (DDoS) è un'arma di sicurezza informatica che mira a interrompere le attività aziendali o estorcere denaro alle organizzazioni prese di mira.

        \subsection*{Deploy}
        Il termine deployment in informatica significa distribuzione, ovvero la consegna, l'installazione, la configurazione e la messa in funzione di una applicazione in un sistema informatico.

        \subsection*{Design}
        Il termine design in informatica si riferisce al processo di progettazione di un applicazione software.

        \subsection*{Design pattern}
        in informatica e specialmente nell'ambito dell'ingegneria del software, è un concetto che può essere definito "una soluzione progettuale generale ad un problema ricorrente".
        Si tratta di una descrizione o modello logico da applicare per la risoluzione di un problema che può presentarsi in diverse situazioni durante le fasi di progettazione e sviluppo del software.

        \newpage  
    \section{E}
    
        \newpage  
    \section{F}
        \subsection*{Framework}
        È un'architettura logica di supporto sulla quale un software può essere progettato e realizzato, facilitandone lo sviluppo.
        
        \subsection*{Front-end}
        Parte visibile all'utente di un programma (interfaccia) che permette l'interazioni con esso.
        
        \newpage  
    \section{G}
        
        \subsection*{Gateway}
        In particolare nelle reti inforatiche è un dispositivo di rete che collega due reti di diverso tipo.

        \subsection*{GitHub}
        Servizio di hosting per progetti software ed è una implementazione dello strumento di controllo versione distribuito Git.

        \subsection*{GitKraken}
        È la base per gli sviluppatori che cercano un'interfaccia più adatta a Git, con interazioni a GitHub. 

        \subsection*{Git}
        Git è un software per il controllo di versione distribuito utilizzabile da interfaccia a riga di comando, creato da Linus Torvalds.
        
        \subsection*{Google Docs}
        Servizio web di Google per la elaborazione di testi, fogli di calcolo, presentazioni e sondaggi (suite per ufficio).
        
        \subsection*{Google Drive}
        Servizio web di Google per la memorizzazione e sincronizzazione su cloud.

        \subsection*{GraphQL}
        Linguaggio per interrogare lato server per API, è in grado di fornire ai client solo idati di cui si ha bisogno.

        \subsection*{Guida Michelin} 
        Insieme di pubblicazioni annuali rivolte all'enogastronomia di un determinato paese. È un riferimento per la valutazione della qualità dei ristoranti.
        \newpage  
    \section{H}
    \newpage  
    \section{I}
        \subsection*{Issue}
        Compito, questione o semplice domanda relativa al repository di riferimento.
        
        \subsection*{Instagram}
        Servizio di rete sociale statunitense che permette agli utenti, tra le varie features, di scattare foto, applicarvi filtri e condividerle via Internet.
        
        \subsection*{Interfaccia}
        Collegamento tra sistemi diversi per permettere loro lo scambio di informazioni.

        \subsection*{IP}
        Internet Protocol, protocollo di rete TCP/IP che si occupa dell'instradamento delle informazioni.

        \subsection*{Istanza}
        in informatica, una determinata esecuzione di un processo, solitamente caratterizzata da una propria configurazione. In pratica, si tratta di una specifica installazione o implementazione di un programma di sistema.
        
        \subsection*{ISO/IEC 12207:1995}   
        ISO 12207 è uno standard dell'ISO per la gestione del ciclo di vita del software. 
        Si propone di diventare lo standard di riferimento definendo tutte le attività svolte nel processo di sviluppo e mantenimento del software. 

        \subsection*{ISO/IEC 15504}
        ISO/IEC 15504, anche conosciuta come SPICE (Software Process improvement and Capability Determination),
        è un insieme di nove documenti di standard tecnici relativi ai processi di sviluppo del software e relative funzioni di business e, in particolare, alla loro valutazione.
        
        \subsection*{ISO/IEC 25000:2005}
        ISO 25000/IEC vuole dare un contributo alla sicurezza, alla funzionalità e manutenibilità del prodotto software, alla accuratezza dei dati, al raggiungimento della soddisfazione dell'utente, che usa servizi, in un'ottica preventiva e di qualità misurabile.
        
        \newpage  
    \section{J}
    \newpage  
    \section{K}
    \newpage  
    \section{L}
        \subsection*{Label (classificazione)}
        Letteralmente etichetta, cioè la classe a cui appartiene un'istanza di un dataset.
        
        \subsection*{LaTeX}
        Sistema software per la preparazione dei documenti in alta qualità.

    \newpage  
    \section{M} 
        \subsection*{Metrica}
        Qualsiasi tipo di misura correlata a un sistema, un processo software  o  documentazione  di  un  prodotto  SW. 

        \subsection*{Milestone}
        In inglese significa pietra miliare, ed indica un traguardo/obbiettivo presente nello svolgimento del progetto.
        
        \subsection*{Modello a V}
        Modello di sviluppo del software che, dopo la fase di programmazione (discesa), risale creando una forma a V. Il modello dimostra la relazione tra ogni fase del ciclo di vita dello sviluppo del software e la sua fase di testing.

        \subsection*{Multi AZ RDS} 
        In una implementazione di Amazon RDS Multi-AZ, Amazon RDS crea automaticamente un’istanza del database primario (DB) e replica in maniera sincrona i dati su un’istanza in una zona di disponibilità differente.

        \subsection*{MVP}
        Minimum viable product ossia prodotto minimo funzionante. \\
        È la versione di un prodotto con caratteristiche appena sufficienti per essere utilizzabile dai primi clienti, che possono fornire un feedback sullo sviluppo futuro del prodotto stesso.

        \subsection*{MySQL}
        Sistema di gestione di database relazionali gratuito ed open source che utilizza SQL.
        \newpage  
    \section{N}
        \subsection*{NAT}
        Network Address Translation, metodo con il quale si può modificare l'indirizzo sorgente/destinatario di un pacchetto nella comunicazione tra due client posti in reti diverse.

        \subsection*{Norme di progetto (NdP)} 
        E' un documento interno che serve a istituire un Way Of Woking. Esso è un documento incrementale, perchè aggiungo regole ogni qualvolta ne necessito.
        
        \subsection*{Neptune}
        Amazon Neptune è un servizio di database a grafo rapido, affidabile e completamente gestito che semplifica la creazione e l'esecuzione di applicazioni.

        \subsection*{Network}
        Sostanzialmente una rete di host collegati e in grado di comunicare tra loro.

        \subsection*{Natural Language Processing (NLP)}
        Analisi del linguaggio naturale, solitamente rappresentato attraverso testo, con l'obiettivo di estrarne informazioni.

        \subsection*{NodeJS}
        Ambiente di esecuzione che permette di eseguire codice Javascript come un qualsiasi linguaggio di programmazione.
        \newpage  
    \section{O}
        \subsection*{Open source}
        Programma del quale è disponibile pubblico il codice sorgente.
        \newpage  
    \section{P}
        \subsection*{Piano di progetto (PdP)}
        Documento che fornisce in maniera dettagliata la pianificazione delle attività di progetto, con i rispettivi obiettivi da raggiungere. 

        \subsection*{Piano di qualifica (PdQ)} 
        Documento che definisce e documenta le strategie inerenti a V\&V (verifica e validazione) necessarie ad assicurare qualità di processo e qualità di prodotto.

        \subsection*{Pipeline}
        Indica una catena di montaggio, in particolare dei componenti software collegati tra loro a cascata in modo che il risultato prodotto da un elemento sia l'ingresso di quello successivo.

        \subsection*{Plugin}
        Software non autonomo, interagisce con un altra applicazione per estenderne le capacità.

        \subsection*{PostgreSQL}
        È un sistema di database relazionale a oggetti open source.

        \subsection*{Proof of concept (PoC)}
        È un prototipo concettuale per valutare, in fase pre-decisionale, le potenzialità di sviluppo di un’idea e i relativi investimenti. Proof of concept significa letteralmente “prova di concetto”, perché ciò che viene simulata è l’idea del nuovo prodotto o servizio.
        
        \subsection*{Processo}
        Insieme di attività correlate che trasformano ingressi in uscite, consumando risorse nel farlo.
        
        \subsection*{Pull Request}
        Le Pull Request ti consentono di comunicare agli altri le modifiche che hai inviato a un ramo in un repository su GitHub. Una volta aperta una Pull Request, puoi discutere e rivedere le potenziali modifiche con i collaboratori e aggiungere commit prima che le modifiche vengano unite nel ramo di base o in quello desiderato.

        \newpage  
    \section{Q}
    \newpage  
    \section{R}
        \subsection*{Ranking} 
        Classifica, graduatoria di merito. 

        \subsection*{Repository}
        Area di gestione principale in cui sono contenuti tutti i progetti e che aiuta a gestire il flusso di lavoro.
        
        \subsection*{React}
        Libreria Javascript indirizzata alla creazione di interfacce utente.

        \subsection*{REST}
        Il termine REST rappresenta un sistema di trasmissione di dati su HTTP senza ulteriori livelli. I sistemi REST non prevedono il concetto di sessione.

        \subsection*{RTB}
        Requirements Technology Baseline
        \newpage  
    \section{S}
        \subsection*{Scaling}       
        Dall'inglese ridimensionare, determina la capacità di un sistema/prodotto ad aumentare a diminuire di scala in funzione delle necessità e disponibilità.

        \subsection*{Schedule} 
        Il termine significa organizzare, pianificare eventi.

        \subsection*{Schedulato}
        Il termine si utilizza per indicare un evento preventivamente organizzato e prefissato.
        
        \subsection*{Scraper}
        Programma o bot per l'estrazione di dati da un sito web che simula la navigazione umana.
        
        \subsection*{Serverless}
        Modello di esecuzione cloud dove il provider del servizio alloca le risorse della macchina appena queste vengono richieste. Quando un app non è in uso, nessuna risorsa viene consumata e il prezzo è basato solo sulle risorse utilizzate. I programmatori non devono più occuparsi di mantenimento o configurazione.

        \subsection*{Slack} 
        Applicazione che permette di inviare messaggi in tempo reale ai membri del gruppo e viene usato come strumento di collaborazione aziendale.

        \subsection*{Software suite} 
        Collezione di programmi e applicativi correlati tra loro.
        
        \subsection*{Sprint}
        Unità di base dello sviluppo in Scrum ed è di durata fissa, generalmente da una a quattro settimane.

        \subsection*{Standard}
        Modello in cui viene uniformata un determinata attività.

        \subsection*{Stakeholder} 
        Dall'inglese portatore d'interesse, si riferisce ai soggetti direttamente interessati ed altamente coinvolti nel buon andamento dell'applicazione.

        \subsection*{Subnet}
        E' divisione logica di rete IP, e ulteriore partizione di una vpc.
        \newpage  
    \section{T}
        \subsection*{Telegram}
        Applicazione che permette di inviare messaggi ed effettuare chiamate a più utenti contemporaneamente in tempo reale, creando anche dei gruppi privati.

        \subsection*{Terminal} 
        Interfaccia a linea di comando per interagire con il sistema operativo.
        
        \subsection*{Tik Tok}
        Social network cinese che permette di creare brevi clip musicali di durata variabile (dai 15 ai 180 secondi) ed eventualmente modificare la velocità di riproduzione, aggiungere filtri, effetti particolari e suoni ai loro video.
        
        \newpage  
    \section{U}
        \subsection*{UML}
		È un linguaggio di modellazione e di specifica basato sul paradigma orientato agli oggetti per l'architettura, la progettazione e l'implementazione di sistemi software.

        \subsection*{Use cases} 
        Dall'inglese Casi d'uso, è un tecnica utilizzata per raccogliere requisiti, in maniera esaustiva e dettagliata, al fine di produrre software qualitativo.

        \subsection*{Unit test}
        Per unit testing si intende l'attività di testing (prova, collaudo) di singole unità software. 
        Per unità si intende normalmente il minimo componente di un programma dotato di funzionamento autonomo; a seconda del paradigma di programmazione.
        
        \newpage  
    \section{V}

        \subsection*{VPC}
        Virtual Private Cloud.

        \subsection*{Validazione} 
        Conferma finale, accerta la conformità di un prodotto alle attese e coinvolge il committente.

        \subsection*{Verifica}
        Attiene alla coerenza e all completezza del prodotto. E’ un processo cche si applica ad ogni “segmento” temporale di un progetto per accertare che le attività svolte in tale segmento non abbiano introdotto errori.

        \subsection*{VSC - Visual Studio Code}
        Editor di codice sorgente sviluppato da Microsoft per Windows, Linux e macOS. È un software libero e gratuito, anche se la versione ufficiale è sotto una licenza proprietaria.

        \newpage  
    \section{W}
        \subsection*{WOW - Way Of Working}
        Traducibile con “il modo in cui un gruppo decide di lavorare” ed è un elemento fondamentale per aumentare la probabilità di successo di un team nella realizzazione di un progetto o per il raggiungimento di un obiettivo.  
        \newpage  
    \section{X}
    \newpage  
    \section{Y}
        \subsection*{YAML}
        È un formato per la serializzazione di dati utilizzabile da esseri umani.
        
        \newpage  
    \section{Z}
\end{document}
