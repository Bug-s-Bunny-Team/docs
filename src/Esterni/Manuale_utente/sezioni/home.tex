\section{Home} {
    Quando l'utente entra nella piattaforma si trova nella pagina iniziale \textbf{Home}". \aCapo
    Se non ha deciso in precedenza la visione predefinita della guida, visualizzerà i luoghi sottoforma di lista. 
    Direttamente dalla pagina Home, si può cambiare la visulizzazione in mappa e viceversa(da mappa in lista), cliccando tale bottone:
    FOTO

    \subsection{Visualizzazione guida come lista} {
        L'utente vedrà un elenco di posti, generati dalle recensioni dei profili social che segue. \aCapo
        Ogni luogo è caratterizzato da tali informazioni: 
        \begin{itemize}
            \item Location: si tratta del nome del posto ed è un link, il quale una volta cliccato aprirà un popup con maggiori informazioni(§8.3); 
            \item Score: si tratta del punteggio che è attribuito al posto, su una scala da 0 a 5 stelle.
        \end{itemize}       
    }

    \subsection{Visualizzazione guida come mappa} {
        L'utente visulizzerà una mappa vera e propria con un "segnaposto" per la locazione di ogni posto. È possibile spostarsi sulla mappa, ingrandendo o diminuendo la visulizzazione. 
        Sono presenti infatti anche i tasti "+" e "-" per facilitare l'operazione di zoom. \aCapo
        Se il "segnaposto" viene cliccato compare il nome del luogo e il suo punteggio su una scala da 0 a 5. Il nome del posto, se cliccato, farà aprire un popup con maggiori informazioni(§8.3).

    }

    \subsection{Popup con maggiori informazioni} {
        Le informazioni che compaiono sono le seguenti:
        \begin{itemize}
            \item Il nome del posto;
            \item L'indirizzo dove è situato il posto;
            \item Il suo punteggio su una scala da 0 a 5 stelle;
            \item Una foto carattestica di quel posto.
        \end{itemize}

        Per chiudere la finestra è necessario cliccare il tasto "\textbf{Close}" situato in alto a sinistra.
        FOTO ESPLICATIVE
    }
}
