\subsection{Obiettivi del prodotto}
L'obiettivo del progetto è la realizzazione di un portale web che renda disponibile, 
sotto forma di rating e recensioni in stile Guida Michelin, dati riguardanti ristoranti 
e locali presi direttamente dai social network Instagram e TikTok.

\subsection{Funzionalità del prodotto}
La piattaforma prevede una tipologia di utente interno, in grado di poter visualizzare ed interagire con i luoghi visualizzati in una mappa. 
L’utente può visualizzare recensioni legate ad un luogo di interesse oppure ad uno specifico utente esterno, dal quale il backend prende le informazioni necessarie.

\subsection{Piattaforme}
Il progetto sarà fruibile tramite browser web, e si appoggerà per il suo backend a una suite di servizi Amazon AWS.

\subsection{Caratteristiche degli utenti}
L'utente, nel nostro applicativo, ha funzionalità limitate classicamente paragonabili a quelle di un normale utente social.
Può visualizzare i contenuti già disponibili sulla piattaforma e filtrare i risultati secondo luogo o profilo social esterno dal quale vengono presi i dati stessi.
Inoltre può richiedere che vengano presi dei dati anche da profili non ancora presenti.

\subsection{Vincoli generali}
\begin{itemize}
	\item Uso di un'architettura a microservizi;
	\item Realizzazione di una WebApp responsive;
\end{itemize}
