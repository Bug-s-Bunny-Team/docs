\subsection{Obiettivi del prodotto}
L'obiettivo del prodotto è la realizzazione di un portale web che renda disponibile, sotto forma di rating e recensioni in stile Guida Michelin\G{}, dati riguardanti 
ristoranti e altre tipologie affini di locali. 
Questi dati dovranno essere estrapolati direttamente dai post presenti nelle piattaforme social, dai quali saranno predisposte procedure automatizzate per l'ottenimento e l'analisi degli stessi.

\subsection{Funzionalità del prodotto}
L'utente della piattaforma sarà in grado di visualizzare ed interagire con i luoghi visualizzati in una mappa o in una lista, a seconda delle preferenze espresse.
Potrà inoltre visualizzare recensioni legate ad un luogo di interesse oppure ad uno specifico profilo social, dal quale il backend\G{} provvede ad ottenere le informazioni necessarie.

\subsubsection{Scraping dei dati dalle piattaforme social}
Lo scraping\G{}, o crawling\G{}, avverrà soltanto in profili social disponibili pubblicamente, e non saranno predisposte procedure per ottenere informazioni da profili privati.
Saranno inoltre tenuti in considerazione, ed opportunamente mitigati, i rischi conseguenti alle operazioni che andranno eseguite sulle piattaforme social, in particolare il rischio di ban\G{} da esse.\\
A seguito delle analisi condotte dal team, e in accordo con il proponente, il progetto si occuperà dell'ottenimento dei dati solo dalla piattaforma social Instagram\G{}. Questo perchè la piattaforma TikTok\G{} si è dimostrata essere eccessivamente complessa da trattare.

\subsubsection{Analisi dei dati ottenuti}
I dati raccolti saranno opportunamente trattati ed analizzati in modo da estrarre informazioni utili allo scopo del prodotto. 
In particolare, verranno utilizzati i servizi Amazon Comprehend\G{} e Rekognition\G{}.
Dopo l'analisi, le informazioni ottenute verranno opportunamente memorizzate.

\subsection{Caratteristiche degli utenti}
L'utente della nostra piattaforma, denominato semplicemente \textit{utente}, ha funzionalità limitate classicamente paragonabili a quelle di un normale utente social.
Può visualizzare i contenuti già disponibili sulla piattaforma e filtrare i risultati secondo luogo o profilo social dal quale provengono i dati stessi.
Inoltre può richiedere che vengano presi dei dati anche da profili non ancora presenti.
Gli utenti delle piattaforme social, cioè quelli da cui provengono le informazioni, sono denomitati \textit{profili social}.

\newpage
\subsection{Vincoli generali}
\begin{itemize}
	\item Uso della suite di servizi Amazon AWS\G{};
	\item Uso di un'architettura a microservizi;
	\item Realizzazione di una WebApp responsive.
\end{itemize}
