\subsection{Introduzione}
Questa sezione ha lo scopo di descrivere i vari casi d'suo che sono stati identificati dal gruppo Bug's Bunny 
come delle potenziali funzionalità dell'applicazione.

\subsection{Attori}
\begin{itemize}
    \item \textbf{Utente non registrato}: 
    è un utente che non ha ancora effettuato la registrazione presso l'applicativo. 
    Non possiede credenziali per effettuare l'autenticazione presso la piattaforma e non ha accesso a nessuna funzionalità dell'applicazione  
    \item \textbf{Utente non autenticato}
    è un utente che ancora deve autenticarsi nell'applicazione. Può essere in possesso di credenziali di accesso oppure no.
    \item \textbf{Utente autenticato}
    è un utente che ha effettuato l'autenticazione della piattaforma ed ha accesso alle funzionalità di essa.
\end{itemize}

\begin{figure}[!h]
    \includegraphics[width=10cm]{sezioni/Images/Actors.png}
    \centering
    \caption{Gerarchia attori}
\end{figure}
\newpage
    
\subsection{UC 1 - Registrazione manuale}

\begin{figure}[!h]
    \includegraphics[width=15cm]{sezioni/Images/UC1.png}
    \centering
    \caption{UC 1 - Registrazione manuale}
\end{figure}

\begin{itemize}
    \item \textbf{Attore}: utente non registrato.
    \item \textbf{Descrizione}: l’utente deve poter avere la possibilità di creare un account personale.
    \item \textbf{Scenario}:
    \begin{enumerate}
        \item l’utente si collega al sistema;
        \item l’utente clicca sul pulsante di registrazione;
        \item l’utente inserisce la propria email \textbf{(UC 1.1)};
        \item l’utente inserisce il proprio nome utente \textbf{(UC 1.2)};
        \item l’utente inserisce una password \textbf{(UC 1.3)};
        \item l’utente conferma i dati inseriti per proseguire.
    \end{enumerate}
    \item \textbf{Estensioni}:\\
        - l’utente inserisce una mail non valida  \textbf{(UC 2)};\\
        - l’utente inserisce il nome utente non valido \textbf{(UC 3)};\\
        - l’utente inserisce una password non valida \textbf{(UC 4)};\\

    \item \textbf{Precondizioni}: l’utente non è ancora registrato nel sistema.
    \item \textbf{Postcondizioni}: l’utente è registrato nel sistema.
\end{itemize}

\subsubsection{UC 1.1 - Inserimento email}
\begin{itemize}
    \item \textbf{Attore}: utente non registrato.
    \item \textbf{Descrizione}: l’utente deve avere la possibilità di inserire l’email.
    \item \textbf{Scenario}:
    \begin{enumerate}
        \item l’utente seleziona il campo relativo alla mail;
        \item l’utente inserisce la propria email.
    \end{enumerate}

    \item \textbf{Precondizioni}: l’utente effettua l’attività di registrazione.
    \item \textbf{Postcondizioni}: l’utente ha compilato il campo relativo alla mail.
\end{itemize}

\subsubsection{UC 1.2 - Inserimento nuovo nome utente}
\begin{itemize}
    \item \textbf{Attore}: utente non registrato.
    \item \textbf{Descrizione}: l’utente deve avere la possibilità di creare un nuovo nome utente.
    \item \textbf{Scenario}:
    \begin{enumerate}
        \item l’utente seleziona il campo relativo il nuovo nome utente;
        \item l’utente crea il proprio nuovo nome utente.
    \end{enumerate}

    \item \textbf{Precondizioni}: l’utente effettua l’attività di registrazione.
    \item \textbf{Postcondizioni}: l’utente ha compilato il campo relativo il nuovo nome utente.
\end{itemize}

\subsubsection{UC 1.3 - Inserimento nuova password}
\begin{itemize}
    \item \textbf{Attore}: utente non registrato.
    \item \textbf{Descrizione}: l’utente deve avere la possibilità di creare una nuova password.
    \item \textbf{Scenario}:
    \begin{enumerate}
        \item l’utente seleziona il campo relativo la nuova password;
        \item l’utente crea la nuova password.
    \end{enumerate}

    \item \textbf{Precondizioni}: l’utente effettua l’attività di registrazione.
    \item \textbf{Postcondizioni}: l’utente ha compilato il campo relativo la nuova password.
\end{itemize}


\begin{figure}[!h]
    \includegraphics[width=15cm]{sezioni/Images/UC1_s.png}
    \centering
    \caption{Registrazione manuale}
\end{figure}

\subsection{UC 2 - Visualizzazione errore di inserimento email in registrazione}
\begin{itemize}
    \item \textbf{Attore}: utente non registrato.
    \item \textbf{Descrizione}: l’utente deve essere notificato con un errore nel caso in cui le informazioni inserite nel campo email siano invalide.
    \item \textbf{Scenario}: L’utente visualizza un messaggio di errore.
    \item \textbf{Precondizioni}: l’utente inserisce una mail non valida nell'apposito input box.
    \item \textbf{Postcondizioni}: l’utente riceve il messaggio di errore.
\end{itemize}

\subsection{UC 3 - Visualizzazione errore di inserimento nome utente in registrazione}
\begin{itemize}
    \item \textbf{Attore}: utente non registrato.
    \item \textbf{Descrizione}: l’utente deve essere notificato con un errore nel caso in cui venga inserito un nome utente non valido.
    \item \textbf{Scenario}: L’utente visualizza un messaggio di errore.
    \item \textbf{Precondizioni}: l’utente inserisce un nome utente non valido nell'apposito input box.
    \item \textbf{Postcondizioni}: l’utente riceve il messaggio di errore.
\end{itemize}

\subsection{UC 4 - Visualizzazione errore di inserimento password in registrazione}
\begin{itemize}
    \item \textbf{Attore}: utente non registrato.
    \item \textbf{Descrizione}: l’utente deve essere notificato con un errore nel caso in cui inserisca una password non valida durante la fase di registrazione.
    \item \textbf{Scenario}: L’utente visualizza un messaggio di errore.
    \item \textbf{Precondizioni}: l’utente inserisce una password non valida nell'apposito input box.
    \item \textbf{Postcondizioni}: l’utente riceve il messaggio di errore.
\end{itemize}

\subsection{UC 5 - Login manuale}

\begin{figure}[!h]
    \includegraphics[width=10cm]{sezioni/Images/UC5.png}
    \centering
    \caption{UC 5 - Login manuale}
\end{figure}

\begin{itemize}
    \item \textbf{Attore}: l’utente non è autenticato.
    \item \textbf{Descrizione}: l’utente accedendo con l’account personale, deve poter autenticarsi.
    \item \textbf{Scenario}:
    \begin{enumerate}
        \item l’utente si collega al sistema;
        \item l’utente clicca sul pulsante di login;
        \item l’utente inserisce il proprio nome utente \textbf{(UC 5.1)};
        \item l’utente inserisce la propria password \textbf{(UC 5.2)};
        \item l’utente decide se memorizzare la sessione \textbf{(UC 5.3 )};
        \item l’utente clicca il pulsante di conferma per proseguire.
    \end{enumerate}
    \item \textbf{Estensioni}:\\
        - l’utente inserisce il nome utente non valido e viene segnalato con un messaggio d’errore \textbf{(UC 6)};\\
        - l’utente inserisce la password non valida e viene segnalato con un messaggio d’errore \textbf{(UC 7)};\\

    \item \textbf{Precondizioni}: l’utente non è ancora autenticato.
    \item \textbf{Postcondizioni}: l’utente si è autenticato al sistema.
\end{itemize}

\begin{figure}[!h]
    \includegraphics[width=10cm]{sezioni/Images/UC5_s.png}
    \centering
    \caption{UC 5 - Login manuale}
\end{figure}

\subsubsection{UC 5.1 - Inserimento nome utente}
\begin{itemize}
    \item \textbf{Attore}: utente non autenticato.
    \item \textbf{Descrizione}: l’utente deve poter inserire il nome utente per autenticarsi.
    \item \textbf{Scenario}:
    \begin{enumerate}
        \item l’utente seleziona il campo relativo al nome utente;
        \item l’utente inserisce il nome utente.
    \end{enumerate}

    \item \textbf{Precondizioni}: l’utente effettua l’attività di autenticazione.
    \item \textbf{Postcondizioni}: l’utente ha compilato il campo relativo il nome utente.
\end{itemize}

\subsubsection{UC 5.2 - Inserimento password}
\begin{itemize}
    \item \textbf{Attore}: utente non autenticato.
    \item \textbf{Descrizione}: l’utente deve poter inserire la password per autenticarsi.
    \item \textbf{Scenario}:
    \begin{enumerate}
        \item l’utente seleziona il campo relativo alla password;
        \item l’utente inserisce la password.
    \end{enumerate}

    \item \textbf{Precondizioni}: l’utente effettua l’attività di autenticazione.
    \item \textbf{Postcondizioni}: l’utente ha compilato il campo relativo il nome utente.
\end{itemize}

\subsubsection{UC 5.3 Memorizzazione sessione}
\begin{itemize}
    \item \textbf{Attore}: utente non autenticato.
    \item \textbf{Descrizione}: l’utente deve poter memorizzare la sessione.
    \item \textbf{Scenario}: l’utente spunta la casella per mantenere memorizzata la sessione.
    \item \textbf{Precondizioni}: l’utente effettua l’attività di autenticazione.
    \item \textbf{Postcondizioni}: l’utente ha chiesto che la sessione venga memorizzata.
\end{itemize}

\subsection{UC 6 - Visualizzazione errore di inserimento mail in autenticazione}
\begin{itemize}
    \item \textbf{Attore}: utente non autenticato.
    \item \textbf{Descrizione}: l’utente deve essere notificato con un errore nel caso in cui l’email inserita sia invalida durante la fase di autenticazione.
    \item \textbf{Scenario}: L’utente visualizza un messaggio di errore. 
    \item \textbf{Precondizioni}: l’utente effettua l’attività di autenticazione ed inserisce l’email non valida.
    \item \textbf{Postcondizioni}: l’utente riceve il messaggio di errore.
\end{itemize}

\subsection{UC 7 - Visualizzazione errore di inserimento password in autenticazione} 
\begin{itemize}
    \item \textbf{Attore}: utente non autenticato.
    \item \textbf{Descrizione}: l’utente deve essere notificato con un errore nel caso in cui la password inserita sia invalida durante la fase di autenticazione.
    \item \textbf{Scenario}: L’utente visualizza un messaggio di errore. 
    \item \textbf{Precondizioni}: l’utente effettua l’attività di autenticazione ed inserisce la password non valida.
    \item \textbf{Postcondizioni}: l’utente riceve il messaggio di errore.
\end{itemize}

\subsection{UC 8 - Login automatico}
\begin{itemize}
    \item \textbf{Attore}: utente non autenticato.
    \item \textbf{Descrizione}: l’utente accedendo con l’account personale, deve poter autenticarsi automaticamente se la sessione rimane memorizzata.
    \item \textbf{Scenario}:
    \begin{enumerate}
        \item l’utente si collega al sistema;
        \item l’utente viene automaticamente autenticato.
    \end{enumerate}

    \item \textbf{Precondizioni}: l’utente ha una sessione attiva memorizzata.
    \item \textbf{Postcondizioni}: l’utente si è autenticato al sistema.
\end{itemize}

\subsection{UC 9 - Logout}
\begin{itemize}
    \item \textbf{Attore}: l’utente è autenticato.
    \item \textbf{Descrizione}: l’utente deve poter uscire dalla sessione.
    \item \textbf{Scenario}:
    \begin{enumerate}
        \item l’utente è collegato al sistema;
        \item l’utente clicca il pulsante di logout.
    \end{enumerate}

    \item \textbf{Precondizioni}: l’utente è autenticato.
    \item \textbf{Postcondizioni}: l’utente non è più autenticato al sistema.
\end{itemize}

\subsection{UC 10 - Modifica password}

\begin{figure}[!h]
    \includegraphics[width=10cm]{sezioni/Images/UC10.png}
    \centering
    \caption{Modifica password}
\end{figure}

\begin{itemize}
    \item \textbf{Attore}: l’utente è autenticato.
    \item \textbf{Descrizione}: l’utente deve poter modificare l’attuale password, con la quale effettua il login.
    \item \textbf{Scenario}:
    \begin{enumerate}
        \item Inserimento password attuale;
        \item Inserimento nuova password;
        \item Conferma nuova password;
        \item Conferma modifica password.
    \end{enumerate}
    \item \textbf{Estensioni}:\\
        - l’utente inserisce l’attuale password e non coincide con quella attuale, viene visualizzato un messaggio di errore \textbf{(UC 11)};\\
        - l’utente inserisce la nuova password ma essa non rispetta determinate condizioni quindi viene visualizzato un messaggio di errore \textbf{(UC 12)};\\
        - l’utente non inserisce in modo corretto la nuova password nella conferma della stessa, quindi viene visualizzato un messaggio di errore \textbf{(UC 13)};\\

    \item \textbf{Precondizioni}: l’utente è autenticato con una determinata password.
    \item \textbf{Postcondizioni}: l’utente è autenticato con una nuova password.
\end{itemize}

\begin{figure}[!h]
    \includegraphics[width=10cm]{sezioni/Images/UC10_s.png}
    \centering
    \caption{Modifica password}
\end{figure}

\subsubsection{UC 10.1 - Inserimento password attuale} 
\begin{itemize}
    \item \textbf{Attore}: l’utente è autenticato.
    \item \textbf{Descrizione}: l’utente deve inserire la password attuale durante la sessione di “modifica password”.
    \item \textbf{Scenario}:
    \begin{enumerate}
        \item l’utente seleziona il campo riferito alla password attuale.
        \item l’utente inserisce la password attuale.
    \end{enumerate}

    \item \textbf{Precondizioni}: l’utente svolge la sessione di modifica password.
    \item \textbf{Postcondizioni}: l’utente ha inserito la propria password attuale.
\end{itemize}

\subsubsection{UC 10.2 - Inserimento nuova password}
\begin{itemize}
    \item \textbf{Attore}: l’utente è autenticato.
    \item \textbf{Descrizione}: durante l’attività di modifica della password l’utente deve poter inserire la nuova password.
    \item \textbf{Scenario}:
    \begin{enumerate}
        \item l’utente seleziona il campo riferito alla nuova password;
        \item l’utente inserisce la nuova password.
    \end{enumerate}

    \item \textbf{Precondizioni}: l’utente svolge la sessione di modifica password.
    \item \textbf{Postcondizioni}: l’utente ha inserito la nuova password.
\end{itemize}

\subsubsection{UC 10.3 Inserimento ripetizione nuova password}
\begin{itemize}
    \item \textbf{Attore}: l’utente è autenticato.
    \item \textbf{Descrizione}: durante l’attività di modifica password l’utente deve poter inserire la ripetizione della nuova password.
    \item \textbf{Scenario}:
    \begin{enumerate}
        \item l’utente seleziona il campo riferito alla ripetizione della nuova password;
        \item l’utente inserisce la ripetizione della nuova password.
    \end{enumerate}

    \item \textbf{Precondizioni}: l’utente svolge la sessione di modifica password.
    \item \textbf{Postcondizioni}: l’utente ha inserito la ripetizione della nuova password.
\end{itemize}

\subsubsection{UC 10.4 - Conferma modifica password}
\begin{itemize}
    \item \textbf{Attore}: l’utente è autenticato.
    \item \textbf{Descrizione}: durante l’attività di modifica password l’utente deve poter confermare la nuova password.
    \item \textbf{Scenario}: l’utente conferma la nuova password inserita. 
    \item \textbf{Precondizioni}: l’utente svolge la sessione di modifica password.
    \item \textbf{Postcondizioni}: l’utente ha provato a modificare la propria password.
\end{itemize}

\subsection{UC 11 - Visualizzazione errore di inserimento password attuale}
\begin{itemize}
    \item \textbf{Attore}: l’utente è autenticato.
    \item \textbf{Descrizione}: durante l’attività di modifica password l’utente deve ricevere un messaggio d’errore se l'inserimento della password attuale non è andato a buon fine.
    \item \textbf{Scenario}: l’utente legge un messaggio d’errore. 
    \item \textbf{Precondizioni}: l’utente svolge la sessione di modifica password e la password attuale inserita non è corretta.
    \item \textbf{Postcondizioni}: l’utente ha ricevuto un messaggio d’errore.
\end{itemize}

\subsection{UC 12 - Visualizzazione errore di inserimento nuova password non valida}
\begin{itemize}
    \item \textbf{Attore}: l’utente è autenticato.
    \item \textbf{Descrizione}: durante l’attività di modifica password l’utente deve ricevere un messaggio d’errore se l'inserimento di una nuova password non è valida.
    \item \textbf{Scenario}: l’utente legge un messaggio d’errore. 
    \item \textbf{Precondizioni}: l’utente svolge la sessione di modifica password e la nuova password inserita non è valida.
    \item \textbf{Postcondizioni}: l’utente ha ricevuto un messaggio d’errore.
\end{itemize}

\subsection{UC 13 - Visualizzazione errore di ripetizione nuova password}
\begin{itemize}
    \item \textbf{Attore}: l’utente è autenticato.
    \item \textbf{Descrizione}: durante l’attività di modifica password l’utente deve ricevere un messaggio d’errore se l'inserimento della ripetizione della nuova password non è uguale alla nuova password.
    \item \textbf{Scenario}: l’utente legge un messaggio d’errore. 
    \item \textbf{Precondizioni}: l’utente svolge la sessione di modifica password e la ripetizione della nuova password inserita non è corretta.
    \item \textbf{Postcondizioni}: l’utente ha ricevuto un messaggio d’errore.
\end{itemize}

\subsection{UC 14 - Visualizzazione guida}

\begin{figure}[!h]
    \includegraphics[width=10cm]{sezioni/Images/UC14.png.png}
    \centering
    \caption{Visualizzazione guida}
\end{figure}

\begin{itemize}
    \item \textbf{Attore}: l’utente è autenticato.
    \item \textbf{Descrizione}: l’utente deve poter visualizzare la guida in formato mappa o formato lista.
    \item \textbf{Scenario}: l’utente si trova nella home.
    \item \textbf{Precondizioni}: l’utente vuole visualizzare la guida.
    \item \textbf{Postcondizioni}: l’utente visualizza la guida.
\end{itemize}

\begin{figure}[!h]
    \includegraphics[width=10cm]{sezioni/Images/UC14_s.png.png}
    \centering
    \caption{Visualizzazione guida}
\end{figure}

\subsubsection{UC 14.1 - Visualizzazione mappa globale}
\begin{itemize}
    \item \textbf{Attore}: l’utente è autenticato.
    \item \textbf{Descrizione}: l’utente visualizza la mappa globale se non ha nessun profilo social seguito.
    \item \textbf{Scenario}:
    \begin{enumerate}
        \item l’utente non segue nessun profilo social;
        \item l’utente visualizza la mappa globale nella home.
    \end{enumerate}

    \item \textbf{Precondizioni}: %da inserire.
    \item \textbf{Postcondizioni}: %da inserire.
\end{itemize}

\subsubsection{UC 14.2 - Visualizzazione lista globale}
\begin{itemize}
    \item \textbf{Attore}: l’utente è autenticato.
    \item \textbf{Descrizione}: l’utente visualizza la lista globale se non ha nessun profilo social seguito.
    \item \textbf{Scenario}:
    \begin{enumerate}
        \item l’utente non segue nessun profilo social;
        \item l’utente visualizza la lista globale nella home.
    \end{enumerate}

    \item \textbf{Precondizioni}: %da inserire.
    \item \textbf{Postcondizioni}: %da inserire.
\end{itemize}

\subsubsection{UC 14.3 - Visualizzazione mappa personalizzata}
\begin{itemize}
    \item \textbf{Attore}: l’utente è autenticato.
    \item \textbf{Descrizione}: l’utente visualizza la mappa personalizzata se ha dei profili social seguiti.
    \item \textbf{Scenario}:
    \begin{enumerate}
        \item l’utente segue dei profilo social;
        \item l’utente visualizza la mappa personalizzata nella home.
    \end{enumerate}

    \item \textbf{Precondizioni}: %da inserire.
    \item \textbf{Postcondizioni}: %da inserire.
\end{itemize}

\subsubsection{UC 14.4 - Visualizzazione lista personalizzata}
\begin{itemize}
    \item \textbf{Attore}: l’utente è autenticato.
    \item \textbf{Descrizione}: l’utente visualizza la mappa personalizzata se ha dei profili social seguiti.
    \item \textbf{Scenario}:
    \begin{enumerate}
        \item l’utente segue dei profilo social;
        \item l’utente visualizza la lista personalizzata nella home.
    \end{enumerate}

    \item \textbf{Precondizioni}: %da inserire.
    \item \textbf{Postcondizioni}: %da inserire.
\end{itemize}

\subsection{UC 15 - Inserimento profili social da seguire}

\begin{figure}[!h]
    \includegraphics[width=10cm]{sezioni/Images/UC15.png}
    \centering
    \caption{Inserimento profili social da seguire}
\end{figure}

\begin{itemize}
    \item \textbf{Attore}: l’utente è autenticato.
    \item \textbf{Descrizione}: l’utente deve poter aggiungere dei profili social da seguire e da cui sarà generata la guida.
    \item \textbf{Scenario}:
    \begin{enumerate}
        \item l’utente naviga nella sezione dei profili seguiti;
        \item l’utente clicca il pulsante di aggiunta profilo;
        \item l’utente sceglie Instagram o TikTok;
        \item l’utente inserisce l’username da aggiungere;
        \item l’utente conferma e salva.
    \end{enumerate}
    \item \textbf{Estensioni}:\\
        - il profilo inserito non esiste \textbf{(UC 16)};\\
        - il profilo inserito è già stato aggiunto \textbf{(UC 17)};\\

    \item \textbf{Precondizioni}: l’utente vuole seguire un nuovo utente.
    \item \textbf{Postcondizioni}: l’utente ha inserito un nuovo utente da seguire.
\end{itemize}

\subsection{UC 16 - Visualizzazione errore di nome utente non esistente}
\begin{itemize}
    \item \textbf{Attore}: l’utente è autenticato.
    \item \textbf{Descrizione}: durante l’attività di aggiunta di un profilo l’utente deve ricevere un messaggio d’errore se il nome utente ricercato non è presente nel sistema.
    \item \textbf{Scenario}: l’utente legge un messaggio d’errore. 
    \item \textbf{Precondizioni}: l’utente inserisce un nome utente inesistente nella procedura di inserimento dei profli social da seguire.
    \item \textbf{Postcondizioni}: l’utente ha ricevuto un messaggio d’errore.
\end{itemize}

\subsection{UC 17 - Visualizzazione errore di utente già presente nel sistema}
\begin{itemize}
    \item \textbf{Attore}: l’utente è autenticato.
    \item \textbf{Descrizione}: durante l’attività di aggiunta di un profilo l’utente deve ricevere un messaggio d’errore se il nome utente ricercato è già stato inserito nei profili seguiti dall’utente.
    \item \textbf{Scenario}: l’utente legge un messaggio d’errore. 
    \item \textbf{Precondizioni}: l’utente inserisce un nome utente nella procedura di inserimento dei profli social da seguire che è già presente tra quelli seguiti.
    \item \textbf{Postcondizioni}: l’utente ha ricevuto un messaggio d’errore.
\end{itemize}

\subsection{UC 18 - Rimozione profili social seguito}

\begin{figure}[!h]
    \includegraphics[width=10cm]{sezioni/Images/UC18.png}
    \centering
    \caption{Rimozione profili social seguito}
\end{figure}

\begin{itemize}
    \item \textbf{Attore}: l’utente è autenticato.
    \item \textbf{Descrizione}: l’utente deve poter rimuovere un profilo dalla lista di quelli seguiti.
    \item \textbf{Scenario}:
    \begin{enumerate}
        \item l’utente naviga nella sezione dei profili seguiti;
        \item l’utente seleziona il profilo che vuole rimuovere;
        \item l’utente clicca il pulsante di rimozione;
    \end{enumerate}

    \item \textbf{Precondizioni}: l’utente vuole rimuovere un profilo social dalla lista dei seguiti.
    \item \textbf{Postcondizioni}: l’utente ha rimosso il profilo social dalla lista dei seguiti.
\end{itemize}

\subsection{UC 19 - Esplorazione profili social più seguiti}

\begin{figure}[!h]
    \includegraphics[width=10cm]{sezioni/Images/UC19.png}
    \centering
    \caption{Esplorazione profili social più seguiti}
\end{figure}

\begin{itemize}
    \item \textbf{Attore}: l’utente è autenticato.
    \item \textbf{Descrizione}: l’utente può esplorare gli utenti più seguiti dagli altri utenti di questa piattaforma.
    \item \textbf{Scenario}:
    \begin{enumerate}
        \item l’utente si trova nella home;
        \item l’utente preme il pulsante per entrare nella pagina di esplorazione;
        \item l’utente vede la lista degli utenti più seguiti sia di instagram che di tiktok.
    \end{enumerate}

    \item \textbf{Precondizioni}: l’utente vuole visualizzare gli utenti più seguiti.
    \item \textbf{Postcondizioni}: viene visualizzata all’utente la lista degli utenti più seguiti.
\end{itemize}

\subsection{UC 20 - Impostazione vista predefinita guida}

\begin{figure}[!h]
    \includegraphics[width=10cm]{sezioni/Images/UC20.png}
    \centering
    \caption{Impostazione vista predefinita guida}
\end{figure}

\begin{itemize}
    \item \textbf{Attore}: l’utente è autenticato.
    \item \textbf{Descrizione}: l’utente deve poter scegliere la vista predefinita per visualizzare la guida.
    \item \textbf{Scenario}:
    \begin{enumerate}
        \item l’utente naviga nella sezione impostazioni;
        \item l’utente sceglie la voce “Vista predefinita”.
    \end{enumerate}

    \item \textbf{Precondizioni}: %da inserire.
    \item \textbf{Postcondizioni}: %da inserire.
\end{itemize}

\begin{figure}[!h]
    \includegraphics[width=10cm]{sezioni/Images/UC20_s.png}
    \centering
    \caption{Impostazione vista predefinita guida}
\end{figure}

\subsubsection{UC 20.1 - Impostazione mappa come vista predefinita}
\begin{itemize}
    \item \textbf{Attore}: l’utente è autenticato.
    \item \textbf{Descrizione}: l’utente sceglie la mappa come vista predefinita.
    \item \textbf{Scenario}: l’utente sceglie mappa come vista predefinita.
    \item \textbf{Precondizioni}: l’utente vuole scegliere mappa come vista predefinita.
    \item \textbf{Postcondizioni}: mappa viene impostata come vista predefinita.
\end{itemize}

\subsubsection{UC 20.2 - Impostazione lista come vista predefinita}
\begin{itemize}
    \item \textbf{Attore}: l’utente è autenticato.
    \item \textbf{Descrizione}: l’utente sceglie la lista come vista predefinita.
    \item \textbf{Scenario}: l’utente sceglie lista come vista predefinita.
    \item \textbf{Precondizioni}: l’utente vuole scegliere lista come vista predefinita.
    \item \textbf{Postcondizioni}: lista viene impostata come vista predefinita.
\end{itemize}





