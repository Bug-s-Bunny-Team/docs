In questa sezione sono classificati ed assegnati i requisiti seguendo la nomenclatura specificata nelle \NdP\G{}.

\subsection{Requisiti Funzionali}

{
      \setlength{\freewidth}{\dimexpr\textwidth-10\tabcolsep}
      \renewcommand{\arraystretch}{1.5}
      \centering
      \setlength{\aboverulesep}{0pt}
      \setlength{\belowrulesep}{0pt}
      \rowcolors{2}{Arancione!10}{white}
      \begin{longtable}{C{.15\freewidth} C{.2\freewidth} C{.528\freewidth} C{.15\freewidth}}
         \toprule
      \rowcolor{Arancione}
      \textcolor{white}{\textbf{Codice}}&
      \textcolor{white}{\textbf{Classificazione}}&
      \textcolor{white}{\textbf{Descrizione}}&
      \textcolor{white}{\textbf{Fonti}}\\	
      \toprule
      \endhead
      
      RF1 & Obbligatorio & L'utente deve poter registrarsi inserendo i suoi dati personali (email, nome utente e password) & UC1 \\
      RF2 & Obbligatorio & L'utente deve poter effettuare il login inserendo i dati personali richiesti (nome utente e password) & UC5 \\
      RF3 & Obbligatorio & L'utente deve poter effettuare il logout e uscire dalla sessione & UC10 \\
      RF4 & Obbligatorio & All'utente viene mostrato un messaggio d'errore se la mail inserita durante la fase di registrazione è già presente nel sistema  & UC1.1 \\
      RF5 & Obbligatorio & All'utente viene mostrato un messaggio d'errore se il nome utente inserito durante la fase di registrazione è già presente nel sistema & UC1.2 \\
      RF6 & Obbligatorio & All'utente viene mostrato un messaggio d'errore se la password inserita durante la fase di registrazione non presenta le specifiche richieste & UC1.3 \\
      RF7 & Obbligatorio & All'utente viene mostrato un messaggio d'errore se la email inserita durante la fase di autenticazione è errata & UC7\\
      RF8 & Obbligatorio & All'utente viene mostrato un messaggio d'errore se la password inserita durante la fase di autenticazione è errata & UC8 \\
      RF9 & Obbligatorio & L'utente deve poter modificare la propria password inserendo la vecchia password e confermando la nuova & UC11 \\
      RF10 & Obbligatorio & All'utente compare un messaggio d'errore se durante il cambio password, la password attuale inserita non è corretta & UC12 \\
      RF11 & Obbligatorio & All'utente compare un messaggio d'errore se durante il cambio password, la nuova password inserita non rispetta le specifiche & UC13 \\
      RF12 & Obbligatorio & All'utente compare un messaggio d'errore se durante il cambio password, la ripetizione della nuova password inserita non è uguale alla password del campo precedente & UC14 \\
      RF13 & Obbligatorio & L'utente deve poter autenticarsi automaticamente, senza inserire le credenziali & UC8\\
      RF14 & Obbligatorio & L'utente deve poter visualizzare la guida in formato mappa o formato lista & UC15, UC15.1, UC15.2, UC15.3, UC15.4 \\
      RF15 & Obbligatorio & L'utente deve poter visualizzare la mappa globale se non ha nessun profilo social seguito & UC15, UC15.1 \\
      RF16 & Obbligatorio & L'utente deve poter visualizzare la lista personalizzata se segue dei profili social & UC15, UC15.4 \\
      RF17 & Obbligatorio & L'utente deve poter visualizzare la mappa personalizzata se segue dei profili social & UC15, UC15.3 \\
      RF18 & Obbligatorio & L'utente deve poter aggiungere dei profili social da seguire, dai quali verrà generata la guida & UC16 \\
      RF19 & Obbligatorio & L'utente deve poter visualizzare un messaggio di errore nel caso in cui il nome del profilo utente inserito non esista nel sistema & UC17 \\
      RF20 & Obbligatorio & L'utente deve poter visualizzare un messaggio di errore nel caso in cui il nome del profilo utente ricercato si trova già nei profili seguiti & UC18 \\
      RF21 & Obbligatorio & L'utente deve essere in grado di rimuovere (smettere di seguire) un qualsiasi profilo tra i presenti nella lista di quelli che sta seguendo  & UC19 \\
      RF22 & Obbligatorio & L'utente può visualizzare quali sono gli utenti più seguiti dagli altri profili della piattaforma & UC20 \\
      RF23 & Obbligatorio & L'utente deve essere in grado di settare il metodo di visualizzazione predefinita per visionare la guida & UC21 \\
      RF24 & Obbligatorio & L'utente deve essere in grado di settare la mappa come vista predefinita della guida & UC22 \\
      RF25 & Obbligatorio & L'utente deve essere in grado di settare la lista come vista predefinita della guida & UC23 \\	
      RF26 & Obbligatorio & La piattaforma deve fornire la possibilità di monitorare le recensioni partendo da un luogo indicato & Capitolato \\
      RF27 & Obbligatorio & La piattaforma deve permettere di indicare profili social da seguire per creare la guida & Capitolato \\
      RF28 & Obbligatorio & I profili social da seguire devono provenire dalla piattaforma Instagram\G{} & Capitolato \\   
      RF29 & Obbligatorio & Il backend\G{} deve utilizzare le informazioni testuali di post derivati da profili social per trarre informazioni su un luogo fisico & Capitolato \\
      RF30 & Molto Desiderabile & Il backend\G{} dovrebbe utilizzare le informazioni grafiche e visuali di post derivati da profili social per trarre informazioni su un luogo fisico & Capitolato \\
      RF31 & Obbligatorio & Deve essere costruito un crawler/scraper\G{} in grado di accedere o scaricare dati dei post da profili social & Capitolato \\
      RF32 & Obbligatorio & Il crawler\G{} deve eludere i servizi anti-bot dei vari social & Capitolato \\
      \bottomrule
      \rowcolor{white} 
      \caption{Tabella dei requisiti funzionali}
      \end{longtable}
}
\subsection{Requisiti di Vincolo}
{
      \setlength{\freewidth}{\dimexpr\textwidth-10\tabcolsep}
      \renewcommand{\arraystretch}{1.5}
      \centering
      \setlength{\aboverulesep}{0pt}
      \setlength{\belowrulesep}{0pt}
      \rowcolors{2}{Arancione!10}{white}
      \begin{longtable}{C{.15\freewidth} C{.2\freewidth} C{.528\freewidth} C{.15\freewidth}}
         \toprule
      \rowcolor{Arancione}
      \textcolor{white}{\textbf{Codice}}&
      \textcolor{white}{\textbf{Classificazione}}&
      \textcolor{white}{\textbf{Descrizione}}&
      \textcolor{white}{\textbf{Fonti}}\\	
      \toprule
      \endhead
      RV1 & Obbligatorio & Il backend\G{} deve essere implementato secondo un'architettura serverless\G{} & Capitolato \\
      RV2 & Obbligatorio & Il backend\G{} deve utilizzare le funzionalità di machine learning fornite dai servizi AWS\G{} & Capitolato \\
      RV3 & Obbligatorio & La piattaforma deve essere fruibile da web browser. In particolare dovrà essere garantita la compatibilità con le versioni più recenti dei browser \textit{Chrome} e \textit{Firefox}. & Capitolato \\
      \bottomrule
      \rowcolor{white} 
      \caption{Tabella dei requisiti di vincolo}
      \end{longtable}
}
\subsection{Requisiti di Qualità}
{
      \setlength{\freewidth}{\dimexpr\textwidth-10\tabcolsep}
      \renewcommand{\arraystretch}{1.5}
      \centering
      \setlength{\aboverulesep}{0pt}
      \setlength{\belowrulesep}{0pt}
      \rowcolors{2}{Arancione!10}{white}
      \begin{longtable}{C{.15\freewidth} C{.2\freewidth} C{.528\freewidth} C{.15\freewidth}}
         \toprule
      \rowcolor{Arancione}
      \textcolor{white}{\textbf{Codice}}&
      \textcolor{white}{\textbf{Classificazione}}&
      \textcolor{white}{\textbf{Descrizione}}&
      \textcolor{white}{\textbf{Fonti}}\\	
      \toprule
      \endhead
      
      RQ1 & Obbligatorio & Il sistema dovrà essere sviluppato secondo le norme descritte nel documento Norme di Progetto\G{} & Decisione Interna \\
      RQ2 & Obbligatorio & Manuale utente in lingua italiana & Decisione Interna \\
      RQ3 & Obbligatorio & Diagrammi UML\G{} relativi agli Use Cases\G{} di progetto & Capitolato \\
      RQ4 & Obbligatorio & Schema Design\G{} relativo alla base di dati & Capitolato \\
      RQ5 & Obbligatorio & Documentazione dettagliata di tutte le API\G{} & Capitolato \\
      RQ6 & Obbligatorio & Piano Unit Tests\G{} & Capitolato \\
      RQ7 & Obbligatorio & Sintesi sui limiti dei social utilizzati & Capitolato \\
      RQ8 & Obbligatorio & Limiti dei servizi e degli algoritmi usati per estrarre le valutazioni sui luoghi di interesse & Capitolato \\	   
      RQ9 & Obbligatorio & Bug Reporting & Capitolato \\
      RQ10 & Obbligatorio & Codice sorgente fornito tramite sistemi di versionamento & Capitolato \\
      \bottomrule
      \rowcolor{white} 
      \caption{Tabella dei requisiti di qualità}
      \end{longtable}
}

\newpage
\subsection{Requisiti Prestazionali}
{
      \setlength{\freewidth}{\dimexpr\textwidth-10\tabcolsep}
      \renewcommand{\arraystretch}{1.5}
      \centering
      \setlength{\aboverulesep}{0pt}
      \setlength{\belowrulesep}{0pt}
      \rowcolors{2}{Arancione!10}{white}
      \begin{longtable}{C{.15\freewidth} C{.2\freewidth} C{.528\freewidth} C{.15\freewidth}}
         \toprule
      \rowcolor{Arancione}
      \textcolor{white}{\textbf{Codice}}&
      \textcolor{white}{\textbf{Classificazione}}&
      \textcolor{white}{\textbf{Descrizione}}&
      \textcolor{white}{\textbf{Fonti}}\\	
      \toprule
      \endhead
      
      RP1 & Desiderabile & Scegliere i servizi di AWS\G{} che richiedono scambio dati nelle stesse zone geografiche & ve\_20220516 \\
      RP2 & Desiderabile & Il crawler\G/scraper\G{} dovrebbe impiegare un tempo inferiore ad 1 minuto per eseguire le proprie operazioni & Capitolato \\	   
      \bottomrule
      \rowcolor{white} 
      \caption{Tabella dei requisiti prestazionali}
      \end{longtable}
}
\subsection{Tracciamento}
\subsubsection{Fonte - Requisiti}
{
      \setlength{\freewidth}{\dimexpr\textwidth-0\tabcolsep}
      \renewcommand{\arraystretch}{1.5}
      \centering
      \setlength{\aboverulesep}{0pt}
      \setlength{\belowrulesep}{0pt}
      \rowcolors{2}{Arancione!10}{white}
      \begin{longtable}{C{.47\freewidth} C{.47\freewidth}}
         \toprule
      \rowcolor{Arancione}
      \textcolor{white}{\textbf{Fonte}}&
      \textcolor{white}{\textbf{Requisiti}}\\
      \toprule
      \endhead
      
      Capitolato & RV1, RV2, RV3, RF27, RF28, RF29, RF30, RF31, RF32, RF33,
                   RQ3, RQ4, RQ5, RQ6, RQ7, RQ8, RQ9, RQ10,
                   RP2\\
      UC1 & RF1\\
      UC1.1 & RF4\\
      UC1.2 & RF5\\
      UC1.3 & RF6\\
      UC2 & Verbale interno 20/05/2022\\
      UC3 & Verbale interno 20/05/2022\\
      UC4 & Verbale interno 20/05/2022\\
      UC5 & RF2, RF7, RF9\\
      UC6.1 & RF7\\
      UC6.2 & RF9\\
      UC6.3 & Discussione interna\\
      UC7 & RF7 \\
      UC8 & RF8, RF13 \\
      UC10 & RF3 \\
      UC11 & RF9\\
      UC12 & RF10\\
      UC13 & RF11\\
      UC14 & RF12\\
      UC15.1 & RF14, RF15\\
      UC15.2 & RF14\\
      UC15.3 & RF14, RF17\\
      UC15.4 & RF14, RF16\\
      UC16 & RF18\\
      UC17 & RF19\\
      UC18 & RF20\\
      UC19 & RF21\\
      UC20 & RF22\\
      UC21 & RF23\\
      UC22 & RF24\\
      UC23 & RF25\\		
      \bottomrule
      \rowcolor{white} 
      \caption{Tabella fonte - requisiti}
      \end{longtable}
}
\newpage
\subsubsection{Requisito - Fonte}
{
      \setlength{\freewidth}{\dimexpr\textwidth-0\tabcolsep}
      \renewcommand{\arraystretch}{1.5}
      \centering
      \setlength{\aboverulesep}{0pt}
      \setlength{\belowrulesep}{0pt}
      \rowcolors{2}{Arancione!10}{white}
      \begin{longtable}{C{.47\freewidth} C{.47\freewidth}}
         \toprule
      \rowcolor{Arancione}
      \textcolor{white}{\textbf{Requisito}}&
      \textcolor{white}{\textbf{Fonti}}\\
      \toprule
      \endhead
      
      RF1 & UC1\\
      RF2 & UC5\\
      RF3 & UC10\\
      RF4 & UC1.1\\
      RF5 & UC1.2\\
      RF6 & UC1.3\\
      RF7 & UC6.1\\
      RF8 & UC8\\
      RF9 & UC11\\
      RF10 & UC12\\
      RF11 & UC13\\
      RF12 & UC14\\
      RF13 & UC8\\
      RF14 & UC15.1, UC15.2, UC15.3, UC15.4\\
      RF15 & UC15.1\\
      RF16 & UC15.4\\
      RF17 & UC15.3\\
      RF18 & UC16\\
      RF19 & UC17\\
      RF20 & UC18\\
      RF21 & UC19\\
      RF22 & UC20\\
      RF23 & UC21\\
      RF24 & UC22\\
      RF25 & UC23\\
      RF26 & Capitolato\\
      RF27 & Capitolato\\
      RF28 & Capitolato\\
      RF29 & Capitolato\\
      RF30 & Capitolato\\
      RF31 & Capitolato\\
      RF32 & Capitolato\\

      RV1 & Capitolato\\
      RV2 & Capitolato\\
      RV3 & Capitolato\\

      %RQ1\\
      %RQ2\\
      RQ3 & Capitolato\\
      RQ4 & Capitolato\\
      RQ5 & Capitolato\\
      RQ6 & Capitolato\\
      RQ7 & Capitolato\\
      RQ8 & Capitolato\\
      RQ9 & Capitolato\\
      RQ10 & Capitolato\\

      %RP1\\
      RP2 & Capitolato\\
      \bottomrule
      \rowcolor{white} 
      \caption{Tabella requisito - fonte}
      \end{longtable}
}
