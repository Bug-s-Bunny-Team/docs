In questa sezione sono classificati ed assegnati i requisiti seguendo la nomenclatura specificata nelle \NdP.

\subsection{Requisiti Funzionali}

{
      \newcolumntype{L}[1]{>{\raggedright\let\newline\\\arraybackslash\hspace{1pt}}m{#1}}
      \newcolumntype{C}[1]{>{\centering\let\newline\\\arraybackslash\hspace{1pt}}m{#1}}
      \newcolumntype{R}[1]{>{\raggedleft\let\newline\\\arraybackslash\hspace{1pt}}m{#1}}

      %\newlength{\freewidth}
      \setlength{\freewidth}{\dimexpr\textwidth-10\tabcolsep}
      \renewcommand{\arraystretch}{1.5}
      \centering
      \setlength{\aboverulesep}{0pt}
      \setlength{\belowrulesep}{0pt}
      \rowcolors{2}{Arancione!10}{white}
      \begin{longtable}{C{.257\freewidth} C{.257\freewidth} C{.257\freewidth} C{.257\freewidth}}
         \toprule
      \rowcolor{Arancione}
      \textcolor{white}{\textbf{Codice}}&
      \textcolor{white}{\textbf{Classificazione}}&
      \textcolor{white}{\textbf{Descrizione}}&
      \textcolor{white}{\textbf{Fonti}}\\	
      \toprule
      \endhead
      
      RF1 & Obbligatorio & L'utente deve poter registrarsi inserendo i suoi dati personali (email, nome utente e password) & UC 1 \\
      RF2 & Obbligatorio & L'utente deve poter effettuare il login inserendo i dati personali richiesti (nome utente e password) & UC 5 \\
      RF3 & Obbligatorio & L'utente deve poter effettuare il logout e uscire dalla sessione & UC 9 \\
      RF4 & Obbligatorio & All'utente viene mostrato un messaggio d'errore se la mail inserita durante la fase di registrazione è già presente nel sistema  & UC 1.1 \\
      RF5 & Obbligatorio & All'utente viene mostrato un messaggio d'errore se il nome utente inserito durante la fase di registrazione è già presente nel sistema & UC 1.2 \\
      RF6 & Obbligatorio & All'utente viene mostrato un messaggio d'errore se la password inserita durante la fase di registrazione non presenta le specifiche richieste & UC 1.3 \\
      RF7 & Obbligatorio & All'utente viene mostrato un messaggio d'errore se il nome utente inserito durante la fase di autenticazione è errato & UC 5, UC 5.1 \\
      RF8 & Obbligatorio & All'utente viene mostrato un messaggio d'errore se la password inserita durante la fase di autenticazione è errata & UC 5, UC 5.2 \\
      RF9 & Obbligatorio & L'utente deve poter modificare la propria password inserendo la vecchia password e confermando la nuova & UC 10 \\
      RF10 & Obbligatorio & All'utente compare un messaggio d'errore se durante il cambio password, la password attuale inserita non è corretta & UC 11 \\
      RF11 & Obbligatorio & All'utente compare un messaggio d'errore se durante il cambio password, la nuova password inserita non rispetta le specifiche & UC 12 \\
      RF12 & Obbligatorio & All'utente compare un messaggio d'errore se durante il cambio password, la ripetizione della nuova password inserita non è uguale alla password del campo precedente & UC 13 \\
      RF13 & Obbligatorio & L'utente deve poter autenticarsi automaticamente, senza inserire le credenziali & UC 7 \\
      RF14 & Obbligatorio & L'utente deve poter visualizzare la guida in formato mappa o formato lista & UC 14, UC 14.1, UC 14.2, UC 14.3, UC 14.4 \\
      RF15 & Obbligatorio & L'utente deve poter visualizzare la mappa globale se non ha nessun profilo social seguito & UC 14, UC 14.1 \\
      RF16 & Obbligatorio & L'utente deve poter visualizzare la lista personalizzata se segue dei profili social & UC 14, UC 14.4 \\
      RF17 & Obbligatorio & L'utente deve poter visualizzare la mappa personalizzata se segue dei profili social & UC 14, UC 14.3 \\
      RF18 & Obbligatorio & L'utente deve poter aggiungere dei profili social da seguire, dai quali verrà generata la guida & UC 15 \\
      RF19 & Obbligatorio & L'utente deve poter visualizzare un messaggio di errore nel caso in cui il nome del profilo utente inserito non esista nel sistema & UC 16 \\
      RF20 & Obbligatorio & L'utente deve poter visualizzare un messaggio di errore nel caso in cui il nome del profilo utente ricercato si trova già nei profili seguiti & UC 17 \\
      RF21 & Obbligatorio & L'utente deve poter rimuovere un profilo dalla lista di quelli che segue & UC 18 \\
      RF22 & Obbligatorio & L'utente può esplorare gli utenti più seguiti dagli altri utenti di questa piattaforma & UC 19 \\
      RF23 & Obbligatorio & L'utente deve poter scegliere la vista predefinita per visualizzare la guida & UC 20 \\
      RF24 & Obbligatorio & L'utente deve poter scegliere di mettere la mappa come vista predefinita & UC 20, UC 20.1 \\
      RF25 & Obbligatorio & L'utente deve poter scegliere di mettere la mappa come vista predefinita & UC 20, UC 20.2 \\	   
      \bottomrule
      \end{longtable}
}
\subsection{Requisiti di Vincolo}
{
      \newcolumntype{L}[1]{>{\raggedright\let\newline\\\arraybackslash\hspace{1pt}}m{#1}}
      \newcolumntype{C}[1]{>{\centering\let\newline\\\arraybackslash\hspace{1pt}}m{#1}}
      \newcolumntype{R}[1]{>{\raggedleft\let\newline\\\arraybackslash\hspace{1pt}}m{#1}}

      %\newlength{\freewidth}
      \setlength{\freewidth}{\dimexpr\textwidth-10\tabcolsep}
      \renewcommand{\arraystretch}{1.5}
      \centering
      \setlength{\aboverulesep}{0pt}
      \setlength{\belowrulesep}{0pt}
      \rowcolors{2}{Arancione!10}{white}
      \begin{longtable}{C{.257\freewidth} C{.257\freewidth} C{.257\freewidth} C{.257\freewidth}}
         \toprule
      \rowcolor{Arancione}
      \textcolor{white}{\textbf{Codice}}&
      \textcolor{white}{\textbf{Classificazione}}&
      \textcolor{white}{\textbf{Descrizione}}&
      \textcolor{white}{\textbf{Fonti}}\\	
      \toprule
      \endhead
      
      RV1 & Obbligatorio & La piattaforma deve fornire la possibilità di monitorare le recensioni partendo da un luogo indicato & Capitolato \\
      RV2 & Obbligatorio & La piattaforma deve permettere di indicare profili social da seguire per creare la guida & Capitolato \\
      RV3 & Obbligatorio & I profili social da seguire devono provenire dalle piattaforme Instagram e TikTok & Capitolato \\
      RV4 & Obbligatorio & Il backend deve essere implementato secondo un'architettura serverless & Capitolato \\
      RV5 & Obbligatorio & Il backend deve utilizzare le funzionalità di machine learning fornite dai servizi AWS & Capitolato \\
      RV6 & Obbligatorio & La piattaforma deve essere fruibile da web browser & Capitolato \\
      RV7 & Obbligatorio & Il backend deve utilizzare le informazioni testuali di post derivati da profili social per trarre informazioni su un luogo fisico & Capitolato \\
      RV8 & Molto Desiderabile & Il backend dovrebbe utilizzare le informazioni grafiche e visuali di post derivati da profili social per trarre informazioni su un luogo fisico & Capitolato \\
      RV9 & Obbligatorio & Deve essere costruito un crawler/scraper in grado di accedere o scaricare dati dei post da profili social & Capitolato \\
      RV10 & Obbligatorio & Il crawler deve eludere i servizi anti-bot dei vari social & Capitolato \\
      \bottomrule
      \end{longtable}
}
\subsection{Requisiti di Qualità}
{
      \newcolumntype{L}[1]{>{\raggedright\let\newline\\\arraybackslash\hspace{1pt}}m{#1}}
      \newcolumntype{C}[1]{>{\centering\let\newline\\\arraybackslash\hspace{1pt}}m{#1}}
      \newcolumntype{R}[1]{>{\raggedleft\let\newline\\\arraybackslash\hspace{1pt}}m{#1}}

      %\newlength{\freewidth}
      \setlength{\freewidth}{\dimexpr\textwidth-10\tabcolsep}
      \renewcommand{\arraystretch}{1.5}
      \centering
      \setlength{\aboverulesep}{0pt}
      \setlength{\belowrulesep}{0pt}
      \rowcolors{2}{Arancione!10}{white}
      \begin{longtable}{C{.257\freewidth} C{.257\freewidth} C{.257\freewidth} C{.257\freewidth}}
         \toprule
      \rowcolor{Arancione}
      \textcolor{white}{\textbf{Codice}}&
      \textcolor{white}{\textbf{Classificazione}}&
      \textcolor{white}{\textbf{Descrizione}}&
      \textcolor{white}{\textbf{Fonti}}\\	
      \toprule
      \endhead
      
      RQ1 & Obbligatorio & Il sistema dovrà essere sviluppato secondo le norme descritte nel documento Norme di Progetto & Decisione Interna \\
      RQ2 & Obbligatorio & Manuale utente in lingua italiana & Decisione Interna \\
      RQ3 & Obbligatorio & Diagrammi UML relativi agli Use Cases di progetto & Capitolato \\
      RQ4 & Obbligatorio & Schema Design relativo alla base di dati & Capitolato \\
      RQ5 & Obbligatorio & Documentazione dettagliata di tutte le API & Capitolato \\
      RQ6 & Obbligatorio & Piano Unit Tests & Capitolato \\
      RQ7 & Obbligatorio & Sintesi sui limiti dei social utilizzati & Capitolato \\
      RQ8 & Obbligatorio & Limiti dei servizi e degli algoritmi usati per estrarre le valutazioni sui luoghi di interesse & Capitolato \\	   
      RQ9 & Obbligatorio & Bug Reporting & Capitolato \\
      RQ10 & Obbligatorio & Codice sorgente fornito tramite sistemi di versionamento & Capitolato \\
      \bottomrule
      \end{longtable}
}
\subsection{Requisiti Prestazionali}
{
      \newcolumntype{L}[1]{>{\raggedright\let\newline\\\arraybackslash\hspace{1pt}}m{#1}}
      \newcolumntype{C}[1]{>{\centering\let\newline\\\arraybackslash\hspace{1pt}}m{#1}}
      \newcolumntype{R}[1]{>{\raggedleft\let\newline\\\arraybackslash\hspace{1pt}}m{#1}}

      %\newlength{\freewidth}
      \setlength{\freewidth}{\dimexpr\textwidth-10\tabcolsep}
      \renewcommand{\arraystretch}{1.5}
      \centering
      \setlength{\aboverulesep}{0pt}
      \setlength{\belowrulesep}{0pt}
      \rowcolors{2}{Arancione!10}{white}
      \begin{longtable}{C{.257\freewidth} C{.257\freewidth} C{.257\freewidth} C{.257\freewidth}}
         \toprule
      \rowcolor{Arancione}
      \textcolor{white}{\textbf{Codice}}&
      \textcolor{white}{\textbf{Classificazione}}&
      \textcolor{white}{\textbf{Descrizione}}&
      \textcolor{white}{\textbf{Fonti}}\\	
      \toprule
      \endhead
      
      RP1 & Desiderabile & Scegliere i servizi di AWS che richiedono scambio dati nelle stesse zone geografiche & ve\_20220516 \\
      RP2 & Desiderabile & Il crawler/scraper dovrebbe essere efficiente e sfruttare al meglio vari servizi, come l'architettura serverless e lo spazio di archiviazione Amazon S3 & Capitolato \\	   
      \bottomrule
      \end{longtable}
}
\subsection{Tracciamento}
\subsubsection{Fonte - Requisiti}
{
      \newcolumntype{L}[1]{>{\raggedright\let\newline\\\arraybackslash\hspace{1pt}}m{#1}}
      \newcolumntype{C}[1]{>{\centering\let\newline\\\arraybackslash\hspace{1pt}}m{#1}}

      \setlength{\freewidth}{\dimexpr\textwidth-0\tabcolsep}
      \renewcommand{\arraystretch}{1.5}
      \centering
      \setlength{\aboverulesep}{0pt}
      \setlength{\belowrulesep}{0pt}
      \rowcolors{2}{Arancione!10}{white}
      \begin{longtable}{C{.47\freewidth} C{.47\freewidth}}
         \toprule
      \rowcolor{Arancione}
      \textcolor{white}{\textbf{Fonte}}&
      \textcolor{white}{\textbf{Requisiti}}\\
      \toprule
      \endhead
      
      Capitolato & RV1, RV2, RV3, RV4, RV5, RV6, RV7, RV8, RV9, RV10,
                   RQ3, RQ4, RQ5, RQ6, RQ7, RQ8, RQ9, RQ10,
                   RP2\\
      UC1 & RF1\\
      UC1.1 & RF4\\
      UC1.2 & RF5\\
      UC1.3 & RF6\\
      %UC2 & BOH\\
      %UC3 & BOH\\
      %UC4 & BOH\\	
      UC5 & RF2, RF7, RF8\\
      UC5.1 & RF7\\
      UC5.2 & RF8\\
      %UC5.3 & BOH\\
      %UC6 & BOH\\
      UC7 & RF13\\
      %UC8 & BOH\\
      UC9 & RF3\\
      UC10 & RF9\\
      %UC10.1 & BOH\\
      %UC10.2 & BOH\\
      %UC10.3 & BOH\\
      %UC10.4 & BOH\\
      UC11 & RF10\\
      UC12 & RF11\\
      UC13 & RF12\\
      UC14 & RF14, RF15, RF16, RF17\\
      UC14.1 & RF14, RF15\\
      UC14.2 & RF14, RF15\\
      UC14.3 & RF14, RF17\\
      UC14.4 & RF14, RF16\\
      UC15 & RF18\\
      UC16 & RF19\\
      UC17 & RF20\\
      UC18 & RF21\\
      UC19 & RF22\\
      UC20 & RF23, RF24, RF25\\
      UC20.1 & RF24\\
      UC20.2 & RF25\\		
      \bottomrule
      \end{longtable}
}
\newpage
\subsubsection{Requisito - Fonte}
{
      \newcolumntype{L}[1]{>{\raggedright\let\newline\\\arraybackslash\hspace{1pt}}m{#1}}
      \newcolumntype{C}[1]{>{\centering\let\newline\\\arraybackslash\hspace{1pt}}m{#1}}

      \setlength{\freewidth}{\dimexpr\textwidth-0\tabcolsep}
      \renewcommand{\arraystretch}{1.5}
      \centering
      \setlength{\aboverulesep}{0pt}
      \setlength{\belowrulesep}{0pt}
      \rowcolors{2}{Arancione!10}{white}
      \begin{longtable}{C{.47\freewidth} C{.47\freewidth}}
         \toprule
      \rowcolor{Arancione}
      \textcolor{white}{\textbf{Fonte}}&
      \textcolor{white}{\textbf{Requisiti}}\\
      \toprule
      \endhead
      
      RF1 & UC1\\
      RF2 & UC5\\
      RF3 & UC9\\
      RF4 & UC1.1\\
      RF5 & UC1.2\\
      RF6 & UC1.3\\
      RF7 & UC5, UC5.1\\
      RF8 & UC5, UC5.2\\
      RF9 & UC10\\
      RF10 & UC11\\
      RF11 & UC12\\
      RF12 & UC13\\
      RF13 & UC7\\
      RF14 & UC14, UC14.1, UC14.2, UC14.3, UC14.4\\
      RF15 & UC14, UC14.1, UC14.2\\
      RF16 & UC14, UC14.4\\
      RF17 & UC14, UC14.3\\
      RF18 & UC15\\
      RF19 & UC16\\
      RF20 & UC17\\
      RF21 & UC18\\
      RF22 & UC19\\
      RF23 & UC20\\
      RF24 & UC20, UC20.1\\
      RF25 & UC20, UC20.2\\

      %RV1
      %RV2
      %RV3
      %RV4
      %RV5
      %RV6
     % RV7
     % RV8
     % RV9
     % RV10

    %  RQ1
     % RQ2
    %  RQ3
   %   RQ4
   %   RQ5
   %%   RQ6
   %   RQ7
   %   RQ8
   %   RQ9
    %  RQ10

    %  RP1
   %   RP2
      \bottomrule
      \end{longtable}
}
