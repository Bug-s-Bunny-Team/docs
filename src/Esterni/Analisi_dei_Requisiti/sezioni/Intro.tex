\subsection{Scopo}
Il documento descrive in maniera dettagliata i casi d'uso\G{} e i requisiti del progetto, ottenuti dall'analisi del capitolato del proponente \proponente{} e dal dialogo con esso tramite Slack\G{} e incontri telematici su Google Meet\G{}.

\subsection{Scopo del progetto}
Le informazioni condivise sui social network\G{} sono in costante crescita, tuttavia non vengono rese facilmente sfruttabili dalle aziende che le possiedono. L'obiettivo di questo progetto è prelevare da due social (Instagram\G{} e TikTok\G{}) contenuto multimediale (video/immagini) e tradurlo, sfruttando i servizi di Amazon AWS\G{}, in informazione testo da usare per creare una guida di ristoranti e locali (in stile Guida Michelin\G{}) accessibile tramite browser. 

\subsection{Glossario}
Per maggiore chiarezza del lessico usato, è stato creato un glossario, il quale 
contiene spiegazioni dei termini più importanti che sono stati usati.

\subsection{Riferimenti}
\subsubsection{Riferimenti normativi}
\begin{itemize}
	\item
	{\textbf{Capitolato d'appalto C4:}}\\\url{https://www.math.unipd.it/~tullio/IS-1/2021/Progetto/C4.pdf}
    \item
    {\textbf{Verbale Esterno 16 Marzo:}}\\\url{https://github.com/Bug-s-Bunny-Team/docs/blob/dev/src/Esterni/Verbali/ve_20220316.tex}
    \item
	{\textbf{Verbale Esterno 21 Aprile:}}\\\url{https://github.com/Bug-s-Bunny-Team/docs/blob/dev/src/Esterni/Verbali/ve_20220421.tex}
    \item
	{\textbf{Verbale Esterno 16 Maggio:}}\\\url{https://github.com/Bug-s-Bunny-Team/docs/blob/dev/src/Esterni/Verbali/ve_20220516.tex}

\end{itemize}
\subsubsection{Riferimenti informativi}
\begin{itemize}
	\item 
    {\textbf{Slide dell'insegnamento di Ingegneria del Software:}}\\\url{https://elearning.unipd.it/math/course/view.php?id=793}
	\item
	{\textbf{Slide Analisi dei Requisiti:}}\\\url{https://www.math.unipd.it/~tullio/IS-1/2021/Dispense/T07.pdf}
	\item
	{\textbf{Slide Diagrammi Casi D'Uso:}}\\\url{https://www.math.unipd.it/~rcardin/swea/2022/Diagrammi%20Use%20Case.pdf}
\end{itemize}
