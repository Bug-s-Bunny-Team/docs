\subsection{Introduzione}
Per garantire la qualità dei processi sono stati usati come riferimento gli standards 
ISO/IEC 12207:1995\G, ISO/IEC 15504\G noto anche come SPICE e ISO/IEC 25000:2005\G.
Il gruppo ha scelto un sottoinsieme dei processi elencati dai modelli e li ha adattati alle 
necessità del progetto.

\subsection{Qualità di processo}
{

    \setlength{\freewidth}{\dimexpr\textwidth-10\tabcolsep}
    \renewcommand{\arraystretch}{1.5}
    \centering
    \setlength{\aboverulesep}{0pt}
    \setlength{\belowrulesep}{0pt}
    \rowcolors{2}{Arancione!10}{white}
    \begin{longtable}{C{.257\freewidth} C{.257\freewidth} C{.257\freewidth} C{.257\freewidth}}
       \toprule
    \rowcolor{Arancione}
    \textcolor{white}{\textbf{Metrica}}&
    \textcolor{white}{\textbf{Nome}}&
    \textcolor{white}{\textbf{Valore accettabile}}&
    \textcolor{white}{\textbf{Valore preferibile}}\\	
    \toprule
    \endhead
    
    
    MPP1 & Schedule variance & $\geq$ -20\% & 0\% \\
    MPP2 & Budget variance & $\geq$ -10\% & 0\% \\
    MPP3 & Budgeted Cost of Work Performed & $\geq$ 0 & $\geq$ 0 \\
    MPP4 & Budgeted Cost of Work Sceduled & $\geq$ 0 & $\geq$ 0 \\
    MPP5 & SPICE Capability & $\geq$ 2 & $\geq$ 4 \\
    MPP6 & Requirements Stabilty Index & 70\% & 100\% \\
    MPP7 & Non-calculated Risk & $\leq$ 5 & 0 \\	   
    \bottomrule
    \caption{Tabella riguardo la qualità di processo}
    \end{longtable}
}

\subsection{Qualità di prodotto}
{

    \setlength{\freewidth}{\dimexpr\textwidth-10\tabcolsep}
    \renewcommand{\arraystretch}{1.5}
    \centering
    \setlength{\aboverulesep}{0pt}
    \setlength{\belowrulesep}{0pt}
    \rowcolors{2}{Arancione!10}{white}
    \begin{longtable}{C{.257\freewidth} C{.257\freewidth} C{.257\freewidth} C{.257\freewidth}}
       \toprule
    \rowcolor{Arancione}
    \textcolor{white}{\textbf{Metrica}}&
    \textcolor{white}{\textbf{Nome}}&
    \textcolor{white}{\textbf{Valore accettabile}}&
    \textcolor{white}{\textbf{Valore preferibile}}\\	
    \toprule
    \endhead
    
    
    MPR1 & Indice di Gulpease & $\geq$ 40 & $\geq$ 80 \\
    MPR2 & Percentuale Requisiti Obbligatori Soddisfatti & 100\% & 100\% \\
    MPR3 & Code Coverage & $\geq$ 70\% & 100\% \\
    MPR4 & Branch Coverage & $\geq$ 70\% & 100\% \\
    MPR5 & Accoppiamento tra Classi & $\leq$ 4 & $\leq$ 2 \\
    MPR6 & Profondità Gerarchie & $\leq$ 5 & $\leq$ 3 \\
    MPR7 & Numero Attributi per Classe & $\leq$ 10 & $\leq$ 6 \\
    MPR8 & Numero Parametri per Metodo & $\leq$ 8 & $\leq$ 4 \\
    MPR9 & Linee Codice per Metodo & $\leq$ 50 & $\leq$ 20 \\
    MPR10 & Linee Commento per Codice & $\geq$ 0.2 & $\geq$ 0.4 \\
    MPR11 & Densità Errori & $\leq$ 20\% & 0\% \\
    MPR12 & Facilità Utilizzo  & $\leq$ 3 & 1 \\
    MPR13 & Errori Ortografici & 5\% & 0\% \\
    MPR14 & Complessità Ciclomatica Media & $\leq$ 20 & $\leq$ 10 \\
    MPR15 & Tempo medio risposta WebApp & $\leq$ 3 s & $\leq$ 1 s \\
    MPR16 & Percentuale test passati & $\geq$ 80\% & 100\% \\
    MPR17 & Percentuale test falliti & $\leq$ 10\% & 0\% \\
    MPR18 & Metriche di qualità soddisfatte & $\geq$ 80\% & 100\% \\	   
    \bottomrule
    \rowcolor{white}
    \caption{Tabella riguardo la qualità di prodotto}
    \end{longtable}
}
