\subsection{Valutazioni per il miglioramento}
In questa sezione si analizza la dinamica di automiglioramento messa in atto al gruppi di seguito ai problemi riscontrati nel cprso del progetto ,per perseguire l’obiettivo di miglioramento continuo fissato nelle Norme di Progetto.Viene quindi stilato un registro temporale dei problemi incontrati ,le soluzioni a essi a il miglioramento che il gruppo ne ha tratto.
Riprendendo la categorizzazione  dei rischi nel Piano di Progetto ,si possono suddividere i problemi raccolti in:

Problemi organizzativi :rappresentano la classe di problemi specifici alla sfera gestionale del progetto e alla suddivisione lavorativa.

Problemi tecnologici:rappresentano la classe di problemi specifici all’utilizzo degli strumenti coinvolti nel progetto.

\subsection{Valutazione Organizzativa}
{
    \newcolumntype{L}[1]{>{\raggedright\let\newline\\\arraybackslash\hspace{1pt}}m{#1}}
    \newcolumntype{C}[1]{>{\centering\let\newline\\\arraybackslash\hspace{1pt}}m{#1}}
    \newcolumntype{R}[1]{>{\raggedleft\let\newline\\\arraybackslash\hspace{1pt}}m{#1}}
    %\newlength{\freewidth}
    \setlength{\freewidth}{\dimexpr\textwidth-10\tabcolsep}
    \renewcommand{\arraystretch}{1.5}
    \centering
    \setlength{\aboverulesep}{0pt}
    \setlength{\belowrulesep}{0pt}
    \rowcolors{2}{Arancione!10}{white}
    \begin{longtable}{C{.115\freewidth} C{.257\freewidth} C{.26\freewidth} C{.4\freewidth}}
       \toprule
    \rowcolor{Arancione}
    \textcolor{white}{\textbf{Incremento}}&
    \textcolor{white}{\textbf{Problema emerso}}&
    \textcolor{white}{\textbf{Rischio associato}}&
    \textcolor{white}{\textbf{Reazione miglorativa}}\\	
    \toprule
    \endhead
    
   Primo & In concomitanza alle festività pasquali,alcuni membri hanno avuto dei brevi impegni personali & ROR2 &I membri hanno recuperato il lavoro perso nei giorni successivi,rientrando così nei tempi previsti dalla pianificazione & \\

   secondo & A causa degli esami universitari,alcuni membri hanno avuto un rallentamento nel lavoro svolto & ROR2 & E stato applicato il piano di contingenza
   previsto \\
}
\subsection{Scopo del tecnologica}

{
    \newcolumntype{L}[1]{>{\raggedright\let\newline\\\arraybackslash\hspace{1pt}}m{#1}}
    \newcolumntype{C}[1]{>{\centering\let\newline\\\arraybackslash\hspace{1pt}}m{#1}}
    \newcolumntype{R}[1]{>{\raggedleft\let\newline\\\arraybackslash\hspace{1pt}}m{#1}}
    %\newlength{\freewidth}
    \setlength{\freewidth}{\dimexpr\textwidth-10\tabcolsep}
    \renewcommand{\arraystretch}{1.5}
    \centering
    \setlength{\aboverulesep}{0pt}
    \setlength{\belowrulesep}{0pt}
    \rowcolors{2}{Arancione!10}{white}
    \begin{longtable}{C{.115\freewidth} C{.257\freewidth} C{.26\freewidth} C{.4\freewidth}}
       \toprule
    \rowcolor{Arancione}
    \textcolor{white}{\textbf{Incremento}}&
    \textcolor{white}{\textbf{Problema emerso}}&
    \textcolor{white}{\textbf{Rischio associato}}&
    \textcolor{white}{\textbf{Reazione miglorativa}}\\	
    \toprule
    \endhead
    
Primo & Durante le prime fasi del progetto,alcuni membri presentavano inesperienza all’uso dello strumento LATEX & RT1 & E stato applicato il piano di contingenza
previsto \\
secondo & Durante le prime fasi del progetto,alcuni membri presentavano inesperienza all’uso degli strumenti Git e GitHub & RT1 & E stato applicato il piano di contingenza
previsto\\





    
}
