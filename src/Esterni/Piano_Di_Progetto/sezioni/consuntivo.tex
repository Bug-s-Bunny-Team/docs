Per ogni attività di ogni componente verranno tracciate le ore di lavoro per ruolo, registrando le effettive spese. Questo risulta utile per ottenere un bilancio, confrontando le aspettative iniziali confrontando i risultati ottenuti. Il bilancio potrà risultare:
\begin{itemize}
    \item \textbf{Positivo}: ovvero le spese del preventivo superano quelle del consuntivo;
    \item \textbf{Negativo}: ovvero le spese del consuntivo superano quelle del preventivo (caso opposto al precedente);
    \item \textbf{In pari}: ovvero le spese del preventivo coincidono con quelle del consuntivo.
\end{itemize}

\subsection{Requirements and Technology Baseline} 
 { 
     Si è deciso di fare una tabella per analizzare meglio i consuntivi dei tre incrementi del periodo \textit{Requirements and Technology Baseline}.

\subsubsection{Primo incremento - Costo} 
{
      \setlength{\freewidth}{\dimexpr\textwidth-30\tabcolsep}
      \renewcommand{\arraystretch}{1.0}
      \centering
      \setlength{\aboverulesep}{0pt}
      \setlength{\belowrulesep}{0pt}
      \rowcolors{2}{Arancione!10}{white}
      \begin{longtable}{C{.4\freewidth} C{.2\freewidth} C{.2\freewidth}}
      \toprule
      \rowcolor{Arancione}
      \textcolor{white}{\textbf{Ruolo}}&
      \textcolor{white}{\textbf{Ore}}&
      \textcolor{white}{\textbf{Costo}}\\
      \toprule
      \endhead

      Responsabile & 6 & \euro180 \\
      Amministratore & 29 & \euro580 \\
      Analista & - & - \\
      Progettista & - & - \\
      Programmatore & - & - \\
      Verificatore & 24 & \euro360 \\
      Totale & 59 & \euro1120 \\
      Preventivo & 56 & \euro1050 \\
      Differenza & +3 & +\euro70 \\
      \bottomrule
      \\
      \rowcolor{white}
      \caption{Primo incremento - Consuntivo costo}
      \end{longtable} 

      
      \textbf{Considerazioni:} 
        Riguardo tale incremento, il gruppo trae le seguenti considerazioni:
        \begin{itemize}
            \item Il gruppo ha avuto bisogno di un impiego maggiore per il ruolo di responsabile in quanto, formato da poco, necessitava di una migliore gestione delle attività per migliorare i risultati previsti;
            \item Sono servite ulteriori ore come amministratore per la stesura dei documenti.
        \end{itemize}

        \textbf{Conclusioni:} il gruppo ha avuto un bilancio negativo per questo incremento.
    }

    \newpage
    \subsubsection{Secondo incremento - Costo}
    {
      \setlength{\freewidth}{\dimexpr\textwidth-30\tabcolsep}
      \renewcommand{\arraystretch}{1.0}
      \centering
      \setlength{\aboverulesep}{0pt}
      \setlength{\belowrulesep}{0pt}
      \rowcolors{2}{Arancione!10}{white}
      \begin{longtable}{C{.4\freewidth} C{.2\freewidth} C{.2\freewidth}}
      \toprule
      \rowcolor{Arancione}
      \textcolor{white}{\textbf{Ruolo}}&
      \textcolor{white}{\textbf{Ore}}&
      \textcolor{white}{\textbf{Costo}}\\
      \toprule
      \endhead

      Responsabile & 5 & \euro150 \\
      Amministratore & - & - \\
      Analista & 49 & \euro1225 \\
      Progettista & - & - \\
      Programmatore & - & - \\
      Verificatore & 22 & \euro345 \\
      Totale & 76 & \euro1705 \\
      Preventivo & 77 & \euro1720 \\
      Differenza & -1 & -\euro15 \\
      \bottomrule
      \\
      \rowcolor{white}
      \caption{Secondo incremento - Consuntivo costo}

      \end{longtable} 
    
      \textbf{Considerazioni:} 
        Riguardo tale incremento, il gruppo trae le seguenti considerazioni:
        \begin{itemize}
            \item Sono servite meno ore di quelle richieste nel ruolo di verificatore;
            \item Le ore degli altri ruoli sono rimaste invariate.
        \end{itemize}

        \textbf{Conclusioni:} il gruppo ha avuto un bilancio positivo per questo incremento. 
    }

    \subsubsection{Terzo incremento - Costo}
    {
      \setlength{\freewidth}{\dimexpr\textwidth-30\tabcolsep}
      \renewcommand{\arraystretch}{1.0}
      \centering
      \setlength{\aboverulesep}{0pt}
      \setlength{\belowrulesep}{0pt}
      \rowcolors{2}{Arancione!10}{white}
      \begin{longtable}{C{.4\freewidth} C{.2\freewidth} C{.2\freewidth}}
      \toprule
      \rowcolor{Arancione}
      \textcolor{white}{\textbf{Ruolo}}&
      \textcolor{white}{\textbf{Ore}}&
      \textcolor{white}{\textbf{Costo}}\\
      \toprule
      \endhead

      Responsabile & 5 & \euro150 \\
      Amministratore & 6 & \euro120 \\
      Analista & 9 & \euro225 \\
      Progettista & 26 & \euro650 \\
      Programmatore & 38 & \euro570 \\
      Verificatore & 12 & \euro180 \\
      Totale & 96 & \euro1895 \\
      Preventivo & 99 & \euro1935 \\
      Differenza & -3 & -\euro40  \\
      \bottomrule
      \\
      \rowcolor{white}
      \caption{Terzo incremento - Consuntivo costo}

      \end{longtable} 
      
      \textbf{Considerazioni:} 
        Rispetto tale incremento, il gruppo trae le seguenti considerazioni:
        \begin{itemize}
            \item Sono servite meno ore del previsto per il ruolo di programmatore, mentre il contrario per l'amministratore;
            \item Per gli altri ruoli le ore previste sono state rispettate.
        \end{itemize}

        \textbf{Conclusioni:} il gruppo ha avuto un bilancio positivo per questo incremento. 
    }

    \newpage    
    \subsubsection{Consuntivo del periodo} {
        \setlength{\freewidth}{\dimexpr\textwidth-30\tabcolsep}
        \renewcommand{\arraystretch}{1.0}
        \setlength{\aboverulesep}{0pt}
        \setlength{\belowrulesep}{0pt}
        \begin{longtable}{C{.4\freewidth} C{.4\freewidth} C{.4\freewidth}}
        \toprule
        \rowcolor{Arancione}
        \textcolor{white}{\textbf{Data inizio}}&
        \textcolor{white}{\textbf{Data previsione revisione}}&
        \textcolor{white}{\textbf{Data richiesta revisione}} \\
        \toprule
        \endhead
            
        2022-04-07 & 2022-06-13 & 2022-07-08 \\
        \bottomrule
        \\
        \caption{RTB - Consuntivo periodo}
        \end{longtable}
        Il gruppo trae le seguenti considerazioni:
        \begin{itemize}
            \item La sessione estiva ha comportato un ritardo per la revisione RTB\G. I membri del gruppo,
                  per prepararsi adeguatamente all'esame di Ingegneria del Software, 
                  haNNO preferito prendersi una pausa di una settimana per dedicare maggior tempo allo studio;
            \item Il secondo colloquio con l'azienda proponente, per conoscere meglio le tecnologie che si sarebbero 
                  utilizzate, si pensava di fissarlo ai primi di Maggio, ma l'azienda ha dato disponibilità 
                  il 2022-05-16;
            \item Sono sorti dei problemi da parte dell'azienda con AWS\G{} per il rilascio dei crediti e il gruppo ha dovuto aspettare 
                  più tempo del previsto per ottenerli (circa 2 settimane);
            \item La revisione è stata richiesta con un ritardo di 26 giorni: il bilancio, in questo caso, è negativo.
        \end{itemize}
        In generale, il gruppo si impegnerà a comunicare con più anticipo eventuali richieste all'azienda, organizzandosi 
        in maniera più efficiente. Inoltre per non creare ulteriori ritardi, si concentrerà a recuperare, per soddisfare i tempi previsti. 
    }
 }
