Per ogni attività di ogni componente verranno tracciate le ore di lavoro per ruolo, registrando le effettive spese. Questo risulta utile per ottenere un bilancio, confrontando le aspettative iniziali confrontando i risultati ottenuti. Il bilancio potrà risultare:
\begin{itemize}
    \item \textbf{Positivo}: ovvero le spese del preventivo superano quelle del consuntivo;
    \item \textbf{Negativo}: ovvero le spese del consuntivo superano quelle del preventivo (caso opposto al precedente);
    \item \textbf{In pari}: ovvero le spese del preventivo coincidono con quelle del consuntivo.
\end{itemize}

\subsection{Requirements and Technology Baseline} 
 { 
     Si è deciso di fare una tabella per analizzare meglio i consuntivi dei tre incrementi del periodo \textit{Requirements and Technology Baseline}.

\subsubsection{Primo incremento - Costo} 
{
      \setlength{\freewidth}{\dimexpr\textwidth-30\tabcolsep}
      \renewcommand{\arraystretch}{1.0}
      \centering
      \setlength{\aboverulesep}{0pt}
      \setlength{\belowrulesep}{0pt}
      \rowcolors{2}{Arancione!10}{white}
      \begin{longtable}{C{.4\freewidth} C{.2\freewidth} C{.2\freewidth}}
      \toprule
      \rowcolor{Arancione}
      \textcolor{white}{\textbf{Ruolo}}&
      \textcolor{white}{\textbf{Ore}}&
      \textcolor{white}{\textbf{Costo}}\\
      \toprule
      \endhead

      Responsabile & 6 & \euro180 \\
      Amministratore & 29 & \euro580 \\
      Analista & - & - \\
      Progettista & - & - \\
      Programmatore & - & - \\
      Verificatore & 24 & \euro360 \\
      Totale & 59 & \euro1120 \\
      Preventivo & 56 & \euro1050 \\
      Differenza & +3 & +\euro70 \\
      \bottomrule
      \\
      \rowcolor{white}
      \caption{Primo incremento - Consuntivo costo}
      \end{longtable} 

      
      \textbf{Considerazioni:} 
        Riguardo tale incremento, il gruppo trae le seguenti considerazioni:
        \begin{itemize}
            \item Il gruppo ha avuto bisogno di un impiego maggiore per il ruolo di responsabile in quanto, formato da poco, necessitava di una migliore gestione delle attività per migliorare i risultati previsti;
            \item Sono servite ulteriori ore come amministratore per la stesura dei documenti.
        \end{itemize}

        \textbf{Conclusioni:} il gruppo ha avuto un bilancio negativo per questo incremento.
    }

    \newpage
    \subsubsection{Secondo incremento - Costo}
    {
      \setlength{\freewidth}{\dimexpr\textwidth-30\tabcolsep}
      \renewcommand{\arraystretch}{1.0}
      \centering
      \setlength{\aboverulesep}{0pt}
      \setlength{\belowrulesep}{0pt}
      \rowcolors{2}{Arancione!10}{white}
      \begin{longtable}{C{.4\freewidth} C{.2\freewidth} C{.2\freewidth}}
      \toprule
      \rowcolor{Arancione}
      \textcolor{white}{\textbf{Ruolo}}&
      \textcolor{white}{\textbf{Ore}}&
      \textcolor{white}{\textbf{Costo}}\\
      \toprule
      \endhead

      Responsabile & 5 & \euro150 \\
      Amministratore & - & - \\
      Analista & 49 & \euro1225 \\
      Progettista & - & - \\
      Programmatore & - & - \\
      Verificatore & 22 & \euro345 \\
      Totale & 76 & \euro1705 \\
      Preventivo & 77 & \euro1720 \\
      Differenza & -1 & -\euro15 \\
      \bottomrule
      \\
      \rowcolor{white}
      \caption{Secondo incremento - Consuntivo costo}

      \end{longtable} 
    
      \textbf{Considerazioni:} 
        Riguardo tale incremento, il gruppo trae le seguenti considerazioni:
        \begin{itemize}
            \item Sono servite meno ore di quelle richieste nel ruolo di verificatore;
            \item Le ore degli altri ruoli sono rimaste invariate.
        \end{itemize}

        \textbf{Conclusioni:} il gruppo ha avuto un bilancio positivo per questo incremento. 
    }

    \subsubsection{Terzo incremento - Costo}
    {
      \setlength{\freewidth}{\dimexpr\textwidth-30\tabcolsep}
      \renewcommand{\arraystretch}{1.0}
      \centering
      \setlength{\aboverulesep}{0pt}
      \setlength{\belowrulesep}{0pt}
      \rowcolors{2}{Arancione!10}{white}
      \begin{longtable}{C{.4\freewidth} C{.2\freewidth} C{.2\freewidth}}
      \toprule
      \rowcolor{Arancione}
      \textcolor{white}{\textbf{Ruolo}}&
      \textcolor{white}{\textbf{Ore}}&
      \textcolor{white}{\textbf{Costo}}\\
      \toprule
      \endhead

      Responsabile & 5 & \euro150 \\
      Amministratore & 6 & \euro120 \\
      Analista & 9 & \euro225 \\
      Progettista & 26 & \euro650 \\
      Programmatore & 38 & \euro570 \\
      Verificatore & 12 & \euro180 \\
      Totale & 96 & \euro1895 \\
      Preventivo & 99 & \euro1935 \\
      Differenza & -3 & -\euro40  \\
      \bottomrule
      \\
      \rowcolor{white}
      \caption{Terzo incremento - Consuntivo costo}

      \end{longtable} 
      
      \textbf{Considerazioni:} 
        Rispetto tale incremento, il gruppo trae le seguenti considerazioni:
        \begin{itemize}
            \item Sono servite meno ore del previsto per il ruolo di programmatore, mentre il contrario per l'amministratore;
            \item Per gli altri ruoli le ore previste sono state rispettate.
        \end{itemize}

        \textbf{Conclusioni:} il gruppo ha avuto un bilancio positivo per questo incremento. 
    }

    \newpage    
    \subsubsection{Consuntivo del periodo} {
        \setlength{\freewidth}{\dimexpr\textwidth-30\tabcolsep}
        \renewcommand{\arraystretch}{1.0}
        \setlength{\aboverulesep}{0pt}
        \setlength{\belowrulesep}{0pt}
        \begin{longtable}{C{.4\freewidth} C{.4\freewidth} C{.4\freewidth}}
        \toprule
        \rowcolor{Arancione}
        \textcolor{white}{\textbf{Data inizio}}&
        \textcolor{white}{\textbf{Data previsione revisione}}&
        \textcolor{white}{\textbf{Data richiesta revisione}} \\
        \toprule
        \endhead
            
        2022-04-07 & 2022-06-13 & 2022-07-13 \\
        \bottomrule
        \\
        \caption{RTB - Consuntivo periodo}
        \end{longtable}
        Il gruppo trae le seguenti considerazioni:
        \begin{itemize}
            \item La sessione estiva ha comportato un ritardo per la revisione RTB\G. I membri del gruppo,
                  per prepararsi adeguatamente all'esame di Ingegneria del Software, 
                  hanno preferito prendersi una pausa di una settimana per dedicare maggior tempo allo studio;
            \item Il secondo colloquio con l'azienda proponente, per conoscere meglio le tecnologie che si sarebbero 
                  utilizzate, si pensava di fissarlo ai primi di Maggio, ma l'azienda ha dato disponibilità 
                  il 2022-05-16;
            \item Sono sorti dei problemi da parte dell'azienda con AWS\G{} per il rilascio dei crediti e il gruppo ha dovuto aspettare 
                  più tempo del previsto per ottenerli (circa 2 settimane);
            \item La revisione è stata richiesta con un mese di ritardo: il bilancio è negativo.
        \end{itemize}
        In generale, il gruppo si impegnerà a comunicare con più anticipo eventuali richieste all'azienda, organizzandosi 
        in maniera più efficiente. Inoltre per non creare ulteriori ritardi, si concentrerà a recuperare, per soddisfare i tempi previsti. 
    }
 }

    \subsection{Product Baseline} {
        Si è deciso di riportare una tabella con le ore svolte da ciascun compenente per analizzare meglio 
        i consuntivi dei due incrementi del periodo \textit{Product Baseline}.
        \subsubsection{Quarto incremento - Orario} {
            \setlength{\freewidth}{\dimexpr\textwidth-30\tabcolsep}
            \renewcommand{\arraystretch}{1.0}
            \setlength{\aboverulesep}{0pt}
            \setlength{\belowrulesep}{0pt}
            \rowcolors{2}{Arancione!10}{white}
            \begin{longtable}{C{.3\freewidth} C{.1\freewidth} C{.1\freewidth} C{.1\freewidth} C{.1\freewidth} C{.1\freewidth} C{.1\freewidth} C{.1\freewidth} C{.1\freewidth} C{.3\freewidth}}
            \toprule
            \rowcolor{Arancione}
            \textcolor{white}{\textbf{Componente}}&
            \textcolor{white}{\textbf{Re}}&
            \textcolor{white}{\textbf{Am}}&
            \textcolor{white}{\textbf{An}}&
            \textcolor{white}{\textbf{Pt}}&
            \textcolor{white}{\textbf{Pr}}&
            \textcolor{white}{\textbf{Ve}}&
            \textcolor{white}{\textbf{Ore}}&
            \textcolor{white}{\textbf{Ore previste}} \\
            \toprule
            \endhead
    
            Arena Angela & - & 2 & 5 & 11 & - & 2 & 20 & 14\\      
            Bellò Marco & - & 2 & - & 15 & - & 1 & 18 & 13\\      
            Bousapnamene & - & - & 10 & - & - & - & 10 & 13\\      
            Di Fant Tommaso & 2 & - & - & 17 & - & 3 & 22 & 13\\      
            Tossuto Matteo & 3 & 4 & - & 11 & - & - & 18 & 15\\      
            Volpato Marco & - & 5 & - & 16 & - & - & 21 & 16 \\      
            Zanatta Giulio & - & 2 & 1 & 13 & - & - & 16 & 15 \\      
            Totali & 5 & 15 & 16 & 83 & - & 6 & 125 & 99 \\
            \bottomrule
            \rowcolor{white}
            \\
            \caption{\centering{Quarto incremento - consuntivo orario}}
    
            \end{longtable} 
            

        }
        \subsubsection{Quarto incremento - Costo} {
            \setlength{\freewidth}{\dimexpr\textwidth-30\tabcolsep}
      \renewcommand{\arraystretch}{1.0}
      \centering
      \setlength{\aboverulesep}{0pt}
      \setlength{\belowrulesep}{0pt}
      \rowcolors{2}{Arancione!10}{white}
      \begin{longtable}{C{.4\freewidth} C{.2\freewidth} C{.2\freewidth}}
      \toprule
      \rowcolor{Arancione}
      \textcolor{white}{\textbf{Ruolo}}&
      \textcolor{white}{\textbf{Ore}}&
      \textcolor{white}{\textbf{Costo}}\\
      \toprule
      \endhead

      Responsabile & 5 & \euro150 \\
      Amministratore & 15 & \euro300 \\
      Analista & 16 & \euro400 \\
      Progettista & 83 & \euro2075 \\
      Programmatore & - & - \\
      Verificatore & 6 & \euro90 \\
      Totale & 125 & \euro3015 \\
      Preventivo & 101 & \euro2260 \\
      Differenza & 24 & +\euro755 \\
      \bottomrule
      \\
      \rowcolor{white}
      \caption{Quarto incremento - consuntivo costo}
      \end{longtable} 
      \textbf{Considerazioni:} Per questo incremento il gruppo trae le seguenti considerazioni:
      \begin{itemize}
        \item Molti ruoli, quali \textit{Amministratore, Analista e Progettista} hanno subito un aumento di ore rispetto al previsto, questo perchè è stato necessario modificare, correggere e aggiornare i documenti fatti in precedenza;
        \item Per il ruolo di \textit{Verificatore}, invece, sono servite meno ore del previsto.
      \end{itemize} 
        \textbf{Conclusioni:} Il gruppo trae un bilancio negativo, in quanto il numero di ore necessarie a svolgere l'incremento, ha comportato, ovviamente, ad un aumento del costo dell'incremento. 
    }

        \subsubsection{Quinto incremento - Orario} {
            \setlength{\freewidth}{\dimexpr\textwidth-30\tabcolsep}
            \renewcommand{\arraystretch}{1.0}
            \setlength{\aboverulesep}{0pt}
            \setlength{\belowrulesep}{0pt}
            \rowcolors{2}{Arancione!10}{white}
            \begin{longtable}{C{.3\freewidth} C{.1\freewidth} C{.1\freewidth} C{.1\freewidth} C{.1\freewidth} C{.1\freewidth} C{.1\freewidth} C{.1\freewidth} C{.1\freewidth} C{.3\freewidth}}
            \toprule
            \rowcolor{Arancione}
            \textcolor{white}{\textbf{Componente}}&
            \textcolor{white}{\textbf{Re}}&
            \textcolor{white}{\textbf{Am}}&
            \textcolor{white}{\textbf{An}}&
            \textcolor{white}{\textbf{Pt}}&
            \textcolor{white}{\textbf{Pr}}&
            \textcolor{white}{\textbf{Ve}}&
            \textcolor{white}{\textbf{Ore}}&
            \textcolor{white}{\textbf{Ore previste}} \\
            \toprule
            \endhead
    
            Arena Angela & - & 1 & - & 10 & 20 & 2 & 33 & 21\\      
            Bellò Marco & - & - & - & 5 & 22 & 1 & 28 & 19\\      
            Bousapnamene & - & - & - & - & - & - & 0 & 21\\      
            Di Fant Tommaso & - & - & - & 3 & 28 & - & 31 & 22\\      
            Tossuto Matteo & - & 1 & - & 9 & 15 & - & 25 & 22\\      
            Volpato Marco & 5 & - & - & - & 30 & - & 35 & 22 \\      
            Zanatta Giulio & - & 1 & - & 6 & 15 & - & 22 & 21 \\      
            Totali & 5 & 3 & - & 33 & 130 & 3 & 174 & 148 \\
            \bottomrule
            \rowcolor{white}
            \\
            \caption{\centering{Quinto incremento - consuntivo orario}}

            \end{longtable}
             
            }

            \subsubsection{Quinto incremento - costo} {
                \setlength{\freewidth}{\dimexpr\textwidth-30\tabcolsep}
                \renewcommand{\arraystretch}{1.0}
                \centering
                \setlength{\aboverulesep}{0pt}
                \setlength{\belowrulesep}{0pt}
                \rowcolors{2}{Arancione!10}{white}
                \begin{longtable}{C{.4\freewidth} C{.2\freewidth} C{.2\freewidth}}
                \toprule
                \rowcolor{Arancione}
                \textcolor{white}{\textbf{Ruolo}}&
                \textcolor{white}{\textbf{Ore}}&
                \textcolor{white}{\textbf{Costo}}\\
                \toprule
                \endhead
          
                Responsabile & 5 & \euro150 \\
                Amministratore & 3 & \euro60 \\
                Analista & - & - \\
                Progettista & 33 & \euro825 \\
                Programmatore & 130 & \euro1950 \\
                Verificatore & 3 & \euro45 \\
                Totale & 174 & \euro3030 \\
                Preventivo & 148 & \euro2755 \\
                Differenza & 26 & +\euro275 \\
                \bottomrule
                \\
                \rowcolor{white}
                \caption{Quinto incremento - Consuntivo costo}
          
                \end{longtable}
                \textbf{Considerazioni:} Il gruppo trae le seguenti considerazioni:
                \begin{itemize}
                    \item Per i ruoli di \textit{Amministratore, Progettista e Verificatore} le ore necessarie per svolgere l'incremento sono state meno rispetto il previsto;
                    \item Al contrario per il ruolo di \textit{Programmatore}, infatti sono servite un numero maggiore di ore per consolidare la piattaforma.
                \end{itemize}
                \textbf{Conclusioni:} Per tale incremento il bilancio orario è negativo, perchè in totale sono servite un numero significativo di ore in più per terminare l'incremento.
                Inoltre l'aumento orario ha comportato un aumento del costo per l'incremento.
            }
            \subsubsection{Consuntivo di periodo} {
                \setlength{\freewidth}{\dimexpr\textwidth-30\tabcolsep}
                \renewcommand{\arraystretch}{1.0}
                \setlength{\aboverulesep}{0pt}
                \setlength{\belowrulesep}{0pt}
                \begin{longtable}{C{.4\freewidth} C{.4\freewidth} C{.4\freewidth}}
                \toprule
                \rowcolor{Arancione}
                \textcolor{white}{\textbf{Data inizio}}&
                \textcolor{white}{\textbf{Data previsione revisione}}&
                \textcolor{white}{\textbf{Data richiesta revisione}} \\
                \toprule
                \endhead
                    
                2022-08-08 & 2022-08-25 & 2022-09-05 \\
                \\
                \rowcolor{white}
                \caption{PB - Consuntivo periodo}
          
                \end{longtable}
                Il gruppo ha ritardato la consegna di 10 giorni rispetto la data prevista. Il gruppo ha dovuto lavorare in un lasso di tempo relativamente breve per riuscire a rimediare il ritardo 
                procurato con la consegna della \textit{Requirements and Technology Baseline}. Il gruppo infatti ha avuto bisogno di 50 ore in più rispetto al preventivo iniziale e quindi la data è stata posticipata.
                }
        \subsection{Customer Acception} {
            Si è deciso di riportare una tabella con le ore svolte da ciascun componente per analizzare meglio i consuntivi dei tre incrementi del periodo \textit{Customer Acceptance}.
            Dal momento in cui il preventivo nella sezione \S 4.3.1 era stato fatto alla versione \textit{1.0.0.} del \PdP, 
            si è ritenuto più opportuno confrontare con il preventivo fatto per la presentazione della \textit{PB}.
            \subsubsection{Sesto incremento - Orario} {
                \setlength{\freewidth}{\dimexpr\textwidth-30\tabcolsep}
            \renewcommand{\arraystretch}{1.0}
            \setlength{\aboverulesep}{0pt}
            \setlength{\belowrulesep}{0pt}
            \rowcolors{2}{Arancione!10}{white}
            \begin{longtable}{C{.3\freewidth} C{.1\freewidth} C{.1\freewidth} C{.1\freewidth} C{.1\freewidth} C{.1\freewidth} C{.1\freewidth} C{.1\freewidth} C{.1\freewidth} C{.3\freewidth}}
            \toprule
            \rowcolor{Arancione}
            \textcolor{white}{\textbf{Componente}}&
            \textcolor{white}{\textbf{Re}}&
            \textcolor{white}{\textbf{Am}}&
            \textcolor{white}{\textbf{An}}&
            \textcolor{white}{\textbf{Pt}}&
            \textcolor{white}{\textbf{Pr}}&
            \textcolor{white}{\textbf{Ve}}&
            \textcolor{white}{\textbf{Ore}} \\
            \toprule
            \endhead
    
            Arena Angela & 3 & - & - & 1 & - & - & 4\\      
            Bellò Marco & 3 & - & - & 2 & 2 & 1 & 8\\      
            Bousapnamene & - & - & - & 2 & - & 3 & 5\\      
            Di Fant Tommaso & - & - & - & 2 & 1 & 2  & 6\\      
            Tossuto Matteo & - & - & - & 4 & 2 & 3  & 9\\      
            Volpato Marco & - & - & - & 3 & 1 & 1 &  5 \\      
            Zanatta Giulio & - & - & - & 4 & 2 & 3 & 9 \\      
            Totali & 6 & - & - & 18 & 8 & 13 & 46 \\
            \bottomrule
            \rowcolor{white}
            \\
            \caption{\centering{Sesto incremento - consuntivo orario}}

            \end{longtable}
            }
            \subsubsection{Sesto incremento - Costo} {
                \setlength{\freewidth}{\dimexpr\textwidth-30\tabcolsep}
                \renewcommand{\arraystretch}{1.0}
                \centering
                \setlength{\aboverulesep}{0pt}
                \setlength{\belowrulesep}{0pt}
                \rowcolors{2}{Arancione!10}{white}
                \begin{longtable}{C{.4\freewidth} C{.2\freewidth} C{.2\freewidth} C{.2\freewidth}}
                \toprule
                \rowcolor{Arancione}
                \textcolor{white}{\textbf{Ruolo}}&
                \textcolor{white}{\textbf{Ore}}&
                \textcolor{white}{\textbf{Costo}}&
                \textcolor{white}{\textbf{Ore previste}}\\
                \toprule
                \endhead
          
                Responsabile & 6 & \euro180 & 5 \\
                Amministratore & - & - & 0 \\
                Analista & - & - & 0 \\
                Progettista & 18 & \euro450 & 14 \\
                Programmatore & 8 & \euro120 & 20 \\
                Verificatore & 13 & \euro195 & 10\\
                Totale & 46 & \euro945 & 49\\
                Preventivo & - & \euro950 & 49 \\
                Differenza & -3 & -\euro5 & -\\
                \bottomrule
                \\
                \rowcolor{white}
                \caption{Sesto incremento - Consuntivo costo}
          
                \end{longtable}

                \textbf{Considerazioni:} Le ore non sono cambiate radicalmente, ma c'è stata una dimunizione. Il gruppo trae le seguenti considerazioni:
                \begin{itemize}
                    \item Per il ruolo di \textit{Responsabile} si è preferito aggiungere un'ora, in quanto era importante che ci fosse una buona coordinazione per la gestione
                    dei test, dal momento in cui il tempo non era molto;
                    \item Per il ruolo di \textit{Progettista} è stato necessario incrementare le ore per progettare al meglio i test da eseguire, 
                    ovvero concentrarsi nell'analizzare le varie tipologie di test da fare, con relative documentazioni. La maggior parte dei componenti del gruppo
                    non avevano familiarità con l'escuzione di test e quindi era necessario spendere del tempo nello studio di essi;
                    \item Per il ruolo di \textit{Programmatore}, le ore sono notevolmente dimuinuite, proprio per il motivo spiegato nel punto precedente: ovvero era necessario studiare
                    alcune documentazioni prima di iniziare ad eseguire i test;
                    \item Per il ruolo di \textit{Verificatore} si è speso più tempo del previsto per supervisionare il lavoro altrui e per controllare che i test funzionassero adeguatamente.
                \end{itemize} 
                \textbf{Conclusioni:} Il gruppo trae un bilancio positivo per tale incremento, in quanto si è avuto una buona gestione del tempo dati i tempi stretti. 
                Le ore sono dimuinite rispetto al previsto, con un risparmio di 5 euro.
            }
            \subsubsection{Settimo incremento - Orario} {
                \setlength{\freewidth}{\dimexpr\textwidth-30\tabcolsep}
            \renewcommand{\arraystretch}{1.0}
            \setlength{\aboverulesep}{0pt}
            \setlength{\belowrulesep}{0pt}
            \rowcolors{2}{Arancione!10}{white}
            \begin{longtable}{C{.3\freewidth} C{.1\freewidth} C{.1\freewidth} C{.1\freewidth} C{.1\freewidth} C{.1\freewidth} C{.1\freewidth} C{.1\freewidth} C{.1\freewidth} C{.3\freewidth}}
            \toprule
            \rowcolor{Arancione}
            \textcolor{white}{\textbf{Componente}}&
            \textcolor{white}{\textbf{Re}}&
            \textcolor{white}{\textbf{Am}}&
            \textcolor{white}{\textbf{An}}&
            \textcolor{white}{\textbf{Pt}}&
            \textcolor{white}{\textbf{Pr}}&
            \textcolor{white}{\textbf{Ve}}&
            \textcolor{white}{\textbf{Ore}} \\
            \toprule
            \endhead
    
            Arena Angela & - & 1 & - & - & 2 & 2 & 5\\      
            Bellò Marco & - & 1 & - & - & 2 & 4 & 7\\      
            Bousapnamene & - & 1 & - & - & - & 3 & 4\\      
            Di Fant Tommaso & 3 & 1 & - & - & - & 1  & 5\\      
            Tossuto Matteo & 2 & 1 & - & - & 1 & 4  & 8 \\      
            Volpato Marco & - & - & - & - & 2 & 2 &  4 \\      
            Zanatta Giulio & - & 1 & - & - & 4 & 4 & 9 \\      
            Totali & 6 & 6 & - & - & 11 & 20 & 42\\
            \bottomrule
            \rowcolor{white}
            \\
            \caption{\centering{Settimo incremento - consuntivo orario}}

            \end{longtable}
            }
            }
            \subsubsection{Settimo incremento - Costo} {
                \setlength{\freewidth}{\dimexpr\textwidth-30\tabcolsep}
                \renewcommand{\arraystretch}{1.0}
                \centering
                \setlength{\aboverulesep}{0pt}
                \setlength{\belowrulesep}{0pt}
                \rowcolors{2}{Arancione!10}{white}
                \begin{longtable}{C{.4\freewidth} C{.2\freewidth} C{.2\freewidth} C{.2\freewidth}}
                \toprule
                \rowcolor{Arancione}
                \textcolor{white}{\textbf{Ruolo}}&
                \textcolor{white}{\textbf{Ore}}&
                \textcolor{white}{\textbf{Costo}}&
                \textcolor{white}{\textbf{Ore previste}}\\
                \toprule
                \endhead
          
                Responsabile & 5 & \euro150 & 5 \\
                Amministratore & 6 & \euro120 & 10\\
                Analista & - & - & - \\
                Progettista & - & - & -\\
                Programmatore & 11 & \euro165 & 10\\
                Verificatore & 20 & \euro300 & 20 \\
                Totale & 42 & \euro735 & 45 \\
                Preventivo & - & \euro800 & 45\\
                Differenza & -3 & -\euro65 & -\\
                \bottomrule
                \\
                \rowcolor{white}
                \caption{Settimo incremento - Consuntivo costo}
          
                \end{longtable}
                \textbf{Considerazioni:} Il gruppo trae le seguenti considerazioni:
                \begin{itemize}
                    \item Per il ruolo di \textit{Responsabile} le ore sono rimaste invariate, in quanto si è scelto, per garantire una buona gestione del lavoro, di lasciarle tassativamente tali;
                    \item Per il ruolo di \textit{Amministratore} sono dimuinite, in quanto i paragrafi dei documenti da aggiornare sono stati meno del previsto;
                    \item Per il ruolo di \textit{Programmatore} si è cercato di mantenere quelle previste per riuscire a non ritardare con l'inizio dell'ultimo incremento; 
                    \item Per il ruolo di \textit{Verificatore} il ragionamento è analogo a quello del punto precedente, in quanto un aumento delle ore avrebbe potuto portare a ritardi.
                \end{itemize} 
                \textbf{Conclusioni:} Il gruppo trae un bilancio positivo anche in questo incremento, le ore previste erano state preventivate durante la preparazione della revisione \textit{PB}, e quindi, 
                essendo state pensate in tempo recente, si è riuscito maggiornmente a rispettarle. Anche in questo caso, le ore sono state minori rispetto al prevsito , con un risparmio di 65 euro.
            }
            }
            \subsubsection{Ottavo incremento - Orario} {
                \setlength{\freewidth}{\dimexpr\textwidth-30\tabcolsep}
            \renewcommand{\arraystretch}{1.0}
            \setlength{\aboverulesep}{0pt}
            \setlength{\belowrulesep}{0pt}
            \rowcolors{2}{Arancione!10}{white}
            \begin{longtable}{C{.3\freewidth} C{.1\freewidth} C{.1\freewidth} C{.1\freewidth} C{.1\freewidth} C{.1\freewidth} C{.1\freewidth} C{.1\freewidth} C{.1\freewidth} C{.3\freewidth}}
            \toprule
            \rowcolor{Arancione}
            \textcolor{white}{\textbf{Componente}}&
            \textcolor{white}{\textbf{Re}}&
            \textcolor{white}{\textbf{Am}}&
            \textcolor{white}{\textbf{An}}&
            \textcolor{white}{\textbf{Pt}}&
            \textcolor{white}{\textbf{Pr}}&
            \textcolor{white}{\textbf{Ve}}&
            \textcolor{white}{\textbf{Ore}} \\
            \toprule
            \endhead
    
            Arena Angela & 1 & 1 & - & - & - & 2 & 4\\      
            Bellò Marco & - & 1 & - & - & 2 & 3 & 6\\      
            Bousapnamene & - & 2 & - & - & 3 & 4 & 9\\      
            Di Fant Tommaso & - & - & - & - & 1 & 2 & 3\\      
            Tossuto Matteo & - & 1 & - & - & 2 & 5  & 8 \\      
            Volpato Marco & 1 & - & - & - & 1 & 2 &  4 \\      
            Zanatta Giulio & 3 & - & - & - & 1 & 4 & 8 \\      
            Totali & 5 & 5 & - & - & 10 & 22 & 42 \\
            \bottomrule
            \rowcolor{white}
            \\
            \caption{\centering{Ottavo incremento - consuntivo orario}}

            \end{longtable}
            }
            
            \subsubsection{Ottavo incremento - Costo} {
                \setlength{\freewidth}{\dimexpr\textwidth-30\tabcolsep}
                \renewcommand{\arraystretch}{1.0}
                \centering
                \setlength{\aboverulesep}{0pt}
                \setlength{\belowrulesep}{0pt}
                \rowcolors{2}{Arancione!10}{white}
                \begin{longtable}{C{.4\freewidth} C{.2\freewidth} C{.2\freewidth} C{.2\freewidth}}
                \toprule
                \rowcolor{Arancione}
                \textcolor{white}{\textbf{Ruolo}}&
                \textcolor{white}{\textbf{Ore}}&
                \textcolor{white}{\textbf{Costo}} &
                \textcolor{white}{\textbf{Ore previste}}\\
                \toprule
                \endhead
          
                Responsabile & 5 & \euro150 & 5 \\
                Amministratore & 5 & \euro100 & 8 \\
                Analista & - & - & - \\
                Progettista & - & - & - \\
                Programmatore & 10 & \euro150 & 15 \\
                Verificatore & 22 & \euro330 & 20 \\
                Totale & 42 & \euro730 & 48  \\
                Preventivo & - & \euro875 & 48 \\
                Differenza & -6 & -\euro145 & - \\
                \bottomrule
                \\
                \rowcolor{white}
                \caption{Ottavo incremento - Consuntivo costo}
          
                \end{longtable}
                \textbf{Considerazioni:} Il gruppo trae le seguenti considerazioni:
                \begin{itemize}
                    \item Per il ruolo di \textit{Responsabile} e \textit{Amministratore} i ragionamenti sono analoghi all'incrmento precedente;
                    \item Per il ruolo di \textit{Programmatore} le ore sono diminuite grazie all'analisi e allo studio che sono stati fatti al \textit{sesto incremento}, i quali hanno portato a dei buoni risultati;
                    \item Per il ruolo di \textit{Verificatore} si è provato a mantenere le ore previste, ma hanno subito un leggero aumento per controllare che tutto funzionasse correttamente prima della consegna finale.
                \end{itemize} 
                \textbf{Conclusioni:} Il gruppo trae un bilancio positivo in quanto si è risciuto a gestire in buon modo il tempo, infatti sono servite 6 ore in meno rispetto al previsto, risparmiando 145 euro. Le ore che sono state impiegate nello studio hanno 
                portato ad una maggiore consapevolezza del lavoro che doveva essere svolto, e di conseguenza ad una migliore gestione di esso.
            }
            
            \subsubsection{Consuntivo di periodo} {
                \setlength{\freewidth}{\dimexpr\textwidth-30\tabcolsep}
                \renewcommand{\arraystretch}{1.0}
                \setlength{\aboverulesep}{0pt}
                \setlength{\belowrulesep}{0pt}
                \begin{longtable}{C{.4\freewidth} C{.4\freewidth} C{.4\freewidth}}
                \toprule
                \rowcolor{Arancione}
                \textcolor{white}{\textbf{Data inizio}}&
                \textcolor{white}{\textbf{Data previsione revisione}}&
                \textcolor{white}{\textbf{Data richiesta revisione}} \\
                \toprule
                \endhead
                    
                2022-09-019 & 2022-09-26 & 2022-09-26 \\
                \\
                \rowcolor{white}
                \caption{CA - Consuntivo periodo}
                \end{longtable}
            
            Il gruppo, a causa del ritardo di richiesta \textit{RTB}, aveva posto come data prevista di richiesta per la \textit{CA} il 26 Settembre, e non il 20 Settembre (data riportata nella sezione \S 4.1.3). \\
            Il gruppo, in questa fase di lavoro per la consegna finale, si è concentrata nei test, in quanto la piattaforma era stata conclusa per la \textit{PB}. Questa è la ragione per cui, le maggior ore spese sono state 
            quelle di progettazione, ovvero per progettare e analizzare i test da eseguire, e quelle di verifica, per controllare che il lavoro svolto non presentasse problemi. E' stato di grande aiuto il preventivo fatto di recente (per la \textit{PB}), il quale 
            ha permesso di proseguire con una migliore gestione dei compiti. Il gruppo effettua la consegna per la data prevista, con un numero minore di ore rispetto al previsto. Il costo totale è 13.175 euro. 
    }
