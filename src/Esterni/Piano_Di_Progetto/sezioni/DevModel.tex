Il gruppo ha deciso di usare una metodologia di sviluppo incrementale, 
basata su degli sprint, per i seguenti motivi:
\begin{itemize}
    \item 
        è stato esplicitamente consigliato dall'azienda
    \item 
        ogni incremento permette di definire ulteriormente 
        obiettivi e il metodo per raggiungerli
    \item 
        è richiesto un uso di tecnologie sconosciute, che forzano un 
        approccio ibrido tra studio ed utilizzo delle stesse
    \item
        testing e verifica sono facilitati dal fatto che possono essere 
        limitati a quanto prodotto nel singolo incremento
\end{itemize}
Gli incrementi dovranno essere prodotti, integrati e testati entro
una scadenza predefinita. Questi incrementi vanno a concretizzare 
prioritariamente i requisiti fondamentali del progetto, cioè quelli a 
priorità più alta. I requisiti a priorità più bassa, saranno realizzati
successivamente ed integrati in un sistema funzionalmente e qualitativamente robusto.
Gli incrementi a priorità più alta sono definiti MVP, cioè una versione
del prodotto con le caratteristiche funzionali minime per soddisfare un requisito.

\subsection{Incrementi Individuati}
È di seguito riportata la tabella con indicati gli incrementi individuati durante la fase 
di analisi, con i rispettivi obiettivo e requisiti associati. 
I requisiti includono tutti i requisiti figli. Tutti i requisiti non riportati sono da 
intendersi soddisfatti, in parte, da ogni incremento.
Nelle colonne "Requisiti" e "Casi d'uso" sono riportati gli stessi sotto forma di codice 
identificativo. Tutti i codici sono reperibili nel documento Analisi dei Requisiti.

{
      \newcolumntype{L}[1]{>{\raggedright\let\newline\\\arraybackslash\hspace{1pt}}m{#1}}
      \newcolumntype{C}[1]{>{\centering\let\newline\\\arraybackslash\hspace{1pt}}m{#1}}
      \newcolumntype{R}[1]{>{\raggedleft\let\newline\\\arraybackslash\hspace{1pt}}m{#1}}

      %\newlength{\freewidth}
      \setlength{\freewidth}{\dimexpr\textwidth-10\tabcolsep}
      \renewcommand{\arraystretch}{1.5}
      \centering
      \setlength{\aboverulesep}{0pt}
      \setlength{\belowrulesep}{0pt}
      \rowcolors{2}{Arancione!10}{white}
      \begin{longtable}{C{.257\freewidth} C{.257\freewidth} C{.257\freewidth} C{.257\freewidth}}
         \toprule
      \rowcolor{Rosso}
      \textcolor{white}{\textbf{Incremento}}&
      \textcolor{white}{\textbf{Descrizione}}&
      \textcolor{white}{\textbf{Requisiti}}&
      \textcolor{white}{\textbf{Casi D'Uso}}\\	
      \toprule
      \endhead
      
      0 & Formazione sia individuale, sia con il proponente, su temi che riguardano principalmente servizi AWS & - & - \\	   
      \bottomrule
      \end{longtable}
}


