Il gruppo ha deciso di usare una metodologia di sviluppo incrementale, basata su degli sprint, per i seguenti motivi:
\begin{itemize}
    \item 
        è stato esplicitamente consigliato dall'azienda
    \item 
        ogni incremento permette di definire ulteriormente 
        obiettivi e il metodo per raggiungerli
    \item 
        è richiesto un uso di tecnologie sconosciute, che forzano un 
        approccio ibrido tra studio ed utilizzo delle stesse
    \item
        testing e verifica sono facilitati dal fatto che possono essere 
        limitati a quanto prodotto nel singolo incremento
\end{itemize}
Gli incrementi dovranno essere prodotti, integrati e testati entro una scadenza predefinita. Questi incrementi vanno a concretizzare prioritariamente i requisiti fondamentali del progetto, cioè quelli a priorità più alta. I requisiti a priorità più bassa, saranno realizzati successivamente ed integrati in un sistema funzionalmente e qualitativamente robusto.
Gli incrementi a priorità più alta sono definiti MVP\G, cioè una versione del prodotto con le caratteristiche funzionali minime per soddisfare un requisito.




