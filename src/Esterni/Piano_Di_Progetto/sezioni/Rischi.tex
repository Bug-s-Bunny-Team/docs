L'analisi dei rischi è pensata come un progetto in divenire, modificato in modo incrementale nel caso compaiano rischi non inizialmente preventivati. Stesura, riconoscimento e risoluzione degli stessi richiede costante monitoraggio e impegno, infatti durante il corso del progetto potrebbero verificarsi problemi in grado di bloccare il lavoro per periodi di tempo anche prolungati. \\
 Di seguito riportiamo i vari rischi riscontrabili in forma schematizzata, divisi nelle seguenti sezioni:
\begin{itemize}
   \item \textbf{Rischi tecnologici}: problematiche legate alla gestione e suddivisione del lavoro su più persone o a rallentamenti imputabili ad una gestione non ottimale delle risorse;
   \item \textbf{Rischi organizzativi}: problematiche legate a comprensione ed uso delle tecnologie necessarie allo svolgimento del progetto;
   \item \textbf{Rischi interpersonali}: problematiche legate specificatamente all'interazione tra i membri del gruppo;
   \item \textbf{Rischi operativi}: problematiche legate alla possibile difficoltà di mettere in atto le pratiche di lavoro concordate.
\end{itemize}

   \subsection{Rischi tecnologici}
   
   \subsubsection*{RT1 - Scarsa esperienza}
   \begin{itemize}
   	\item \textbf{Descrizione}: quasi nessuno dei membri ha affrontato un progetto di questa ampiezza e complessità o utilizzato i tools AWS\G{} necessari;
   	\item \textbf{Notifica}: ogni membro si impegna a comunicare con trasparenza le difficoltà incontrate;
   	\item \textbf{Occorrenza}: elevata;
   	\item \textbf{Pericolosità}: elevata;
   	\item \textbf{Gestione}: il responsabile si occuperà di dividere i compiti onerosi su più membri, dividendo lo studio della documentazione, aiutando i componenti più in difficoltà.
   \end{itemize}

	\subsubsection*{RT2 - Tecnologie impiegate}
	\begin{itemize}
		\item \textbf{Descrizione}: la documentazione disponibile per la suite\G{} AWS\G{} è estremamente ampia, e lo studio della stessa potrebbe comportare una dilatazione delle tempistiche;
		\item \textbf{Notifica}: ogni membro si assume la responsabilità di essere al passo con il resto del gruppo o notificare una difficoltà in tal senso. Gli altri membri monitorano passivamente la preparazione degli altri;
		\item \textbf{Occorrenza}: elevata;
		\item \textbf{Pericolosità}: elevata;
		\item \textbf{Gestione}: il responsabile si dovrà occupare di spronare i membri in deficit, mediando col gruppo per fornire un aiuto in tal senso.
	\end{itemize}

   \subsubsection*{RT3 - Strumenti software}
   \begin{itemize}
   	\item \textbf{Descrizione}: il team si affida a strumenti software di terze parti (es: VSC\G) o piattaforme online (es: Drive\G, GitHub\G) per svolgere il progetto. Problemi di/con questi servizi si tradurrebbero in rallentamenti;
   	\item \textbf{Notifica}: posto che ogni membro deve evitare per quanto possibile di causare tali problemi, in caso occorrano la persona ha il compito di notificare il gruppo;
   	\item \textbf{Occorrenza}: media;
   	\item \textbf{Pericolosità}: bassa;
   	\item \textbf{Gestione}: i membri del gruppo si impegnano a fornire supporto tecnico quanto più completo e repentino possibile, entro i limiti delle proprie competenze.
   \end{itemize}

	\subsubsection*{RT4 - Problemi hardware}
	\begin{itemize}
		\item \textbf{Descrizione}: tutti i membri lavorano al progetto con dispositivi personali, quindi per la maggior parte sprovvisti di assistenza tecnica. Un guasto causerebbe non pochi disagi;
		\item \textbf{Notifica}: posto che ogni membro deve evitare per quanto possibile di causare tali problemi, in caso occorrano la persona ha il compito di notificare il gruppo;
		\item \textbf{Occorrenza}: bassa;
		\item \textbf{Pericolosità}: alta;
		\item \textbf{Gestione}: l'utilizzo di repository\G{} online permette perlomeno l'assenza di perdita dati, tuttavia per la mancanza di strumento lavorativo causa guasto non esiste una soluzione immediata, oltre la sostituzione.
	\end{itemize}

	\subsubsection*{RT5 - Servizi di terze parti}
	\begin{itemize}
		\item \textbf{Descrizione}: una delle parti cruciali del progetto consiste nel reperire informazioni da servizi di terze parti, quali Instragram\G{} e TikTok\G{} i quali potrebbero in qualsiasi momento interrompere o impedire azioni di scraping\G{} e crawling\G. Ciò porterebbe all'impossibilità di svolgere attività che fanno affidamento alla disponibilità di tali servizi e inoltre metterebbe a repentaglio la conclusione del progetto stesso;
		\item \textbf{Notifica}: ogni membro si impegna a comunicare il prima possibile eventuali problematiche in questo ambito, in modo da attuare celermente contromisure, anche in collaborazione con il proponente;
		\item \textbf{Occorrenza}: impossibile da calcolare;
		\item \textbf{Pericolosità}: alta;
		\item \textbf{Gestione}: essendo problematiche esterne al team e totalmente fuori il controllo di esso, è necessario un confronto con il proponente per valutare soluzioni alternative o mitigazioni da mettere in atto.
	\end{itemize}

   \subsection{Rischi organizzativi}

	\subsubsection*{ROR1 - Collaborazione a distanza}
	\begin{itemize}
		\item \textbf{Descrizione}: per motivi logistici il grosso del lavoro di organizzazione dei processi avviene a distanza, presentando quindi uno scoglio;
		\item \textbf{Notifica}: un qualsiasi membro del gruppo può chiedere un incontro in presenza se lo ritenesse necessario;
		\item \textbf{Occorrenza}: elevata;
		\item \textbf{Pericolosità}: bassa;
		\item \textbf{Gestione}: il gruppo può decidere di trovarsi in presenza per risolvere problemi di questo tipo.
	\end{itemize}

	\subsubsection*{ROR2 - Impegni}
	\begin{itemize}
		\item \textbf{Descrizione}: a causa di impegni universitari o altri, i componenti del gruppo potrebbero rallentare la stesura del progetto;
		\item \textbf{Notifica}: ogni componente è giusto che dichiari se ha  impegni che non gli consentono una buona dedizione al progetto;
		\item \textbf{Occorrenza}: media;
		\item \textbf{Pericolosità}: bassa;
		\item \textbf{Gestione}: ogni membro ad inizio settimana conosce il suo incarico e quello degli altri componenti, con relative scadenze degli obiettivi.
	\end{itemize}

	\subsubsection*{ROR3 - Analisi requisiti}
	\begin{itemize}
		\item \textbf{Descrizione}: parte fondamentale del progetto è l'analisi dei requisiti\G. Errori durante questo processo possono portare a gravi problemi del progetto;
		\item \textbf{Notifica}: ogni membro deve comunicare eventuali dubbi repentinamente;
		\item \textbf{Occorrenza}: bassa;
		\item \textbf{Pericolosità}: elevata;
		\item \textbf{Gestione}: per risolvere tale problema sarà necessario organizzare un incontro con l'azienda proponente.
	\end{itemize}

	\subsubsection*{ROR4 - Collaborazione con l'azienda}
	\begin{itemize}
		\item \textbf{Descrizione}: è necessario confrontarsi con l'azienda in caso di dubbi o conferme. L'azienda potrebbe non rispondere in tempi brevi (1/2 giorni lavorativi);
		\item \textbf{Notifica}: in caso di necessità e attese prolungate, il gruppo solleciterà il proponente\G;
		\item \textbf{Occorrenza}: bassa;
		\item \textbf{Pericolosità}: media;
		\item \textbf{Gestione}: il gruppo deve cercare di procedere, per quanto possibile, dove ha maggiori sicurezze, per non rimanere indietro nel progetto.
	\end{itemize}

   \subsection{Rischi interpersonali}

	\subsubsection*{RI1 - Collaborazione a distanza}
	\begin{itemize}
		\item \textbf{Descrizione}: a causa della distanza geografica tra i membri del gruppo, è sicuro che molti incontri saranno a distanza;
		\item \textbf{Notifica}: il gruppo si impegna a rivedere criticamente le modalità di collaborazione del gruppo e discutere con trasparenza di eventuali problemi;
		\item \textbf{Occorrenza}: alta;
		\item \textbf{Pericolosità}: media;
		\item \textbf{Gestione}: partecipazione attiva agli incontri del gruppo.
	\end{itemize}

	\subsubsection*{RI2 - Conflitti decisionali}
	\begin{itemize}
		\item \textbf{Descrizione}: potrebbero esserci idee non condivise e in disaccordo tra i membri, provocando situazioni di dibattito;
		\item \textbf{Notifica}: ogni membro deve sentirsi libero di esprimere il proprio disaccordo se un'idea non lo convince;
		\item \textbf{Occorrenza}: media;
		\item \textbf{Pericolosità}: media;
		\item \textbf{Gestione}: ogni proposta è giusto che sia valuta da ciascun componente e verrà scelta quella ritenuta più corretta dal gruppo.
	\end{itemize}

	\subsubsection*{RI3 - Condotta interna}
	\begin{itemize}
		\item \textbf{Descrizione}: potrebbe essere che uno o più componenti non siano reperibili o che non abbiano rispettato le scadenze prefissate;
		\item \textbf{Notifica}: ogni componente deve avvisare anticipatamente se non è riuscito a svolgere un lavoro o se non sarà disponibile per dei giorni;
		\item \textbf{Occorrenza}: media;
		\item \textbf{Pericolosità}: media;
		\item \textbf{Gestione}: settimanalmente ci sono degli incontri che servono per allinearsi tra noi membri per la stesura dei nostri compiti.
	\end{itemize}

   \subsection{Rischi operativi}
  
  	\subsubsection*{ROP1 - Collaborazione a distanza}
  	\begin{itemize}
  		\item \textbf{Descrizione}: collaborare su analisi o stesura di un documento, o decidere come procedere operativamente nei successivi slot di tempo risulta più complesso a distanza piuttosto che in presenza;
  		\item \textbf{Notifica}: rischio più generale che personale, notifica non richiesta (a meno che un membro specifico del gruppo abbia difficoltà particolari);
  		\item \textbf{Occorrenza}: elevata;
  		\item \textbf{Pericolosità}: bassa;
  		\item \textbf{Gestione}: l'impiego di strumenti come Google Docs\G{} e Discord\G{} è la soluzione per ora adottata al fine di arginare il problema il più possibile.
  	\end{itemize}