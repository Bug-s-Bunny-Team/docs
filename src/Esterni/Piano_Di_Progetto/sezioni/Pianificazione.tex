Il progetto viene partizionato in 8 incrementi, raggruppati dentro le 3 fasi corrispondenti alle revisioni obbligatorie. Ogni incremento funge da baseline per quello successivo. Fasi ed incrementi vengono descritti in dettaglio nelle sezioni a seguire:
\begin{enumerate}
    \item \textbf{Requirements and Technology Baseline}:
    \begin{itemize} 
    \item Baseline dei documenti;
    \item Baseline dei requisiti;
    \item Baseline delle tecnologie.
    \end{itemize}
    \item \textbf{Product baseline}:
    \begin{itemize} 
        \item baseline architetturale; 
        \item baseline del prodotto.
    \end{itemize}
    \item \textbf{Customer acceptance}:
    \begin{itemize}
        \item baseline delle funzionalità obbligatorie;
        \item baseline delle funzionalità opzionali e desiderabili;
        \item validazione di sistema.
    \end{itemize}
\end{enumerate}

\subsection{Fasi} 
{
    \subsubsection{Requirements and Technology Baseline} 
    {
        \setlength{\freewidth}{\dimexpr\textwidth-30\tabcolsep}
        \renewcommand{\arraystretch}{1.0}
        \setlength{\aboverulesep}{0pt}
        \setlength{\belowrulesep}{0pt}
        \begin{longtable}{C{.4\freewidth} C{.4\freewidth}}
        \toprule
        \rowcolor{Arancione}
        \textcolor{white}{\textbf{Data inizio}}&
        \textcolor{white}{\textbf{Data previsione revisione}} \\
        \toprule
        \endhead
            
        2022-04-11 & 2022-06-13 \\
        \bottomrule
        \\
        \caption{Revisione RTB}
        \end{longtable}
    
        \textbf{Incremento 1 - Baseline dei documenti} \\ 
    In questo primo incremento, il gruppo appena formato, definisce il proprio way of working\G, stendendo tali documenti:
    \begin{itemize}
        \item Norme di Progetto\G:
            \begin{itemize}
                \item strumenti per la documentazione: è necessario per poter proseguire con gli altri documenti;
                \item vengono specificate i processi: primari, di supporto e organizzativi. 
            \end{itemize}
        Entrambe vengono svolte dagli Amministratori.
        \item Piano di Progetto\G: 
            \begin{itemize}
                \item analisi dei rischi riscontrabili nel gruppo;
                \item il modello di sviluppo che verrà utilizzato;
                \item pianificazione generale delle attività: viene fatta una stima dei costi attesi e osservati;
                \item descrizione degli obiettivi che sono stati raggiungi e che devono essere ancora completati.
            \end{itemize}
        \item Piano di Qualifica\G: 
            \begin{itemize}
                \item specifica e misurazione degli obiettivi quantitativi di qualità di prodotto e di processo;
                \item verifica e approvazione: svolte dai Verificatori e Responsabili.
            \end{itemize}
        \item Glossario: 
            \begin{itemize}
                \item contiene le definizioni di alcune parole che si trovano nei documenti, dunque sarà aggiornato molto spesso.
            \end{itemize}
    \end{itemize}
    \textbf{Incremento 2 - Baseline dei requisiti} \\
    Per questo incremento, il gruppo si occupa nella stesura dei seguenti documenti:
    \begin{itemize}
        \item Analisi dei Requisiti\G:
            \begin{itemize}
                \item Casi d'uso\G: analisti
                \item Acquisizione dei Requisiti: formulazione in linguaggio naturale dei requisiti primari derivanti da capitolato e incontri con il proponente (analisti);
                \item Incontro con l'azienda proponente: discussione dei requisiti acquisiti con le aspettative del proponente, acquisizione di nuovi requisiti discussi con l'azienda;
                \item Specifica dei requisiti: traduzione in linguaggio formale e redazione dei requisiti all'interno del documento \AdR\G{} (analisti);
                \item Verifica: verifica e approvazione dei requisiti (verificatori).
            \end{itemize}
    \item Piano di Progetto\G: 
        \begin{itemize}
            \item Estensione della Pianificazione: grazie ai nuovi requisiti la pianificazione può essere aggiustata e migliorata.
        \end{itemize} 
    \end{itemize}
    \textbf{Incremento 3 - Baseline delle tecnologie} \\
    Il gruppo produce un PoC\G{} che dimostra la fattibilità del prodotto concordato con gli stakeholders\G. \\
    \begin{itemize}
        \item Documento (interno) di specifica e progettazione del PoC\G:
            \begin{itemize}
                \item scelta delle tecnologie: rapido resoconto delle tecnologie testate con motivazione della scelta delle stesse, svolto dai Progettisti;
                \item progettazione PoC\G: descrizione generale del PoC\G, svolto dai progettisti;
                \item verifica e approvazione: a carico dei Verificatori e del Responsabile.
            \end{itemize}
        \item PoC\G: 
            \begin{itemize}
                \item Formazione: 
                    \begin{itemize}
                        \item Studio delle tecnologie scelte, tramite documentazione e risorse fornite dall'azienda Zero12.
                    \end{itemize}
                \item Implementazione: 
                    \begin{itemize}
                        \item Codifica prototipo funzionante delle componenti software principali;
                        \item Uso di dati fittizi per la dimostrazione del funzionamento del sistema composto dai componenti codificati.
                    \end{itemize}
                \item Verifica: a carico dei Verificatori e del Responsabile
            \end{itemize}
    \end{itemize}
    
    \subsubsection{Product Baseline} {
        \setlength{\freewidth}{\dimexpr\textwidth-30\tabcolsep}
        \renewcommand{\arraystretch}{1.0}
        \setlength{\aboverulesep}{0pt}
        \setlength{\belowrulesep}{0pt}
        \begin{longtable}{C{.4\freewidth} C{.4\freewidth}}
        \toprule
        \rowcolor{Arancione}
        \textcolor{white}{\textbf{Data inizio}}&
        \textcolor{white}{\textbf{Data previsione revisione}} \\
        \toprule
        \endhead
            
        2022-06-14 & 2022-07-18 \\
        \bottomrule
        \\
        \caption{Revisione PB}
        \end{longtable}
    \textbf{Incremento 4 - Baseline architetturale} \\
    I componenti si occupano della stesura della Specifica Architetturale: 
    \begin{itemize}
        \item contiene i diagrammi delle classi e di sequenza e avviene una contestualizzazione al prodotto rispetto ai design pattern\G{} che vengono adottati (backend);
        \item struttura e funzionamento del crawler (backend);
        \item struttura pipeline\G{} di analisi dati tramite servizi AWS\G{} (backend);
        \item struttura dell'interfaccia\G{} della webapp (frontend).
    \end{itemize}
    Inoltre vengono aggiornati e redatti i documenti precedenti e la verifica è a carico dei Verificatori e del Responsabile.\\ \\
    \textbf{Incremento 5 - Baseline del prodotto} \\
    Questo incremento è fondamentale per la codifica del prodotto e si concentra sui seguenti punti:
    \begin{itemize}
        \item Implementazione dell'interfaccia\G: codifica di fronte-end e back-end, verifica e approvazione;
        \item Manuale d'uso: stesura a carico degli Amministratori.
    \end{itemize}
    }
\newpage
    \subsubsection{Customer Acceptance} {
        \setlength{\freewidth}{\dimexpr\textwidth-30\tabcolsep}
        \renewcommand{\arraystretch}{1.0}
        \setlength{\aboverulesep}{0pt}
        \setlength{\belowrulesep}{0pt}
        \begin{longtable}{C{.4\freewidth} C{.4\freewidth}}
        \toprule
        \rowcolor{Arancione}
        \textcolor{white}{\textbf{Data inizio}}&
        \textcolor{white}{\textbf{Data previsione revisione}} \\
        \toprule
        \endhead
            
        2022-07-19 & 2022-09-05 \\
        \bottomrule
        \\
        \caption{Revisione CA}
        \end{longtable}
        
    \textbf{Incremento 6 - Baseline funzionalità obbligatorie} \\
    Questo incremento include tutti i requisiti obbligatori che il sistema deve soddisfare.\\ \\
    %non so che altro scrivere
    \textbf{Incremento 7 - Baseline funzionalità desiderabili} \\ 
    Questo incremento è relativo all'implementazione dei requisiti desiderabili e opzionali. Ogni requisito che viene aggiunto deve essere incrementato nella Specifica Architetturale e nella definizione dei vari test che si eseguono. \\ \\
    \textbf{Incremento 8 - Test} \\ 
    Il gruppo si occupa di eseguire i  seguenti test  per correggere bug ed errori e per permettere una buona esecuzione della piattaforma. Vengono eseguiti:
    \begin{itemize}
        \item Collaudo o test di accettazione;
        \item Test di sistema per accertare la copertura dei requisiti;
        \item Test di integrazione per verificare il sistema in modo incrementale;
        \item Test di unità per verificare la correttezza del codice.
    \end{itemize} 
    }

}
