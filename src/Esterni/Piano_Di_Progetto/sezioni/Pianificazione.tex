Il progetto è partizionato in 8 incrementi, ognuno dei quali viene associato a una milestone e prevede dei risultati attesi, in modo da poter confrontare con quanto ipotizzato. 
Ogni incremento  serve a creare una baseline per quello successivo. 
Ogni revisione, tre totali, comprende degli incrementi che vengono suddivisi nel modo seguente:
\begin{enumerate}
    \item Requirements and Technology Baseline:
    \begin{itemize} 
    \item Baseline dei documenti
    \item Baseline dei requisiti
    \item Baseline delle tecnologie
    \end{itemize}
    \item Product baseline:
    \begin{itemize} 
        \item baseline architetturale 
        \item baseline del prodotto
    \end{itemize}
    \item Customer acceptance:
    \begin{itemize}
        \item baseline delle funzionalità obbligatorie
        \item baseline delle funzionalità opzionali e desiderabili
        \item validazione di sistema 
    \end{itemize}
\end{enumerate}

\subsection*{Periodi} 
{    
    \textbf{Periodo 1: Candidatura} \\
    \textbf{Inizio:} 4 Marzo \\
    \textbf{Fine:} 10 Aprile \\
    In questo periodo è stato scelto il nome e il logo del gruppo ed è stata creata la mail di riferimento. 
    Maggior tempo è stato dedicato alla scelta del capitolato, individuando le preferenze di ogni membro. 
    Sono stati svolti due incontri esterni online con 2 aziende, per poi scegliere definitivamente la nostra preferenza. 
    Successivamente è stata presentata una candidatura per tale capitolato, la quale poi è stata valutata positivamente.
    Stesura dei seguenti documenti, da consegnare per la candidatura:
    \begin{itemize}
        \item Lettera di Presentazione
        \item Valutazione dei Capitolati
    \end{itemize} 
    \textbf{Periodo 2: Preparazione RTB} \\
    \textbf{Inizio:} 11 Aprile \\
    \textbf{Fine:} revisione RTB - data \\
    In questa prima fase il gruppo, appena formato, definisce il proprio way of working, stendendo tali documenti:
    \begin{itemize} 
        \item Norme di Progetto: (vengono specificate le procedure e gli strumenti per i processi e le metriche di qualità)
        \item Piano di Progetto: (contiene i rischi riscontrabili nel gruppo, il modello di sviluppo che verrà utilizzato, i costi attesi e osservati e gli obiettivi che sono stati raggiungi e che devono essere ancora completati )
        \item Piano di Qualifica: (contiene la specifica e la misurazione degli obiettivi quantitativi di qualità di prodotto e di processo, con eventuali proposte di miglioramento)
        \item Analisi dei Requisiti
        \item Glossario
    \end{itemize} 
    Si precisa che tali documenti vengono costantemente aggiornati e modificati, ovvero redatti in maniera just in time, 
    questo perché si cerca di renderli migliori e più efficaci possibili. 
    Viene inoltre realizzato il Proof of Concept.
}
