\subsection{Scopo del documento}
Le \NdP{} hanno l'obbiettivo di stabilire e normare i processi utili allo sviluppo del prodotto attraverso la definizione di \emph{best practices} e del \emph{way of working}.
Il presente documento deve essere visionato da ogni membro del gruppo, il quale dovrà attenersi alle norme qui descritte durante l'intera durata del progetto.
Questo documento viene frequentemente aggiornato per assicurarsi che le norme descritte rispecchino la realtà operativa del gruppo rendendo il documento un importante risorsa.
\subsection{Scopo del prodotto}
Il capitolato propone lo sviluppo di una piattaforma simile ad una guida Michelin, basandosi sulle esperienze che vengono condivise sui social network Instagram e TikTok.
La richiesta prevede che la piattaforma sia in grado di ispezionare ed estrarre determinate informazioni quali immagini, audio o commenti relativi al contenuto analizzato, dalle storie dei relativi social network.
L'obiettivo è quello di riuscire a formare una mappa di location e determinare se quest'ultime vengono recensita negativamente o positivamente, e a tal scopo stilare un ranking di esse incrociando ciò che viene analizzato dalla piattaforma con altre classifiche per rendere omogeneo il risultato.
Il progetto sarà avrà un'architettura a microservizi.
%Inserire Glossario?
\subsection{Riferimenti} %Da rivedere cosa inserire 
\subsubsection{Riferimenti normativi}
\begin{itemize}
	\item
	\href{https://www.math.unipd.it/~tullio/IS-1/2021/Progetto/C4p.pdf}{\textbf{Capitolato d'Appalto C4}}
	\item
	\href{https://www.math.unipd.it/~tullio/IS-1/2021/Dispense/PD2.pdf}{\textbf{Regolamento del progetto didattico}}
\end{itemize}
\subsubsection{Riferimenti informativi}
\begin{itemize}
	\item \href{https://www.math.unipd.it/~tullio/IS-1/2009/Approfondimenti/ISO_12207-1995.pdf}{\textbf{ISO/IEC 12207:1997}}
	\item
	\href {https://www.math.unipd.it/~tullio/IS-1/2021/Dispense/T06.pdf}{\textbf{Gestione di Progetto}}
	\item
	\href{https://www.math.unipd.it/~tullio/IS-1/2021/Dispense/T09.pdf}{\textbf{Progettazione software}}
\end{itemize}
	\paragraph{\textbf{Servizi Amazon:}} 
\begin{itemize}
	\item
	\href {https://docs.aws.amazon.com/cognito/latest/developerguide/cognito-user-identity-pools.html}{\textbf{Cognito User Pools}}
	\item
	\href{https://docs.aws.amazon.com/comprehend/latest/dg/what-is.html}{\textbf{Comprehend}}
	\item 
	\href{https://docs.aws.amazon.com/rekognition/latest/dg/what-is.html}{\textbf{Rekognition}}
	\item 
	\href{https://docs.aws.amazon.com/apigateway/latest/developerguide/welcome.html}{\textbf{API Gateway}}
	\item
	\href{https://docs.aws.amazon.com/lambda/latest/dg/welcome.html}{\textbf{AWS Lambda}}
	\item 
	\href{https://docs.aws.amazon.com/serverless-application-model/latest/developerguide/what-is-sam.html}{\textbf{AWS Serverless}}
\end{itemize}
