\subsection{Documentazione}
	\subsubsection{Scopo}
	Il processo di documentazione comprende le attività di stesura e aggiornamento di tutti i documenti creati durante il ciclo di vita del \emph{software} in modo da renderli formalmente concordi. In particolare, in questa sezione ne vengono normate le attività, comprendenti stesura, \emph{verifica} e \emph{validazione}. 
	
	\subsubsection{Aspettative}
	Durante questo processo il \emph{team} ha le seguenti aspettative:
		\begin{itemize}
			\item Ideare una struttura ben organizzata comune a tutti i documenti;
			\item Stesura di norme per facilitare tale processo.
		\end{itemize}
	
	\subsubsection{Documenti prodotti}
		I documenti prodotti saranno di due tipi:
			\begin{itemize}
				\item \textbf{Formali}:
				\begin{itemize}
					\item \textbf{interni}: riguardanti le dinamiche del gruppo;
					\item \textbf{esterni}: di interesse ai committenti e al \emph{proponente}.
				\end{itemize} 
				In particolare, i documenti formali prodotti saranno:
					\begin{itemize}
						\item \textbf{\NdP}: Contiene tutte le regole stabilite dai membri alle quali attenersi durante l'intera durata del progetto (Interno);
						\item \textbf{\VdC}: Contiene l'analisi dei capitolati messi a disposizione, evidenziandone pregi e difetti. Contiene, inoltre, il capitolato scelto dal gruppo (Interno);
						\item \textbf{\AdR}: Descrive i \emph{requisiti} che il prodotto dovrà possedere per essere in linea con le richieste dei committenti (Esterno);
						\item \textbf{\PdP}: Contiene la pianificazione di tutte le attività previste, comprendente il preventivo delle spese e una previsione dell'impegno in ore per ogni membro del gruppo (Esterno);
						\item \textbf{\PdQ}: Descrive i criteri di valutazione della qualità impiegate dal gruppo (Esterno).
					\end{itemize}
		
			\item \textbf{Informali}: 
			In particolare, i documenti informali prodotti saranno:
				\begin{itemize}
					\item \textbf{Verbali interni}: Contengono le informazioni e le decisioni prese durante gli incontri tra i membri del gruppo;
					\item \textbf{Verbali esterni}: Comprendono le informazioni ed i chiarimenti ricevuti durante gli incontri tra i membri ed il committente o tra i membri ed il \emph{proponente};
				\end{itemize}
		\end{itemize}
	
	\subsubsection{Ciclo di vita di un documento} %da rivedere
		Ogni documento passa per le seguenti fasi:
			\begin{itemize}
				\item \textbf{Stesura}: Il documento viene creato, aggiornato e modificato fino ad \emph{Approvazione};
				\item  \textbf{Verifica}: Il documento viene verificato sia dal punto di vista grammaticale che contenutistico; la verifica viene effettuata da una o più persone diverse dall'autore del documento;
				\item \textbf{Approvazione}: Il documento viene approvato quando tutti i verificatori concordano che sia tutto corretto. %Non mi convince molto
			\end{itemize}
	\subsubsection{Struttura dei documenti}
		%Magari spiegare nome dei verbali
		\paragraph{Frontespizio}    
			La prima pagina di ogni documento sarà così strutturata:
				\begin{itemize}
					\item \textbf{Logo Università e didascalia}: Posizionato in alto a sinistra e al suo fianco, al centro, la scritta "Università degli studi di Padova", corso del progetto e anno accademico;
					\item \textbf{Logo gruppo e nome}: Entrambi al centro, il nome del gruppo è sotto il logo;
					\item \textbf{Titolo del documento};
					\item \textbf{Recapito email del gruppo};
					\item \textbf{Informazioni del documento}: Redattore, Verificatore, Approvatore ed Uso(interno od esterno).
				\end{itemize}
	
		\paragraph{Corpo del documento}       
			Tutte le pagine del corpo del documento contengono un intestazione composta da:
				\begin{itemize}
					\item \textbf{Logo Gruppo}: a sinistra;
					\item \textbf{Titolo Documento}: a destra;
				\end{itemize}
			e un piè di pagina al cui centro c'è la pagina corrente del documento.
	
	\subsubsection{Norme tipografiche}
		\paragraph{Convenzioni di denominazione}   
			%Aggiungere Introduzione
		\paragraph{Stili di testo} %Ha senso?
			\begin{itemize}
				\item \textbf{Grassetto}
				\item \textbf{Corsivo}
			\end{itemize}
	
		%Aggiungere Paragrafo Tabelle e Paragrafo Formato Data e Ora?
						
		%Elementi grafici ? Come tabelle ed immagini
		
		\subsubsection{Strumenti}
			\begin{itemize}
				\item \textbf{LaTeX:} stesura in bella copia dei documenti caricati poi sul repository del gruppo;
				\item \textbf{Visual Studio Code:} usato per la scrittura di codice e documenti, e visione del versioning tramite collegamento a GitHub;
				\item \textbf{Google Docs:} cartella condivisa contenente documenti in brutta copia, permette scrittura collaborativa contemporanea da parte di tutti i membri con visualizzazione delle modifiche in live;
			\end{itemize}
		
	\subsection{Processo di Configurazione}
		%Scopo e aspettative ?
		\paragraph{Tecnologie adottate}
			Per gestire il versionamento del codice sorgente, viene utilizzato il sistema di versionamento distribuito \emph{Git}, con un \emph{repository} remoto presente su \emph{GitHub}. %Aggiungere link ??

		\paragraph{Struttura dei repository}
		Il gruppo mantiene diversi repository separati, ognuno con struttura e fini propri.
		Di base, ogni repository conterrà i seguenti file:
		\begin{itemize}
			\item \verb#README.md#: contiene istruzioni d'uso, installazione, e sviluppo proprie del repository. Ne viene inoltre descritta la struttura nel dettaglio. Redatto in formato \textit{Markdown}.
			\item \verb#.gitignore#: descrive quali file ignorare dal sistema di versionamento, deve essere utilizzato per evitare la proliferazione di file inutili e che non necessitano di versionamento. È consigliato utilizzare \href{https://www.toptal.com/developers/gitignore}{servizi di generazione} per generare le regole.
			\item \verb#.editorconfig#: descrive regole di formattazione, quali indentazione e stile dei fine riga. Viene utilizzato dai formattatori automatici degli \textit{IDE} ed editor di testo. Segue il formato \href{https://editorconfig.org/}{\textit{EditorConfig}}.
		\end{itemize}

		\paragraph{Norme di branching}
			Il \emph{repository} inerente alla documentazione sarà composto da diversi \textit{branch}:
			\begin{itemize}
				\item \verb#master#: \emph{branch} principale che viene aggiornato quando un documento è approvato o per introdurre nuove funzionalità; 
				\item un \textit{branch} per ogni documento o tipologia di documento.
			\end{itemize}
		
		\subsubsection{Strumenti}
			Per la gestione della configurazione o versionamento il gruppo utilizza \emph{Visual Studio Code} o i seguenti client \emph{Git}:
				
				\begin{itemize}
					\item \textbf{GitHub Desktop}: Client ufficiale di \emph{GitHub} utilizzato per la gestione delle \emph{repository} \emph{Git};
					\item \textbf{GitKraken}.
				\end{itemize}\
		
		\subsection{Gestione della qualità}
			\subsubsection{Scopo}
				Per la gestione della qualità è dedicato il documento \PdQ{}: il documento fissa i \emph{requisiti} qualitativi individuati dagli \emph{stakeholder} e le metriche per la \emph{verifica} e \emph{validazione} per garantire la qualità del prodotto finale.
		
			\subsubsection{Aspettative}
				 Le aspettative di questo processo sono:
					\begin{itemize}
						\item conseguimento della qualità nel prodotto, secondo le richieste del \emph{proponente};
						\item prova oggettiva della qualità del prodotto;						
						\item conseguimento della qualità nell'organizzazione delle attività del gruppo e dei processi;
						\item raggiungimento della piena soddisfazione del \emph{proponente}.
					\end{itemize}
		
			%Magari subsubsection per norme e regole per facilitare gestione della qualità da parte del gruppo: come continua comunicazione, lavoro costante, rispetto delle norme e cosi via
	
		\subsection{Verifica}
			\subsubsection{Scopo}
				Il processo di \emph{verifica} viene applicato per individuare eventuali errori introdotti nel prodotto durante la fase di sviluppo di un processo. La \emph{verifica} viene applicata sia alla documentazione che al codice. %in futuro
			\subsubsection{Aspettative}
				Le aspettative di questo processo sono:
					\begin{itemize}
						\item seguire procedure definite con criteri chiari ed affidabili;
						\item verificare ad ogni fase;
						\item la verifica deve garantire che il prodotto si trovi in uno stato stabile;
						\item la verifica deve risultare più automatica possibile.
					\end{itemize}
				
			\subsubsection{Attività}
				\paragraph{Analisi}
					Il processo di analisi si suddivide in Analisi statica e Analisi dinamica:
						\subparagraph{Analisi statica}
							L'analisi statica permette di effettuare controlli su documenti e codice, verificando così l'assenza di errori e difetti. Esistono due metodologie per applicarla:
								\begin{itemize}
									\item \textbf{Walkthrough:} consiste nella lettura da parte del \emph{team} dell'intero documento o codice in cerca di anomalie. Viene applicata quando non si conosce in modo chiaro la sorgente dei difetti. Questa tecnica risulta molto onerosa in termini di efficienza ed efficacia.
									\item \textbf{Inspection:} consiste in una lettura mirata del documento o del codice nei punti in cui si sa già che possano essere presenti degli errori. Risulta meno dispendiosa in termini di tempo ma richiede una buona conoscenza della situazione.
								\end{itemize}
						\subparagraph{Analisi dinamica}
							L'analisi dinamica prevede l'esecuzione del prodotto \emph{software} e la sua analisi tramite l'utilizzo di \emph{test} che verificano se il prodotto funziona o se vi sono presenti anomalie
						\subparagraph{Test}
							L'attività di testing è la base dell'analisi dinamica. I \emph{test} permettono di individuare tutti i possibili errori che possono essere stati commessi e tutti i casi limite che possono risultare problematici. \newline
							I test sono ben progettati e scritti se e solo se:
								\begin{itemize}
									\item Sono ripetibili;
									\item Sono automatici;
									\item Forniscono informazioni tramite artefatti di vario genere, quali file di log;
								\end{itemize}
		
		\subsection{Processo di validazione} % Terminare
			\subsubsection{Scopo}
				Il processo di \emph{validazione} prende in esame il prodotto ottenuto dalla fase di \emph{verifica} e stabilisce se esso rispetti i requisiti e le aspettative del \emph{committente}.
			\subsubsection{Aspettative}
				Le aspettative di questo processo sono:
					\begin{itemize}
						\item Help%Aggiungere aspettative
					\end{itemize}
