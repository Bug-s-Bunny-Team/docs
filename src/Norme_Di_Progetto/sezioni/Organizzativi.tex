\subsection{Gestione dei Processi} %Da lavorarci su
\subsubsection{Scopo}
La presente sezione espone gli strumenti impiegati dal gruppo \teamname{} per quanto concerne le attività di comunicazione interna ed esterna, l'organizzazione e la gestione dei ruoli di ogni componente. %Manca qualcosa sicuro
A tale processo è dedicato il documento \PdP{}.

\subsubsection{Aspettative}
Le aspettative di questo processo sono:
\begin{itemize}
	\item Definizione strumenti e modalità di comunicazione;
	\item Definizione strumenti, modalità e norme di organizzazione;
	\item Definizione norme per la gestione dei ruoli dei membri del gruppo.
\end{itemize}

\subsection{Strumenti di comunicazione}
Il gruppo \teamname{} utilizza vari strumenti per la comunicazione sia interna che esterna:
	\subsubsection{Comunicazione Interna}:
		Le comunicazioni interne riguardano solamente i membri del gruppo \teamname{} attraverso l'utilizzo dei seguenti strumenti:
			\begin{itemize}
				\item \textbf{Telegram}: Usato per pianificare incontri su \emph{Discord} e per comunicazioni rapide e non troppo importanti;
				\item \textbf{Discord}: Usato per incontri di discussione e pianificazione sul lungo periodo; inoltre è stato integrato con \emph{GitHub} per la ricezione di messaggi riguardo al versionamento;
			\end{itemize}
	\subsubsection{Comunicazioni Esterne}:
		Le comunicazioni esterne riguardano le comunicazione tra i membri del gruppo \teamname{} e persone esterne al \emph{gruppo}:
			\begin{itemize}
				\item \textbf{ProtonMail}: Usato per la comunicazione scritta crittografata con l'azienda \proponente{} e i \emph{committenti};
				\item \textbf{Slack}: Usato per la comunicazione con il rappresentante dell'azienda \proponente{};
			\end{itemize}

%Inserire norme che regolano comunicazione interna ed esterna

\subsection{Ruoli di progetto}
Il \RdP{} deve assicurarsi che ogni membro del \emph{gruppo} ricopra almeno una volta ogni ruolo, che sono i seguenti:
	\begin{itemize}
		\item \textbf{\RdP}: è la figura centrale del progetto, esso ha il compito di coordinare il \emph{team} e il ruolo di rappresentante del progetto durante le comunicazioni con l'esterno;
		\item \textbf{\Amm}: ha il compito di gestire e configurare adeguatamente le piattaforme utilizzate dal gruppo per lo svolgimento del progetto. Da lui dipendono l'affidabilità e l'efficacia dei mezzi scelti per lo svolgimento del progetto;
		\item \textbf{\Ana}: segue il progetto principalmente nelle fasi iniziali ed è fortemente coinvolto nella stesura dell' \AdR{}. Il suo ruolo è quello di analizzare i problemi posti dal progetto e chiarire le dipendenze e le ramificazioni di ogni attività necessaria alla consegna del prodotto;
		\item \textbf{\Prog}: segue lo sviluppo del progetto e, a partire dai \emph{requisiti}, definisce le scelte tecniche necessarie per lo sviluppo del prodotto;
		\item \textbf{\Progm}: ha il compito di codificare i modelli realizzati dal \Prog{}. Il codice prodotto dal \Progm{} deve attenersi il più possibile alle specifiche elaborate dal \Prog{} e documentare opportunamente il codice creato per aumentarne la manutenibilità;
		\item \textbf{\Ver}: segue l'intero ciclo di vita del progetto. Egli si assicura che la qualità della documentazione prodotta aderisca alla norme stabilite;
	\end{itemize}

\subsubsection{Assegnazione dei compiti}
Lorem ipsum %Da aggiungere