La scelta di questo capitolato è stata determinata da due fattori principali: l'interesse nei confronti delle tecnologie utilizzate e l'impressione positiva data dalla riunione con il referente aziendale Michele Massaro.
Il progetto richiede infatti l'uso di vari servizi offerti da AWS per implementare un'infrastruttura serverless, che analizza i dati ottenuti dai crawlers servendosi di neural networks pre-addestrate al riconoscimento e all'estrapolazione di dati dalle immagini. \newline
Il proponente si è dimostrato veloce nella comunicazione (fornendo anche un canale Slack) e disponibile a dare supporto dal punto di vista formativo tramite un aiuto nella fase di analisi ed organizzando veri e propri corsi comunitari sui servizi AWS che si andranno ad utilizzare.
Per un resoconto più completo dell'incontro con zero12 e sulla scelta del progetto si rimanda al \href{https://github.com/Bug-s-Bunny-Team/docs/blob/master/docs/Verbali/Esterni/ve_20220316.pdf}{verbale} del 16/3/2022.
