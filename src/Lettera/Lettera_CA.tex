\documentclass{articoletteracdp}
\usepackage{graphicx}
\usepackage{xcolor}
\usepackage[colorlinks=true, urlcolor=blue]{hyperref}
\usepackage{eurosym}
\usepackage{tabularx} %tabelle larghe

\newcommand{\teamname}{\emph{Bug's Bunny}}
\newcommand{\progetto}{Guida Michelin Social}
\newcommand{\proponente}{zero12}
\newcommand{\github}{https://github.com/Bug-s-Bunny-Team/docs}
\newcommand{\preventivo}{Preventivo Costi e Impegni Orari}
\newcommand{\email}{bugsbunnyteam@protonmail.com}

\newcommand{\angela}{Angela Arena}
\newcommand{\giulio}{Giulio Zanatta}
\newcommand{\ruth}{Ruth Genevieve Bousapnamene}
\newcommand{\tommaso}{Tommaso Di Fant}
\newcommand{\matteo}{Matteo Tossuto}
\newcommand{\marcob}{Marco Bellò}
\newcommand{\marcov}{Marco Volpato}

\newcommand{\RdP}{\emph{Responsabile di Progetto}}
\newcommand{\Amm}{\emph{Amministratore}}
\newcommand{\Ana}{\emph{Analista}}
\newcommand{\Prog}{\emph{Progettista}}
\newcommand{\Progm}{\emph{Programmatore}}
\newcommand{\progrs}{\emph{Programmatori}}
\newcommand{\Ver}{\emph{Verificatore}}
\newcommand{\verf}{\emph{Verificatori}}

\newcommand{\VdC}{\emph{Valutazione dei Capitolati}}
\newcommand{\NdP}{\emph{Norme di Progetto}}
\newcommand{\AdR}{\emph{Analisi dei Requisiti}}
\newcommand{\PdP}{\emph{Piano di Progetto}}
\newcommand{\PdQ}{\emph{Piano di Qualifica}}
\newcommand{\Glo}{\emph{Glossario}}

\newcommand{\aCapo}{ ~ \vspace{0.25cm} \\} %magari fare renewcommand di paragraph 

\newcolumntype{L}[1]{>{\raggedright\let\newline\\\arraybackslash\hspace{0pt}}m{#1}}
\newcolumntype{C}[1]{>{\centering\let\newline\\\arraybackslash\hspace{0pt}}m{#1}}
\newcolumntype{R}[1]{>{\raggedleft\let\newline\\\arraybackslash\hspace{0pt}}m{#1}}

\definecolor{Arancione}{RGB}{245,100,0}


\address{
	\begin{figure}[t!]
		\begin{minipage}{.5\textwidth}
			\centering
			\includegraphics[width=.5\linewidth]{assets/unipd}
		\end{minipage}%
		\begin{minipage}{.5\textwidth}
			\centering
			\includegraphics[width=.5\linewidth]{assets/logo}
		\end{minipage}
	\end{figure}
	\vspace{0.2cm}
	Ingegneria del Software\\
	Gruppo: \textit{\teamname} \\
	Email: \href{mailto:bugsbunnyteam@protonmail.com}{\textit{\email}}
}

\begin{document}
	\begin{letter}{
			Prof. Tullio Vardanega \\
			Prof. Riccardo Cardin \\
			Università degli Studi di Padova \\
			Dipartimento di Matematica \\
			Via Trieste, 63 \\
			35121 Padova
		}

		\opening{Egregio Professore Tullio Vardanega, \newline Egregio Professore Riccardo Cardin,}

		\begin{quotation}
			\noindent
			Con la presente il gruppo \textit{\teamname} intende comunicarVi ufficialmente la partecipazione
			alla terza Revisione di Avanzamento (\textit{CA}), con lo scopo di presentare il
			prodotto da Voi commissionato e proposto dall'azienda \textit{\proponente}. \newline
			Al seguente link: \href{\github}{\textit{\github}} è presente la documentazione necessaria,
			di seguito esplicitamente catalogata: 
			\vspace{0.3cm}

			\begin{center}
				\newcolumntype{L}[1]{>{\raggedright\let\newline\\\arraybackslash\hspace{0pt}}m{#1}}
				\newcolumntype{R}[1]{>{\raggedleft\let\newline\\\arraybackslash\hspace{0pt}}m{#1}}
				\textbf{Documenti Esterni}\\ \vspace{0.25cm}
				\begin{tabular}{L{4.2cm} | R{5cm}}
					\PdP{} \textit{v3.0.0} & \PdQ{} \textit{v3.0.0} \\
					\MU{} \textit{v2.0.0} & \SA{} \textit{v2.0.0} \\
					\Glo & Verbale Esterno 23-09-2022 \\
				\end{tabular} 

				\vspace{0.5cm}
				\textbf{Documenti Interni}\\ \vspace{0.25cm}
				\begin{tabular}{L{4.2cm} | R{4.2cm}}
					\NdP{} \textit{v2.0.0} & Verbale Interno 19-09-2022 \\
				\end{tabular}
			\end{center}
        \end{quotation}

		\begin{quotation}
			Nel seguente repository è possibile trovare il codice sorgente del prodotto:
			\begin{center}
				\href{https://github.com/Bug-s-Bunny-Team/bunnyfood}{\textit{https://github.com/Bug-s-Bunny-Team/bunnyfood}}
			\end{center}
		\end{quotation}
		
		\newpage

        \begin{quotation}
			\noindent
			Di seguito si riportano i nomi dei componenti del gruppo e i rispettivi numeri di matricola:

				\begin{center}
					\setlength{\extrarowheight}{.75ex}
					\begin{tabular}{ c | c }
						\textbf{Nominativo}           & \textbf{Matricola} \\
						\hline

						Arena Angela{}                & 1226299{}          \\
						Bellò Marco{}                 & 1142874{}          \\
						Bousapnamene Ruth Genevieve{} & 1192088{}          \\
						Di Fant Tommaso{}             & 1217932{}          \\
						Tossuto Matteo{}              & 1193493{}          \\
						Volpato Marco{}               & 1224826{}          \\
						Zanatta Giulio{}              & 1163163{}          \\

						\hline
					\end{tabular}
				\end{center}
			
		\end{quotation}

		\begin{quotation}
			\noindent
			Il gruppo consegna il prodotto il
			2022-09-27 come indicato nel \PdP{} \textit{v3.0.0}.
		\end{quotation}

		\vspace{0.5cm}
		Rimaniamo a Vostra completa disposizione per qualsiasi chiarimento.

		\vspace{0.5cm}
		\closing{Cordiali Saluti,}
		\begin{figure}[!h]
			\begin{flushright}
				\includegraphics[width=0.2\linewidth]{sezioni/firma_mb.png}
			\end{flushright}
			\begin{flushright}
				\rule{100pt}{0.1pt}\\
				Firma Responsabile
			\end{flushright}
		\end{figure}
	\end{letter}
\end{document}
