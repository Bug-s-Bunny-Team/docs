\documentclass{classes/base}

\title{Verbale interno}
\date{2022/03/31}
\author{Nomone Cognomone}

\renewcommand{\maketitle}{
    \begin{titlepage}
    \begin{center}
        \makeatletter
        \vspace*{\fill}
        
        %\includegraphics[width=2.5cm]{assets/unipd}
        %\subsection*{Università degli Studi di Padova}
        %\vspace{2cm}
        
        \begin{minipage}[]{0.3\textwidth}
            \centering
            \includegraphics[width=3cm]{assets/unipd}
            \bigskip
        \end{minipage}
        \begin{minipage}[]{0.7\textwidth}
            \centering
            \color[HTML]{B5121B}{
                \textbf{Università degli Studi di Padova} \\
                Ingegneria del Software \\
                Anno Accademico: 2021/2022 \\
                }
                \vspace*{2cm}
        \end{minipage}
        

        \includegraphics[width=5cm]{assets/logo}

        \Huge
        \textbf{\teamname}
        
        \vspace{3cm}
        
        \Huge
        \textbf{\@title}

        \Large
        \@date

        \vspace{3cm}
        
        \textbf{Redazione:} \@author\\
        \textbf{Verifica:} \@verificatore\\
        \textbf{Approvazione:} \@approvatore\\
        
        \vfill
        \makeatother
    \end{center}
\end{titlepage}

}

\begin{document}
    \maketitle

    \section{Generalità}
    \begin{itemize}
        \item \textbf{Ora inizio:} 15.00
        \item \textbf{Ora fine:} 17.00
    \end{itemize}
    
    \section{Resoconto}
    E' stata fatta una discussione sui documenti e sulla valutazione delle tempistiche di consegna, in particolare è stata posta attenzione sui documenti necessari alla presentazione della candidatura.\\
    Nella sezione riguardante il capitolato scelto si è deciso di redigere un documento necessario per riassumere gli incontri avuti con il committente e motivarne la scelta.\\\\

    \section*{Valutazione tempistiche}
    Ipotizzando 100 ore produttive a persona, 2 ore al giorno (quindi 10 ore settimanali con settimana lavorativa da lunedì a venerdì), esse vengono esaurite in un totale di 10 settimane.\\
    Considerando la partenza del progetto da lunedì 25 aprile 2022 (necessariamente scostato rispetto all'assegnazione del 7 aprile per poter fare un periodo di formazione) il gruppo si aspetta di terminare i lavori per il giorno 1 luglio 2022, con un totale ore di gruppo di 700 ore.\\\\
    Sulla base di 70 ore settimanali:
    \subsection*{Responsabile}
    \textit{Ore settimanali: 4}.\\
    Si occupa di gestire il gruppo e il contatto con il proponente, è sempre presente ma il suo ruolo lo spende in velocità.
    \subsection*{Amministratore}
    \textit{Ore settimanali: 10}.\\
    Il gruppo pensa di dover assegnare più ore a questi ruolo perché deve gestire due repository (quello interno di AWS e quello pubblico) e i molteplici strumenti tecnologici di Amazon AWS (Comprehend, Rekognition, API Gateway, Lambda).
    \subsection*{Analista e Progettista}
    \textit{Ore settimanali : 20}.\\
    Abbiamo voluto dare la maggioranza delle ore alle figure dell'analista e del progettista perchè il progetto consiste, più che nella creazione di algoritmi e strumenti da zero, nel prendere tecnologie già esistenti e capire come usarle e farle comunicare al meglio al fine di raggiungere lo scopo. 
    \subsection*{Programmatore}
    \textit{Ore settimanali: 7}.\\
    Abbiamo voluto dare meno importanza alla figura del programmatore perchè in questo progetto si tratta di utilizzare per lo più strumenti esterni, pertanto il programmatore ha il solo compito di connettere queste tecnologie.\\ Non riteniamo che questo sia un lavoro che occuperà molto tempo se e solo se le figure di analisti e progettisti svolgono un lavoro completo nella fase precedente alla scrittura vera e propria del codice.
    \subsection*{Verificatore}
    \textit{Ore settimanali: 9}.\\
    Il verificatore ha un monte ore leggermente superiore a quella del programmatore perché abbiamo ritenuto di fondamentale importanza garantire la qualità del codice visto che la comunicazione tra tecnologie diverse può essere ostica e grande fonte di errori.

\end{document}
