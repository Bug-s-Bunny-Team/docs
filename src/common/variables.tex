\newcommand{\teamname}{\emph{Bug's Bunny}}
\newcommand{\progetto}{Guida Michelin Social}
\newcommand{\proponente}{Zero12}
\newcommand{\github}{https://github.com/Bug-s-Bunny-Team/docs}
\newcommand{\preventivo}{Preventivo Costi e Impegni Orari}
\newcommand{\email}{bugsbunnyteam@protonmail.com}

\newcommand{\angela}{Angela Arena}
\newcommand{\giulio}{Giulio Zanatta}
\newcommand{\ruth}{Ruth Genevieve Bousapnamene}
\newcommand{\tommaso}{Tommaso Di Fant}
\newcommand{\matteo}{Matteo Tossuto}
\newcommand{\marcob}{Marco Bellò}
\newcommand{\marcov}{Marco Volpato}

\newcommand{\RdP}{\emph{Responsabile di Progetto}}
\newcommand{\Amm}{\emph{Amministratore}}
\newcommand{\Ana}{\emph{Analista}}
\newcommand{\Prog}{\emph{Progettista}}
\newcommand{\Progm}{\emph{Programmatore}}
\newcommand{\progrs}{\emph{Programmatori}}
\newcommand{\Ver}{\emph{Verificatore}}
\newcommand{\verf}{\emph{Verificatori}}

\newcommand{\VdC}{\emph{Valutazione dei Capitolati}}
\newcommand{\NdP}{\emph{Norme di Progetto}}
\newcommand{\AdR}{\emph{Analisi dei Requisiti}}
\newcommand{\PdP}{\emph{Piano di Progetto}}
\newcommand{\PdQ}{\emph{Piano di Qualifica}}
\newcommand{\Glo}{\emph{Glossario}}

\newcommand{\aCapo}{ ~ \vspace{0.25cm} \\} %magari fare renewcommand di paragraph 

\newcommand{\G}{\textsubscript{G}}

% commandi presi da package tabularx, serve a impostare le varie opzioni delle tabelle senza doverle riscrivere ogni volta;
% anche se basterebbe la colonna centrale si scrivono comunque tutte e tre per buona norma;
% \raggedright, \centering o \raggedleft per avere l'allineamento orizzontale voluto;
% Dichiara \let\newline\\ per consentire l'uso di \newline per interruzioni di riga manuali; all'interno di una cella in modo coerente con il resto ( da notare che \centering e simili cambiano il significato di \\ );
% \arraybackslash per consentire di utilizzare \\ per le righe finali nelle tabelle;
% m{} serve a specificare larghezza colonna, in questo modo si può modificare 
\newcolumntype{L}[1]{>{\raggedright\let\newline\\\arraybackslash}m{#1}}
\newcolumntype{C}[1]{>{\centering\let\newline\\\arraybackslash}m{#1}}
\newcolumntype{R}[1]{>{\raggedleft\let\newline\\\arraybackslash}m{#1}}

\definecolor{Arancione}{RGB}{245,100,0}

% colori per highlighting
\definecolor{codegreen}{rgb}{0,0.6,0}
\definecolor{codegray}{rgb}{0.5,0.5,0.5}
\definecolor{codepurple}{rgb}{0.58,0,0.82}
\definecolor{backcolour}{rgb}{0.95,0.95,0.92}