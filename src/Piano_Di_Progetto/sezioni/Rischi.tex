L'analisi dei rischi è pensata come un progetto in divenire, modificato incrementalmente in caso
compaiano rischi non inizialmente preventivati. 
Stesura, riconoscimento e risoluzione degli stessi richiede costante monitoraggio e impegno, infatti
durante il corso del progetto potrebbero verificarsi problemi in grado di bloccare il lavoro per 
periodi di tempo anche prolungati. \\ Di seguito riportiamo delle tabelle per schematizzare i vari rischi riscontrabili, divise nelle 
seguenti sezioni:
\begin{itemize}
   \item Codice: identificazione del tipo di rischio che potrebbe causare problemi
   \item Descrizione: descrive di cosa tratta il tipo di rischio
   \item Notifica: i membri dichiarano difficoltà incontrate o problemi che pensano che potrebbero insorgere
   \item Grado: suddiviso in pericolosità e occorrenza viene valutato il rischio ( basso, medio, elevata)
   \item Gestione: vengono proposte soluzioni per i problemi e i rischi
\end{itemize}

   \subsection{Rischi Tecnologici}
   {
      \newcolumntype{L}[1]{>{\raggedright\let\newline\\\arraybackslash\hspace{0pt}}m{#1}}
      \newcolumntype{C}[1]{>{\centering\let\newline\\\arraybackslash\hspace{0pt}}m{#1}}
      \newcolumntype{R}[1]{>{\raggedleft\let\newline\\\arraybackslash\hspace{0pt}}m{#1}}

      \newlength{\freewidth}
      \setlength{\freewidth}{\dimexpr\textwidth-10\tabcolsep}
      \renewcommand{\arraystretch}{1.5}
      \centering
      \setlength{\aboverulesep}{0pt}
      \setlength{\belowrulesep}{0pt}
      \begin{longtable}{C{.13\freewidth} C{.15\freewidth} C{.26\freewidth} C{.18\freewidth} C{.282379\freewidth}}
         \toprule
      \rowcolor{Rosso}
      \textcolor{white}{\textbf{Versione}}&
      \textcolor{white}{\textbf{Data}}&
      \textcolor{white}{\textbf{Nominativo}}&
      \textcolor{white}{\textbf{Ruolo}}&
      \textcolor{white}{\textbf{Descrizione}}\\	
      \toprule
      \endhead

      0.0.1 & 16-11-2020 & \teamname{} & Analisti & Creazione bozza documento, introduzione e paragrafi. \\	
      \bottomrule
      \end{longtable}
}

   \subsection{Rischi Organizzativi}
   \subsection{Rischi Organizzativi}
   \subsection{Rischi Interpersonali}
