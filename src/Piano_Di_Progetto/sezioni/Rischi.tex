L'analisi dei rischi è pensata come un progetto in divenire, modificato incrementalmente in caso
compaiano rischi non inizialmente preventivati. 
Stesura, riconoscimento e risoluzione degli stessi richiede costante monitoraggio e impegno, infatti
durante il corso del progetto potrebbero verificarsi problemi in grado di bloccare il lavoro per 
periodi di tempo anche prolungati. 

Di seguito riportiamo delle tabelle per schematizzare i vari rischi riscontrabili, divise nelle 
seguenti sezioni:
   -Codice: identificazione del tipo di rischio che potrebbe causare problemi
   -Descrizione: descrive di cosa tratta il tipo di rischio
   -Notifica: i membri dichiarano difficoltà incontrate o problemi che pensano che potrebbero insorgere
   -Grado: suddiviso in pericolosità e occorrenza viene valutato il rischio ( basso, medio, elevata)
   -Gestione: vengono proposte soluzioni per i problemi e i rischi

   \subsection{Rischi Tecnologici}
   \subsection{Rischi Organizzativi}
   \subsection{Rischi Organizzativi}
   \subsection{Rischi Interpersonali}
