\documentclass{classes/base}

\title{Verbale esterno}
\date{2022/03/11}
\author{Nomone Cognomone}

\renewcommand{\maketitle}{
    \begin{titlepage}
    \begin{center}
        \makeatletter
        \vspace*{\fill}
        
        %\includegraphics[width=2.5cm]{assets/unipd}
        %\subsection*{Università degli Studi di Padova}
        %\vspace{2cm}
        
        \begin{minipage}[]{0.3\textwidth}
            \centering
            \includegraphics[width=3cm]{assets/unipd}
            \bigskip
        \end{minipage}
        \begin{minipage}[]{0.7\textwidth}
            \centering
            \color[HTML]{B5121B}{
                \textbf{Università degli Studi di Padova} \\
                Ingegneria del Software \\
                Anno Accademico: 2021/2022 \\
                }
                \vspace*{2cm}
        \end{minipage}
        

        \includegraphics[width=5cm]{assets/logo}

        \Huge
        \textbf{\teamname}
        
        \vspace{3cm}
        
        \Huge
        \textbf{\@title}

        \Large
        \@date

        \vspace{3cm}
        
        \textbf{Redazione:} \@author\\
        \textbf{Verifica:} \@verificatore\\
        \textbf{Approvazione:} \@approvatore\\
        
        \vfill
        \makeatother
    \end{center}
\end{titlepage}

}

\begin{document}
    \maketitle

    \section{Generalità}
    \begin{itemize}
        \item \textbf{Ora inizio:} 14.00
        \item \textbf{Ora fine:} 15.00
    \end{itemize}

    \section{Domande e risposte}
    \subsection*{Domande e risposte}
    
    \begin{itemize}
        \item  \textbf{Il repository è fornito o dobbiamo crearlo?}
        \\Il repository deve essere pubblico del team ed unico per documenti e sorgenti, è stato consigliato l'utilizzo di GitHub.
        \item  \textbf{Obbligatori lo sviluppo di app native mobile o si può usare un webapp?}
        \\È richiesta una Progressive Web App con disponibilità di dati offline, è stato consigliato di utilizzare \textit{Angular} piuttosto di \textit{React}.
        \item  \textbf{I linguaggi da utilizzare vengono decisi a monte?} 
        \\Linguaggi non obbligatori ma consigliati: \textit{Angular} (consigliato), \textit{NodeJS}, \textit{SpringBoot} per \textit{Java} e \textit{D3JS} per i grafici.
        \item  \textbf{Come viene effettuata l'analisi dei dati?}
        \\Vedere risorse fornite su analisi dei grafici di controllo, regole standard e implementazione a discrezione del team.
        \item  \textbf{Viene fornita formazione da parte dell'azienda? Se sì, come?}
        \\E’ prevista una formazione relativa alla parte documentale iniziale, non relativa alle tecnologie che si andranno ad utilizzare. Questa formazione iniziale è comunque ben apprezzata da parte di tutto il gruppo.
        \item  \textbf{Verrà fornito un server, anche di modeste capacità, dove poter fare delle prove?}
        \\In pratica non è necessario, al massimo per testare con molte macchine (+ di 100) problemi di sicurezza. Sul locale almeno 10 macchine concorrenti che aggiornano ogni 100ms “senza che il pc se ne accorga”.
        \item  \textbf{E' richiesto un focus verso alcuni tipi particolari di macchinari da monitorare?}
        \\Viene richiesto di focalizzarsi su macchine generiche, in particolare è stata posta l’attenzione sull’obiettivo vero e proprio del progetto, cioè la velocità (real time) di comunicazione dell’avviso. Il metodo di calcolo dell’anomalia che lancia l’avviso è lasciato alla fantasia del gruppo.
    \end{itemize}

    \section{Conclusioni}
    Il gruppo ha trovato il referente Beggiato molto professionale e disponibile a chiarire i nostri dubbi, ci ha fornito risposte esaustive e dettagliate, suscitando in noi interesse e fornendoci risposte esaustive sul capitolato stesso e sulle domande che erano emerse alla riunione del 9/3/22. 
    \\È stato comunque deciso di posporre la scelta del progetto a dopo la riunione con l'azienda zero12, sebbene positivamente colpito da San Marco Informatica.

\end{document}
