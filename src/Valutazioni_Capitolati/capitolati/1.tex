\subsubsection{Descrizione:}
Il capitolato propone lo sviluppo di un chatbot per aiutare le persone, in particolare i neoassunti, a interagire con la realtà aziendale.
La richiesta prevede che l’utente sia in grado di comunicare con il chatbot con un linguaggio naturale(italiano) per poter eseguire delle operazioni. \\
L’obiettivo è quello di rendere disponibile, tramite chatbot, le seguenti attività:
\begin{itemize}
	\item Consuntivare le attività giornaliere(obbligatorio);
	\item Tracciare le presenze in sede(obbligatorio);
	\item Aprire il cancello;
	\item Creare una nuova riunione;
	\item Ricercare documenti;
	\item Creare un ticket.
	
\end{itemize}

\subsubsection{Tecnologie:}
Il committente non raccomanda o consiglia nessuna tecnologia specifica ma dagli obiettivi del progetto possiamo intuire che:
\begin{itemize}
	\item Uso di linguaggi per creazione di App per Android o iOS
	\item Uso di protocollo MQTT per alcuni obiettivi
	\item Uso di CMIS per ricerca documenti
	\item Uso di API Rest per azioni di consuntivazione  
\end{itemize}

\subsubsection{Vincoli Generali:}
È richiesta la creazione di una App mobile (iOS o Android) per interagire con il chatbot
In forse(progetto deve essere documentato e dovrà garantire sicurezza del sistema e qualità codice e processi)


\subsubsection{Fattori Critici:}
Il progetto è risultato poco interessante ai membri del gruppo.

\subsubsection{Conclusioni:}
Dato il poco interesse da parte del nostro gruppo e il grande interesse di altri gruppi per questo capitolato, si è deciso di non considerarlo.