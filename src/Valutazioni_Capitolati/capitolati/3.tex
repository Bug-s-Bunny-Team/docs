\subsubsection{Descrizione:}
Il progetto si suddivide in 4 macrostrutture:
\begin{itemize}
	\item Web app, dedicata ad admin e utente del sistema che permetta di censire le macchine produttive e le relative caratteristiche da raccogliere e visualizzare;
	\item Creazione di un API per la raccolta dati (rest o GraphQL), in particolare per l'immissione della misurazione di una determinata caratteristica;
	\item Creazione di un motore di calcolo che alla ricezione di una nuova misurazione( da relativa API) si occupi di metterla in relazione con le misurazioni precedenti al fine di calcolare se la serie di punti evidenza un processo fuori controllo; 
	\item Creazione di Web app che permetta di selezionare una o più delle caratteristiche censite, a parità di macchina, e visualizzi, a rotazione la relativa carta di controllo.
\end{itemize}

\subsubsection{Tecnologie:}
Il committente consiglia l'utilizzo di determinati linguaggi e conoscenza di determinate tecnologie, tra cui:
\begin{itemize}
	\item \textbf{Java, NodeJS} per lo sviluppo delle API e il motore di calcolo;
	\item \textbf{Angular, React, Vue} per lo sviluppo della Web app;
	\item \textbf{d3js} per lo sviluppo dei grafici;
	\item \textbf{Sql / NoSql} per il salvataggio delle configurazioni generate dalla Web app;
	\item \textbf{Time-series DB} per il salvataggio delle informazioni prelevate dalla API.
\end{itemize}

\subsubsection{Vincoli Generali:}

\subsubsection{Fattori Critici:}

\subsubsection{Conclusioni:}
