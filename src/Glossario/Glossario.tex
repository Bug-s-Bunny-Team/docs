\documentclass{classes/base}
\usepackage{hyperref}
\usepackage{glossaries}

\title{Glossario}
\author{\giulio}
\verificatore{\marcov}
\approvatore{\matteo}
\uso{Esterno}

\begin{document}
	\maketitle
	\newpage
	\tableofcontents
	
	\addcontentsline{toc}{section}{A}
    \addcontentsline{toc}{section}{B}
    \addcontentsline{toc}{section}{C}
    \addcontentsline{toc}{section}{D}
    \addcontentsline{toc}{section}{E}
    \addcontentsline{toc}{section}{F}
    \addcontentsline{toc}{section}{G}
    \addcontentsline{toc}{section}{H}
    \addcontentsline{toc}{section}{I}
    \addcontentsline{toc}{section}{J}
    \addcontentsline{toc}{section}{K}
    \addcontentsline{toc}{section}{L}
    \addcontentsline{toc}{section}{M}
    \addcontentsline{toc}{section}{N}
    \addcontentsline{toc}{section}{O}
    \addcontentsline{toc}{section}{P}
    \addcontentsline{toc}{section}{Q}
    \addcontentsline{toc}{section}{R}
    \addcontentsline{toc}{section}{S}
    \addcontentsline{toc}{section}{T}
    \addcontentsline{toc}{section}{U}
    \addcontentsline{toc}{section}{V}
    \addcontentsline{toc}{section}{W}
    \addcontentsline{toc}{section}{X}
    \addcontentsline{toc}{section}{Y}
    \addcontentsline{toc}{section}{Z}

    \section*{A}
        \subsection*{API}
        Application Program Interface, è un'interfaccia che offre servizi a computer o software (non ha quindi interazione con utenti).
       
        \subsection*{AWS}
        Amazon Web Services, è una suite di applicativi amazon dedicati alla gestione e sviluppo di applicativi web.

        \subsection*{Amazon API Gateway}
        Amazon API Gateway è un servizio AWS per la creazione, la pubblicazione, la gestione, il monitoraggio e la protezione di API.

        \subsection*{Amazon Appsync}
        AWS AppSync è un servizio completamente gestito che facilita lo sviluppo di API gestendo le attività impegnative derivanti dalla connessione sicura a origini dati come AWS DynamoDB, AWS Lambda.

        \subsection*{Amazon Cognito}
        Amazon Cognito fornisce autenticazione, autorizzazione e gestione degli utenti per le app Web e per dispositivi mobili. Gli utenti possono accedere direttamente con un nome utente e una password, oppure tramite terze parti, ad esempio Facebook, Amazon, Google o Apple.
        
        \subsection*{Amazon CloudFront}
        Amazon CloudFront è un servizio Web che accelera la distribuzione di contenuto Web statico e dinamico, come file immagine, .html, .css e .js, agli utenti.

        \subsection*{Amazon DynamoDB}
        Database fornito da AWS, NoSQL. Esso permette agli sviluppatori di caricare qualsiasi volume di dati garantendo prestazioni elevate per l'esecuzione degli applicativi web.

        \subsection*{Amazon EC2}
        Amazon Elastic Compute Cloud (Amazon EC2) fornisce capacità di calcolo scalabile in AWS Cloud. L'utilizzo Amazon EC2 elimina la necessità di investimenti anticipati in hardware e permette di sviluppare e distribuire più rapidamente le applicazioni.

        \subsection*{Amazon Fargate}
        Amazon Fargate è una tecnologia che permette l'esecuzione di container senza doversi occupare della gestione e della configurazione delle relative macchine virtuali presenti nei server/cluster. 

        \subsection*{Amazon Lambda}
        Servizio di AWS che permette di eseguire codici per qualsiasi tipo di applicazione o servizio senza doversi occupare della gestione del server.

        \subsection*{Amazon RDS}
        Amazon Relational Database Service (Amazon RDS) è un servizio Web che semplifica la configurazione, l'uso e il dimensionamento di un database relazionale in AWS Cloud.

        \subsection*{Amazon Rekognition}
        Servizio di AWS che semplifica l’aggiunta di analisi delle immagini nelle applicazioni. Può rilevare molteplici cose tra cui oggetti, scene e volti.

        \subsection*{Amazon S3}
        Servizio di AWS per l'archiviazione di oggetti che offre notevole scalabilità, disponibilità dei dati sicurezza e prestazioni. Offre la possibilità di archiviare facilmente qualsiasi quantità di dati per qualsiasi caso d'uso.

        \subsection*{Amazon VPC}
        Amazon Virtual Private Cloud (Amazon VPC) offre il pieno controllo sull' ambiente di reti virtuali, inclusi posizionamento della risorsa, connettività e sicurezza.

        \subsection*{Angular} 
        Framework open source (inserirle nel glossario?) per lo sviluppo di applicazioni web.

        \subsection*{AZ}
        Availability Zone
    
        \newpage  
    \section*{B}
        
        \subsection*{Best practices} 
        Si tratta delle pratiche, abitudini e comportamenti che è buona norma seguire per svolgere un lavoro.

        \subsection*{Branch} 
        Letteralmente “ramo”. Viene usato per isolare il lavoro di sviluppo senza influire su altri nel repository. 
        
        \newpage  
    \section*{C}
        \subsection*{Casi d’uso} 
        E' un tecnica utilizzata per raccogliere requisiti, in maniera esaustiva e dettagliata, al fine di produrre software qualitativo.

        \subsection*{CDN}
        Content Delivery Network, è un gruppo di server distribuiti in tutto il mondo, in modo da coprire più aree geografiche avvicinandosi alle posizioni degli utenti in modo da velocizzare il delivery dei contenuti web.
    
        \subsection*{CIDR}
        Classless Inter Domain Routing, metodo che sfrutta la subnet mask per creare una sottorete partendo da un indirizzo IP.  Questo metodo sostituisce lo schema classful dove gli indirizzi dovevano appartenere ad una specifica classe (A, B e C).
    
        \subsection*{CI/CD}
        Continuous Integration Continuous Deployment, il metodo CI/CD introduce l'automazione costante e il monitoraggio continuo in tutto il ciclo di vita delle applicazioni, dalle fasi di integrazione e test a quelle di distribuzione e deployment.
    
        \subsection*{Cloud}
        Insieme di server distribuiti connessi tra loro tramite un'architettura distribuita. Il loro scopo è quello di erogare servizi come l'archiviazione, l'elaborazione, la trasmissione dati, l'esecuzione di applicazione tramite Internet.
        
        \subsection*{Cluster}
        Un cluster è un insieme di computer connessi tra di loro tramite una rete, con lo scopo di distribuire un'elaborazione molto complessa tra i vari computer, aumentando la potenza di calcolo del sistema e/o garantendo una maggiore disponibilità di servizio.
    
        \subsection*{Commit}
        Una serie di modifiche ad un documento. 
        
        \newpage  
    \section*{D}
       
        \subsection*{Discord}
        Piattaforma di VoIp, progettata per la comunicazione. Gli utenti possono comunicare tramite chiamate vocali, videochiamate, messaggi di testo, media e file in chat privata come membri di un server Discord.

        \subsection*{DDOS}
        Un attacco di tipo Distributed Denial of Service (DDoS) è un'arma di sicurezza informatica che mira a interrompere le attività aziendali o estorcere denaro alle organizzazioni prese di mira.

        \subsection*{Deploy}
        Il termine deployment in informatica significa distribuzione, ovvero la consegna, l'installazione, la configurazione e la messa in funzione di una applicazione in un sistema informatico.

        \subsection*{Design}
        Il termine design in informatica si riferisce al processo di progettazione di un applicazione software.

        \subsection*{DataBase}
        Un database è un insieme di informazioni (o dati) strutturate in genere archiviate elettronicamente in un sistema informatico.
        \newpage  
    \section*{E}
    
        \newpage  
    \section*{F}
        \subsection*{Framework}
        
        \newpage  
    \section*{G}
        
        \subsection*{Gateway}
        In particolare nelle reti inforatiche è un dispositivo di rete che collega due reti di diverso tipo.

        \subsection*{GitHub}
        Servizio di hosting per progetti software ed è una implementazione dello strumento di controllo versione distribuito Git.

        \subsection*{GitKraken}
        È la base per gli sviluppatori che cercano un'interfaccia più adatta a Git, con interazioni a GitHub. 

        \subsection*{Git}
        Git è un software per il controllo di versione distribuito utilizzabile da interfaccia a riga di comando, creato da Linus Torvalds

        \subsection*{GraphQL}
        Linguaggio per interrogare lato server per API, è in grado di fornire ai client solo idati di cui si ha bisogno.

        \subsection*{Guida Michelin} 
        Insieme di pubblicazioni annuali rivolte all’enogastronomia di un determinato paese. È un riferimento per la valutazione della qualità dei ristoranti.
        \newpage  
    \section*{H}
    \newpage  
    \section*{I}
        \subsection*{Issue}
        Compito da svolgere all'interno del repo

        \subsection*{Interfaccia}
        Collegamento tra sistemi diversi per permettere loro lo scambio di informazioni.
        \subsection*{IP}
        Internet Protocol, protocollo di rete TCP/IP che si occupa dell'instradamento delle informazioni.
        \newpage  
    \section*{J}
    \newpage  
    \section*{K}
    \newpage  
    \section*{L}
    \newpage  
    \section*{M} 
        \subsection*{Metrica}

        \subsection*{Milestone}
        In inglese significa pietra miliare, ed indica un traguardo/obbiettivo presente nello svolgimento del progetto.
        \subsection*{Multi AZ} 

        \subsection*{MySQL}
        Sistema di gestione di database relazionali gratuito ed open source che utilizza SQL.
        \newpage  
    \section*{N}
        \subsection*{NAT}
        Network Address Translation, metodo con il quale si può modificare l'indirizzo sorgente/destinatario di un pacchetto nella comunicazione tra due client posti in reti diverse.

        \subsection*{Norme di Progetto (NdP)} 

        \subsection*{Neptune}

        \subsection*{Network}
        Sostanzialmente una rete di host collegati e in grado di comunicare tra loro.
        \subsection*{NodeJS}
        Ambiente di esecuzione che permette di eseguire codice Javascript come un qualisasi linguaggio di programmazione.
        \newpage  
    \section*{O}
        \subsection*{Open Source}
        Programma del quale è disponibile pubblico il codice sorgente 
        \newpage  
    \section*{P}
        \subsection*{Piano di progetto (PdP)}

        \subsection*{Piano di qualifica (PdQ)} 

        \subsection*{Pipeline}
        Indica una catena di montaggio, in particolare dei componenti software collegati tra loro a cscata in modo che il risultato prodotto da un elemento sia l'ingresso di quello successivo.
        \subsection*{Plugin}
        Software non autonomo, interagisce con un altra applicazione per estenderne le capacità.
        \subsection*{PostgreSQL}
        \newpage  
    \section*{Q}
    \newpage  
    \section*{R}
        \subsection*{Ranking} 
        Classifica, graduatoria di merito. 

        \subsection*{Repository}
        Area di gestione principale in cui sono contenuti tutti i progetti e che aiuta a gestire il flusso di lavoro.
        
        \subsection*{React}
        Libreria Javascript indirizzata alla creazione di interfacce utente.
        \subsection*{REST}
        
        \subsection*{RTB}
        Requirements Technology Baseline
        \newpage  
    \section*{S}
        \subsection*{Scaling}       

        \subsection*{Schedule} 
        Il termine significa orgnizzare, pianificare eventi.
        \subsection*{Schedulato}
        Il termine si utilizza per indicare un evento preventivamente organizzato e prefissato.
        \subsection*{Secret Manager}

        \subsection*{Slack} 
        Applicazione che permette di inviare messaggi in tempo reale ai membri del gruppo e viene usato come strumento di collaborazione aziendale.

        \subsection*{Software Suite} 
        Collezione di programmi e applicativi correlati tra loro.

        \subsection*{Stakeholder} 
        Dall'inglese portatore d'interesse, si riferisce ai soggetti direttamente interessati ed altamente coinvolti nel buon andamento dell'applicazione.
        \subsection*{Subnet}
        E' divisione logica di rete IP, e ulteriore partizione di una vpc.
        \newpage  
    \section*{T}
        \subsection*{Telegram}
        Applicazione che permette di inviare messaggi ed effettuare chiamate a più utenti contemporaneamente in tempo reale, creando anche dei gruppi privati.

        \subsection*{Terminal} 
        Interfaccia a linea di comando per interagire con il sistema operativo 
        \newpage  
    \section*{U}
        \subsection*{UML}

        \subsection*{Use Cases} 
        Si veda casi d’uso.
        \newpage  
    \section*{V}
        \subsection*{Valutazione capitolati}

        \subsection*{VPC}
        Virtual Private Cloud.
        \newpage  
    \section*{W}
        \subsection*{WOW - Way Of Working}
        Traducibile con “il modo in cui un gruppo decide di lavorare” ed è un elemento fondamentale per aumentare la probabilità di successo di un team nella realizzazione di un progetto o per il raggiungimento di un obiettivo.  
        \newpage  
    \section*{X}
    \newpage  
    \section*{Y}
        \subsection*{YAML}
        
        \newpage  
    \section*{Z}
\end{document}
