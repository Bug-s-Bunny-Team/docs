\documentclass{classes/base}
\usepackage{hyperref}
\usepackage{glossaries}

\title{Glossario}
\author{\giulio}
\verificatore{\marcov}
\approvatore{\matteo}
\uso{Esterno}

\begin{document}
	\maketitle
	\newpage
	\tableofcontents
	
	\addcontentsline{toc}{section}{A}
    \addcontentsline{toc}{section}{B}
    \addcontentsline{toc}{section}{C}
    \addcontentsline{toc}{section}{D}
    \addcontentsline{toc}{section}{E}
    \addcontentsline{toc}{section}{F}
    \addcontentsline{toc}{section}{G}
    \addcontentsline{toc}{section}{H}
    \addcontentsline{toc}{section}{I}
    \addcontentsline{toc}{section}{J}
    \addcontentsline{toc}{section}{K}
    \addcontentsline{toc}{section}{L}
    \addcontentsline{toc}{section}{M}
    \addcontentsline{toc}{section}{N}
    \addcontentsline{toc}{section}{O}
    \addcontentsline{toc}{section}{P}
    \addcontentsline{toc}{section}{Q}
    \addcontentsline{toc}{section}{R}
    \addcontentsline{toc}{section}{S}
    \addcontentsline{toc}{section}{T}
    \addcontentsline{toc}{section}{U}
    \addcontentsline{toc}{section}{V}
    \addcontentsline{toc}{section}{W}
    \addcontentsline{toc}{section}{X}
    \addcontentsline{toc}{section}{Y}
    \addcontentsline{toc}{section}{Z}

    \section*{A}
        \subsection*{API}
        Application Program Interface, è un'interfaccia che offre servizi a computer o software (non ha quindi interazione con utenti).
       
        \subsection*{AWS}
        Amazon Web Services, è una suite di applicativi amazon dedicati alla gestione e sviluppo di applicativi web.

        \subsection*{Amazon API Gateway}

        \subsection*{Amazon Appsync}

        \subsection*{Amazon Cognito}

        \subsection*{Amazon Cloud Front}

        \subsection*{Amazon EC2}

        \subsection*{Amazon Fargate}

        \subsection*{Amazon Lambda}

        \subsection*{Amazon RDS}

        \subsection*{Amazon Rekognition}

        \subsection*{Amazon S3}

        \subsection*{Amazon VPC}

        \subsection*{Angular} 
        Framework open source (inserirle nel glossario?) per lo sviluppo di applicazioni web.

        \subsection*{AZ}
        Availability Zone
    
        \newpage  
    \section*{B}
        
        \subsection*{Best practices} 
        Si tratta delle pratiche, abitudini e comportamenti che è buona norma seguire per svolgere un lavoro.

        \subsection*{Branch} 
        Letteralmente “ramo”. Viene usato per isolare il lavoro di sviluppo senza influire su altri nel repository. 
        \newpage  
    \section*{C}
        \subsection*{Casi d’uso} 

        \subsection*{CDN}
        Content Delivery Network
    
        \subsection*{CIDR}
        Classless Inter Domain Routing, 
    
        \subsection*{CI/CD}
        Continuous Integration Continuous Deployment 
    
        \subsection*{Cloud}
    
        \subsection*{Commit}
        Una serie di modifiche ad un documento. 
        \newpage  
    \section*{D}
       
        \subsection*{Discord}
        Piattaforma di VoIp, progettata per la comunicazione. Gli utenti possono comunicare tramite chiamate vocali, videochiamate, messaggi di testo, media e file in chat privata come membri di un server Discord.

        \subsection*{DDOS}

        \subsection*{Deploy}

        \subsection*{Design}

        \subsection*{DataBase}
        \newpage  
    \section*{E}
    \newpage  
    \section*{F}
        \subsection*{Framework}
        \newpage  
    \section*{G}
        
        \subsection*{Gateway}

        \subsection*{GitHub}
        Servizio di hosting per progetti software ed è una implementazione dello strumento di controllo versione distribuito Git.

        \subsection*{GitKraken}
        E' la base per gli sviluppatori che cercano uninterfaccia più adatta a Git, con interazioni a GitHub. 

        \subsection*{Git}
        Git è un software per il controllo di versione distribuito utilizzabile da interfaccia a riga di comando, creato da Linus Torvalds

        \subsection*{GraphQL}

        \subsection*{Guida Michelin} Insieme di pubblicazioni annuali rivolte all’enogastronomia di un determinato paese. È un riferimento per la valutazione della qualità dei ristoranti.
        \newpage  
    \section*{H}
    \newpage  
    \section*{I}
        \subsection*{Issue}
        Compito da svolgere all'interno del repo

        \subsection*{Interfaccia}
        
        \subsection*{IP}
        \newpage  
    \section*{J}
    \newpage  
    \section*{K}
    \newpage  
    \section*{L}
    \newpage  
    \section*{M} 
        \subsection*{Metrica}

        \subsection*{Milestone}

        \subsection*{Multi AZ} 

        \subsection*{MySQL}
        \newpage  
    \section*{N}
        \subsection*{NAT}
        Network Address Translation

        \subsection*{Norme di Progetto (NdP)} 

        \subsection*{Neptune}

        \subsection*{Network}

        \subsection*{NodeJS}
        \newpage  
    \section*{O}
        \subsection*{Open Source}
        Programma del quale è disponibile pubblico il codice sorgente 
        \newpage  
    \section*{P}
        \subsection*{Piano di progetto (PdP)}

        \subsection*{Piano di qualifica (PdQ)} 

        \subsection*{Pipeline}

        \subsection*{Plugin}

        \subsection*{PostgreSQL}
        \newpage  
    \section*{Q}
    \newpage  
    \section*{R}
        \subsection*{Ranking} 
        Classifica, graduatoria di merito. 

        \subsection*{Repository}
        Area di gestione principale in cui sono contenuti tutti i progetti e che aiuta a gestire il flusso di lavoro.
        
        \subsection*{React}
        
        \subsection*{REST}
        
        \subsection*{RTB}
        Requirements Technology Baseline
        \newpage  
    \section*{S}
        \subsection*{Scaling}       

        \subsection*{Schedule} 

        \subsection*{Schedulato}

        \subsection*{Secret Manager}

        \subsection*{Slack} 
        Applicazione che permette di inviare messaggi in tempo reale ai membri del gruppo e viene usato come strumento di collaborazione aziendale.

        \subsection*{Software Suite} 
        Collezione di programmi e applicativi correlati tra loro.

        \subsection*{Stakeholder} 

        \subsection*{Subnet}
        E' divisione logica di rete IP, e ulteriore partizione di una vpc.
        \newpage  
    \section*{T}
        \subsection*{Telegram}
        Applicazione che permette di inviare messaggi ed effettuare chiamate a più utenti contemporaneamente in tempo reale, creando anche dei gruppi privati.

        \subsection*{Terminal} 
        Interfaccia a linea di comando per interagire con il sistema operativo 
        \newpage  
    \section*{U}
        \subsection*{UML}

        \subsection*{Use Cases} 
        Si veda casi d’uso.
        \newpage  
    \section*{V}
        \subsection*{Valutazione capitolati}

        \subsection*{VPC}
        Virtual Private Cloud.
        \newpage  
    \section*{W}
        \subsection*{}
        Traducibile con “il modo in cui un gruppo decide di lavorare” ed è un elemento fondamentale per aumentare la probabilità di successo di un team nella realizzazione di un progetto o per il raggiungimento di un obiettivo.  
        \newpage  
    \section*{X}
    \newpage  
    \section*{Y}
        \subsection*{YAML}
        \newpage  
    \section*{Z}
\end{document}
